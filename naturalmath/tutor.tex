% File tutor.tex created from tutor.nat
% by the program Natural Math version 0.5

\documentclass[12pt]{article}

\begin{document}

\noindent \def\titleblock
{
Tutor for Natural Math version 0.5\\
last modified July 21, 2001\\
Copyright 1999 Stephen J Montgomery-Smith.  All rights reserved.
}
% This package is free software; you can redistribute it and/or modify
% it under the terms of either:
%
%     a) the GNU General Public License as published by the Free
%     Software Foundation; either version 2, or (at your option) any
%     later version, or
%
%     b) "Stephen's Artistic License" which comes with this Kit.
%
% This package is distributed in the hope that it will be useful,
% but WITHOUT ANY WARRANTY; without even the implied warranty of
% MERCHANTABILITY or FITNESS FOR A PARTICULAR PURPOSE.  See either
% the GNU General Public License or Stephen's Artistic License for more 
% details.
%
% You should have received a copy of Stephen's Artistic License with this
% package, in the file named "Stephens-Artistic.txt".  If not, I'll be glad
% to provide one.
%
% You should also have received a copy of the GNU General Public License
% along with this program; if not, write to the Free Software
% Foundation, Inc., 675 Mass Ave, Cambridge, MA 02139, USA.

\noindent \begin{center}
\Large \bf The Natural Math Program
\end{center}
\begin{center}
\large \bf Stephen Montgomery-Smith
\end{center}
\begin{center}\small\titleblock\end{center}

\noindent \section{Introduction}

\noindent \indent
Here we describe the Natural Math program.  It is easy to use.
Start with a file whose extension is {\tt .nat}, for example,
{\tt test.nat}.  This tutorial was created by the file
{\tt tutor.nat}.

\noindent \indent
Each line of your file {\tt xxx.nat} is written in what we call ``natural
math,'' that is, math written as you might naturally express it if you
only had a simple typewriter.
You will use numbers, letters, and symbols, although anything that
can be expressed in symbols can also almost always be expressed in letters.

\noindent \indent
What the program will do is to convert the natural math file into a La\TeX\
file.  You run it like this:
\begin{center} {\tt naturalmath xxx.nat} \end{center}
This will create a file {\tt xxx.tex}.  
%Indeed, as it stands right now,
%it will also run La\TeX, creating the file {\tt xxx.dvi}, and then go on
%to run {\tt dvips} to create the postscript file {\tt xxx.ps}.

\noindent \indent
Here is an example of lines of input, followed by the output that would
be created.

\begin{verbatim}
integral from 0 to infinity of e ^ (-x^2/2) dx 
= sqrt (pi over 2)
\end{verbatim}

\[
\int _ {0} ^ {\infty} {e} ^ { - {x} ^ {2} / 2} \,d x = \sqrt{\frac {\pi} {2}}
\]

\noindent Each formula is created by a sequence of such lines, terminated by a 
a blank line.  Let us first give an example, where we attempt to solve
a homework problem.  First we give the input, then the output.

\newpage

\noindent \section{Example} \label{example}

\noindent \begin{verbatim}# Start the question
\end{verbatim}

\noindent \begin{verbatim}
text Chapter 8.6 Question 25
\end{verbatim}

\noindent \begin{verbatim}
text Evaluate the following sum
\end{verbatim}

\noindent \begin{verbatim}
sum from n = 2 to infinity of 1 over (n^2 - 1)
\end{verbatim}

\noindent \begin{verbatim}
text Answer: use partial fractions
\end{verbatim}

\noindent \begin{verbatim}
n^2 - 1 = (n-1)(n+1)
\end{verbatim}

\noindent \begin{verbatim}
1 over (n^2 - 1) = A over (n-1) + B over (n + 1)
\end{verbatim}

\noindent \begin{verbatim}
= (A(n+1) + B (n-1)) over ((n-1)(n+1))
\end{verbatim}

\noindent \begin{verbatim}
1 = A n + A + B n - B
\end{verbatim}

\noindent \begin{verbatim}
text Equate coefficients
\end{verbatim}

\noindent \begin{verbatim}
0 = A - B
\end{verbatim}

\noindent \begin{verbatim}
1 = A + B
\end{verbatim}

\noindent \begin{verbatim}
text add equations
\end{verbatim}

\noindent \begin{verbatim}
1 = 2A
\end{verbatim}

\noindent \begin{verbatim}
A = 1 over 2
\end{verbatim}

\noindent \begin{verbatim}
B = - 1 over 2
\end{verbatim}

\noindent \begin{verbatim}
1 over (n squared - 1) = 1 over (2(n-1)) - 1 over (2(n+1))
\end{verbatim}

\noindent \begin{verbatim}
S _ N = sum from n = 2 to N of 1 over (n^2 - 1)
\end{verbatim}

\noindent \begin{verbatim}
=
(1 over 2 - 1 over 6) + (1 over 4 - 1 over 8) + (1 over 6 - 1 over 10)
+ (1 over 8 - 1 over 12)
+ ... +
\end{verbatim}

\noindent \begin{verbatim}
+ (1 over (2(N-3)) - 1 over (2(N-1)) )
+ (1 over (2(N-2)) - 1 over (2N) )
\end{verbatim}

\noindent \begin{verbatim}
+ (1 over (2(N-1)) - 1 over (2(N+1)) )
\end{verbatim}

\noindent \begin{verbatim}
=
1 over 2 + 1 over 4 - 1 over (2N) - 1 over (2(N+1))
\end{verbatim}

\noindent \begin{verbatim}
limit as N to infinity of S_N = 3 over 4
\end{verbatim}

\newpage

\noindent Chapter 8.6 Question 25

\noindent Evaluate the following sum

\[
\sum _ {n = 2} ^ {\infty} \frac {1} {{n} ^ {2} - 1}
\]

\noindent Answer: use partial fractions

\[
{n} ^ {2} - 1 = \left( n - 1 \right) \left( n + 1 \right)
\]

\[
\frac {1} {{n} ^ {2} - 1} = \frac {A} {n - 1} + \frac {B} {n + 1}
\]

\[
 = \frac {A \left( n + 1 \right) + B \left( n - 1 \right)} {\left( n - 1 \right) \left( n + 1 \right)}
\]

\[
1 = A n + A + B n - B
\]

\noindent Equate coefficients

\[
0 = A - B
\]

\[
1 = A + B
\]

\noindent add equations

\[
1 = 2 A
\]

\[
A = \frac {1} {2}
\]

\[
B =  - \frac {1} {2}
\]

\[
\frac {1} {{n} ^ 2 - 1} = \frac {1} {2 \left( n - 1 \right)} - \frac {1} {2 \left( n + 1 \right)}
\]

\[
{S} _ {N} = \sum _ {n = 2} ^ {N} \frac {1} {{n} ^ {2} - 1}
\]

\[
 = \left( \frac {1} {2} - \frac {1} {6} \right) + \left( \frac {1} {4} - \frac {1} {8} \right) + \left( \frac {1} {6} - \frac {1} {10} \right) + \left( \frac {1} {8} - \frac {1} {12} \right) + \dots + 
\]

\[
 + \left( \frac {1} {2 \left( N - 3 \right)} - \frac {1} {2 \left( N - 1 \right)} \right) + \left( \frac {1} {2 \left( N - 2 \right)} - \frac {1} {2 N} \right)
\]

\[
 + \left( \frac {1} {2 \left( N - 1 \right)} - \frac {1} {2 \left( N + 1 \right)} \right)
\]

\[
 = \frac {1} {2} + \frac {1} {4} - \frac {1} {2 N} - \frac {1} {2 \left( N + 1 \right)}
\]

\[
\lim _ {N \to \infty} {S} _ {N} = \frac {3} {4}
\]

\newpage

\noindent \section{The basic commands}

\noindent \subsection{Numbers and Variables}

\noindent \indent
Natural Math allows numbers and one letter variables written in the usual
way.  There is also the complete collection of greek letters, both
lower and upper case, and infinity.  There are also
the dots.

\begin{verbatim}
30.45 , x , pi, Pi, Phi, phi, infinity, ...,dots
\end{verbatim}

\[
30.45 , \ x , \ \pi , \ \Pi , \ \Phi , \ \phi , \ \infty , \ \dots , \ \dots
\]

\noindent \subsection{Operations}
Natural Math supports a large set of operations from mathematics. 
The arithmetic operators:

\begin{verbatim}
a + b, a plus b, a - b, a minus b,
\end{verbatim}

\[
a + b , \ a + b , \ a - b , \ a - b , \ 
\]

\begin{verbatim}
a * b, a times b, a / b, a divide b,
\end{verbatim}

\[
a \times b , \ a \times b , \ a / b , \ a \div b , \ 
\]

\begin{verbatim}
a ^ b, a power b, a . b, a dot b
\end{verbatim}

\[
{a} ^ {b} , \ {a} ^ {b} , \ a \cdot b , \ a \cdot b
\]

\noindent The fraction operator, and the implicit multiplication operator:
(in the case of the implicit multiplication operator, the space between
the two quantities can be crucual if they are both letter variables or 
numbers):

\begin{verbatim}

a over b, a b
\end{verbatim}

\[
\frac {a} {b} , \ a b
\]

\noindent The relational operators:

\begin{verbatim}
a = b, a eq b, a <= b, a le b, a >= b, a > b,
\end{verbatim}

\[
a = b , \ a = b , \ a \le b , \ a \le b , \ a \ge b , \ a > b , \ 
\]

\begin{verbatim}
a <> b, a ne b, a < b, a lt b, a > b, a gt b
\end{verbatim}

\[
a \ne b , \ a \ne b , \ a < b , \ a < b , \ a > b , \ a > b
\]

\noindent Other operators (the last one tells you that the comma is considered
as an operator):

\begin{verbatim}
a _ b, a sub b, a subst b, a -> b, a to b, a tendsto b, 
a , b, a comma b
\end{verbatim}

\[
{a} _ {b} , \ {a} _ {b} , \ \left.a\right|_{b} , \ a \to b , \ a \to b , \ a \to b , \ a , \ b , \ a , \ b
\]

\noindent The plus and minus can also appear at the beginning of some expressions:

\begin{verbatim}
a * (-b) , a^+b
\end{verbatim}

\[
a \times \left(  - b \right) , \ {a} ^ { + b}
\]

\noindent Operations can appear right at the beginning of the formula, like
comma, the relational operators, and plus/minus.

\begin{verbatim}
= a + b
\end{verbatim}

\[
 = a + b
\]

\noindent Also, an operation can be left `dangling' at the end of input:

\begin{verbatim}
a+
\end{verbatim}

\[
a + 
\]

\noindent The value of these last two allowable activities is to let long formulae
range over several lines.  This is illustrated in the long example given
is Section~\ref{example}.

\noindent \indent
Finally, the operations plus and minus may be used in a contex where they
are treated as quantities.  This allows expressions like

\begin{verbatim}
a to 4^+ , b = 3_-
\end{verbatim}

\[
a \to {4} ^ {+} , \ b = {3} _ {-}
\]

\noindent \subsection{Order of Operations and Brackets}

\noindent \indent
Natural Math does a careful analysis, pulling apart the expressions
so as to figure out what comes first.  So in the following example,
the division is done before the addition.

\begin{verbatim}
a + b over c
\end{verbatim}

\[
a + \frac {b} {c}
\]

\noindent You can change the order of operations: in this example the addition is
done before the division.

\begin{verbatim}
(a + b) over c
\end{verbatim}

\[
\frac {a + b} {c}
\]

\noindent Here is another example.

\begin{verbatim}
(a+b) times c
\end{verbatim}

\[
\left( a + b \right) \times c
\]

\noindent In this last example, the brackets appeared when typeset.  Usually, brackets
written in will appear as you wrote them: 

\begin{verbatim}
(a b)c
\end{verbatim}

\[
\left( a b \right) c
\]

\noindent In a few cases, brackets are needed to change the natural order of doing
operations, but it would not be right to typeset them.  This happens
with fractions (as above), and also with powers and subscripts.  
(It also happens with the square root and absolute value, and with
limits of integration, and with the substitution operator --- see below.)
However,
you can always force the brackets to appear in this situation by adding an
extra pair of unneccesary brackets:

\begin{verbatim}
a^(b+c), a_(b+c)
\end{verbatim}

\[
{a} ^ {b + c} , \ {a} _ {b + c}
\]

\begin{verbatim}
((x+y)) over ((x^2 - y^2)) , f^((2))(x)
\end{verbatim}

\[
\frac {\left( x + y \right)} {\left( {x} ^ {2} - {y} ^ {2} \right)} , \ {f} ^ {\left( 2 \right)} \left( x \right)
\]

\noindent There are also square brackets:

\begin{verbatim}
[ x over y ]
\end{verbatim}

\[
\left[ \frac {x} {y} \right]
\]

\noindent What is the order in which natural math would evaluate the operations without
brackets?  First powers and subscripts.  Then fractions.  Then multiplication
and division.  Then addition and subtraction.  Then the `tends to'.  Then
the `comma'.  Finally the relational operators.  Otherwise the operations
are performed left to right, except that the power and subscript operators
are performed from right to left.

\begin{verbatim}
a^b^c^d , a_b_c_d
\end{verbatim}

\[
{a} ^ {{b} ^ {{c} ^ {d}}} , \ {a} _ {{b} _ {{c} _ {d}}}
\]

\noindent Here is an example with the substitution operator.  Notice that in this case,
a rather large number of brackets is needed.  Natural Math has its limitations!

\begin{verbatim}
((df^-1(x)) over dx) subst (x=f(a))
=
1 over (((df(x)) over dx) subst (x = a))
\end{verbatim}

\[
\left.\frac {{\,d f} ^ { - 1} \left( x \right)} {\,d x}\right|_{x = f \left( a \right)} = \frac {1} {\left.\frac {\,d f \left( x \right)} {\,d x}\right|_{x = a}}
\]

\noindent \subsection{Functions}

\noindent \indent
Natural Math supports a range of functions

\begin{verbatim}
sqrt a, abs a, |a|, a squared, a !, a factorial
\end{verbatim}

\[
\sqrt{a} , \ \left| a \right| , \ \left| a \right| , \ {a} ^ 2 , \ a ! , \ a !
\]

\noindent and the trig and hyperbolic functions:

\begin{verbatim}
sin, cos, tan, sec, csc, cot, 
arcsin, arccos, arctan, 
sinh, cosh, tanh, coth
\end{verbatim}

\[
\sin , \ \cos , \ \tan , \ \sec , \ \csc , \ \cot , \ \arcsin , \ \arccos , \ \arctan , \ \sinh , \ \cosh , \ \tanh , \ \coth
\]

\noindent and functions that you can create yourself, either by using quotes, or by
using the ``textsymb'' command:

\begin{verbatim}
"sech"(x) = textsym sech(x) = 2 over (e^x + e^-x)
\end{verbatim}

\[
\hbox{sech} \left( x \right) = \hbox{sech} \left( x \right) = \frac {2} {{e} ^ {x} + {e} ^ { - x}}
\]

\noindent Some of
these functions interact with brackets in interesting ways:

\begin{verbatim}
sqrt(a+b) , sqrt((a+b)) , abs(a+b) , abs((a+b))
\end{verbatim}

\[
\sqrt{a + b} , \ \sqrt{\left( a + b \right)} , \ \left| a + b \right| , \ \left| \left( a + b \right) \right|
\]

\noindent \def\abs{\hbox{abs}}
The absolute value construction is even more interesting, and
there is a potential for ambiguity: does $|a|b|c|$ represent
$\abs(a\,\abs(b)\,c)$, or $\abs(a)\,b\,\abs(c)$?
Natural Math will use the second interpretation, but this can be
changed using brackets:

\begin{verbatim}
|x over y| 5 |x over y|,
|(x over y| 5 |x over y)|
\end{verbatim}

\[
\left| \frac {x} {y} \right| 5 \left| \frac {x} {y} \right| , \ \left| \frac {x} {y} \left| 5 \right| \frac {x} {y} \right|
\]

\noindent Finally, the trig functions can be raised to powers:

\begin{verbatim}
sin^2 x + cos^2 x = 1,
sin^-1 (sqrt3 over 2) = pi over 3
\end{verbatim}

\[
{\sin} ^ {2} x + {\cos} ^ {2} x = 1 , \ {\sin} ^ { - 1} \left( \frac {\sqrt{3}} {2} \right) = \frac {\pi} {3}
\]

\noindent \subsection{Integrals, Sums and Limits}

\noindent \indent
Here is the integral symbol

\begin{verbatim}
integral
\end{verbatim}

\[
\int
\]

\noindent Here is the integral sign used: note the optional use of `of'.  Also
`dx' is a single symbol.  It can be expressed as two separate symbols,
but the use of the single symbol slightly improves the typesetting.

\begin{verbatim}
integral x^3 dx,
integral of x^3 dx,
integral of x^3 d x
\end{verbatim}

\[
\int {x} ^ {3} \,d x , \ \int {x} ^ {3} \,d x , \ \int {x} ^ {3} d x
\]

\noindent Definite integrals are a little trickier, because Natural Math has to
figure out what should be in the limits.  It will make an intelligent
guess, but
sometimes it needs help.  This can be provided, either with brackets,
or with `of'.

\begin{verbatim}
integral from 1 to a + b x^3 dx,
integral from 1 to a b x^3 dx,
integral from 1 to (a b) x^3 dx,
integral from 1 to a b of x^3 dx
\end{verbatim}

\[
\int _ {1} ^ {a + b} {x} ^ {3} \,d x , \ \int _ {1} ^ {a} b {x} ^ {3} \,d x , \ \int _ {1} ^ {a b} {x} ^ {3} \,d x , \ \int _ {1} ^ {a b} {x} ^ {3} \,d x
\]

\noindent Sums are exactly the same.  The following example shows that you
don't need to use both of the `from' and `to' quantifiers.

\begin{verbatim}
sum from n <= 20 of a_n
\end{verbatim}

\[
\sum _ {n \le 20} {a} _ {n}
\]

\noindent Limits are similar: we have the `as' quanitifier:

\begin{verbatim}
rho = lim as n to infinity | a_(n+1) over a_n | ,
rho = lim as n to infinity of | a_(n+1) over a_n | ,
\end{verbatim}

\[
\rho = \lim _ {n \to \infty} \left| \frac {{a} _ {n + 1}} {{a} _ {n}} \right| , \ \rho = \lim _ {n \to \infty} \left| \frac {{a} _ {n + 1}} {{a} _ {n}} \right| , \ 
\]

\noindent `From' and `to' may also be used with brackets 
(both round and square), although
instead of `of' you can use `end'.

\begin{verbatim}
[ x^3 ] from 0 to 6 / a end ,
[ x^3 ] from 0 to 6 / a ,
[ x^3 ] from 0 to (6 / a)
\end{verbatim}

\[
\left[ {x} ^ {3} \right] _ {0} ^ {6 / a} , \ \left[ {x} ^ {3} \right] _ {0} ^ {6} / a , \ \left[ {x} ^ {3} \right] _ {0} ^ {6 / a}
\]

\noindent More examples:

\begin{verbatim}
integral u dv over dx dx = u v - integral v du over dx dx
\end{verbatim}

\[
\int u \frac {\,d v} {\,d x} \,d x = u v - \int v \frac {\,d u} {\,d x} \,d x
\]

\begin{verbatim}
integral from 0 to 10 of theta^3 dtheta
=
[x^4 over 4] from 0 to 10
=
10^4 over 4 - 0^4 over 4
=
1000 over 4
\end{verbatim}

\[
\int _ {0} ^ {10} {\theta} ^ {3} \,d \theta = \left[ \frac {{x} ^ {4}} {4} \right] _ {0} ^ {10} = \frac {{10} ^ {4}} {4} - \frac {{0} ^ {4}} {4} = \frac {1000} {4}
\]

\begin{verbatim}
integral from -1 to 1 1 over x^(2/3) dx
=
limit as a to 0^- of
integral from -1 to a 1 over x^(2/3) dx
+ 
limit as b to 0^+ of
integral from b to 1 1 over x^(2/3) dx
\end{verbatim}

\[
\int _ { - 1} ^ {1} \frac {1} {{x} ^ {2 / 3}} \,d x = \lim _ {a \to {0} ^ {-}} \int _ { - 1} ^ {a} \frac {1} {{x} ^ {2 / 3}} \,d x + \lim _ {b \to {0} ^ {+}} \int _ {b} ^ {1} \frac {1} {{x} ^ {2 / 3}} \,d x
\]

\begin{verbatim}
= 
limit as a to 0^- of
[3 x^(1/3) ] from -1 to a
+ 
limit as b to 0^+ of
[3 x^(1/3) ] from b to 1
\end{verbatim}

\[
 = \lim _ {a \to {0} ^ {-}} \left[ 3 {x} ^ {1 / 3} \right] _ { - 1} ^ {a} + \lim _ {b \to {0} ^ {+}} \left[ 3 {x} ^ {1 / 3} \right] _ {b} ^ {1}
\]

\noindent Note that in the last example, the use of the `of's is rather crucial.  See
what happens if we don't use them:

\begin{verbatim}
= 
limit as a to 0^-
[3 x^(1/3) ] from -1 to a
+ 
limit as b to 0^+
[3 x^(1/3) ] from b to 1
\end{verbatim}

\[
 = \lim _ {a \to {0} ^ { - \left[ 3 {x} ^ {1 / 3} \right] _ { - 1} ^ {a}} + \lim _ {b \to {0} ^ { + \left[ 3 {x} ^ {1 / 3} \right] _ {b} ^ {1}}}}
\]

\noindent \subsection{Inserting Text}

\noindent To put a paragraph of text in your output, start the line with the
word {\tt text}.  The text following, and the following non-blank
lines will be inserted directly into the La\TeX\ file.  Indeed, if you
know La\TeX, you can even use La\TeX\ commands.  (We won't illustrate
that here, but that is how this document was created.)

\noindent \begin{verbatim}
text Here are some lines
of text.  How do they look?
\end{verbatim}

\noindent Here are some lines
of text.  How do they look?

\noindent \

\noindent You can also insert single words into formulae as follows:

\begin{verbatim}
1 over x to 0 text as x to infinity
\end{verbatim}

\[
\frac {1} {x} \to 0 \hbox{ as } x \to \infty
\]

\noindent or several words

\begin{verbatim}
x over (x^2 + 1) text (grows at the same rate as) 1 over x
\end{verbatim}

\[
\frac {x} {{x} ^ {2} + 1} \hbox{   grows at the same rate as  } \frac {1} {x}
\]

\noindent You can also do this using quotes (note the spaces between the quotes and
the words --- it will look different if they are not there):

\begin{verbatim}
x over (x^2 + 1) " grows at the same rate as " 1 over x
\end{verbatim}

\[
\frac {x} {{x} ^ {2} + 1} \hbox{ grows at the same rate as } \frac {1} {x}
\]

\noindent These commands are very fussy - they must have {\em only\/} letter based
text in them.  Otherwise you get an error message, which brings us to the
next topic.

\noindent \subsection{Error Messages}

\noindent \indent
If Natural Math finds an error, it will spit out the part of the lines
it was able to process, and then follow it with a kind of descriptive
error message, including a rough idea of which line number it was
in the {\tt .nat} file where the error happened.  This same
error will also be written on the command line at which you ran
naturalmath.

\noindent \indent
Here are examples:

\begin{verbatim}
These errors were inserted deliberately.
\end{verbatim}

\begin{verbatim}
These errors were inserted deliberately.
^
Error: what's this: `These' just before line 614
\end{verbatim}

\begin{verbatim}
16 * x -
1 + 2 over (x + yy) - 11.2235 +
24 / 13
\end{verbatim}

\begin{verbatim}
1 + 2 over (x + yy) - 11.2235 +
                ^
Error: what's this: `yy' just before line 619
\end{verbatim}

\begin{verbatim}
(1+2
\end{verbatim}

\begin{verbatim}
(1+2
    ^
Error: right bracket missing just before line 622
\end{verbatim}

\noindent \subsection{Debug and Newpage and Comments}

\noindent \indent
Finally, if you want to see how you wrote the command along with
the typeset version, put the word {\tt debug} at the beginning of
your math.  That is how this tutorial was created.

\noindent \begin{verbatim}
debug x + y
\end{verbatim}

\noindent will produce

\begin{verbatim}
x + y
\end{verbatim}

\[
x + y
\]

\noindent To put comments in the {\tt .nat} file, note that any line beginning with
{\tt \#} will not be processed by Natural Math.

\noindent \indent
To start a new page, issue the one line command (followed by a blank line)

\noindent \begin{verbatim}
newpage
\end{verbatim}

\newpage

\noindent \section{Want More Features?  Bugs to report?}

\noindent \indent
Obviously, if you want to make really complicated math formulae, or have more
control over how it looks, you should learn the \TeX, AMS\TeX\ or
La\TeX\ programs.  

\noindent \indent
Otherwise
email Stephen Montgomery-Smith at
\par
\begin{center}{\tt stephen@math.missouri.edu}.\end{center}
Same if you have bug reports.  Also, if you solve bugs,
or made improvements, please, please
tell me about it.

\end{document}
