%This is the first draft
%% Typed in AmSLaTeX
%% Carmen 1/13/95
%% Yuri 1/13/95
%% Carmen 1/25/95
%% Yuri 
%% Carmen 1/31/95
\documentstyle[12pt,righttag]{amsart}
%\setlength{\oddsidemargin}{0in}
%\setlength{\evensidemargin}{0in}
%\setlength{\textheight}{9in}
%\setlength{\topmargin}{-.37in}
%\setlength{\textwidth}{6.5in}
%\renewcommand{\baselinestretch}{1.5}
%\renewcommand{\theequation}{\arabic{section}.\arabic{equation}}
\newtheorem{thm}{Theorem}
\newtheorem{prop}{Proposition}[section]
\newtheorem{conj}{Conjecture} \renewcommand{\theconj}{\Alph{conj}}
\newtheorem{hyp}[thm]{Hypothesis}
\newtheorem{cor}[thm]{Corollary}
\newtheorem{ques}[thm]{Question}
\newtheorem{lem}{Lemma}
\newtheorem{res}[thm]{Result}
\newtheorem{prob}[thm]{Problem}
\newtheorem{defn}[thm]{Definition}
\newtheorem{step}{Step} \renewcommand{\thestep}{\Roman{step}}

\theoremstyle{definition}
%\newtheorem{conj}{Conjecture}  \renewcommand{\theconj}{}
\newtheorem{exmp}[prop]{Example}  %\renewcommand{\theexmp}{}

\theoremstyle{remark}
\newtheorem{rem}[prop]{Remark} % \renewcommand{\therems}{}
\newtheorem{rems}[prop]{Remarks}  \renewcommand{\therems}{}
\newtheorem{ack}{Acknowledgments} \renewcommand{\theack}{}

%%%%%%%%%%%%%%FONTS%%%%%%%%%%%%%%%%%%%%%%%%%%%%%%%%
\newcommand{\bbD}{{\Bbb{D}}}
\newcommand{\bbN}{{\Bbb{N}}}
\newcommand{\bbR}{{\Bbb{R}}}
\newcommand{\bbP}{{\Bbb{P}}}
\newcommand{\bbZ}{{\Bbb{Z}}}
\newcommand{\bbC}{{\Bbb{C}}}
\newcommand{\bbT}{{\Bbb{T}}}
\newcommand{\calV}{{\cal{V}}}
\newcommand{\calE}{{\cal{E}}}
\newcommand{\calM}{{\cal{M}}}
\newcommand{\calL}{{\cal{L}}}
\newcommand{\calD}{{\cal{D}}}
\newcommand{\calC}{{\cal{C}}}
\newcommand{\calS}{{\cal{S}}}
\newcommand{\calP}{{\cal{P}}}
\newcommand{\calF}{{\cal{F}}}
\newcommand{\calT}{{\cal{T}}}
\newcommand{\calH}{{\cal{H}}}
\newcommand{\bH}{{\bold H}}
\newcommand{\bB}{{\bold B}}
\newcommand{\bC}{{\bold C}}
\newcommand{\cnd}{{{\bold C}_0}}
\newcommand{\bx}{{\bar x}}
\newcommand{\bv}{{\bar v}}

%%%%%%%%%%%%%%%GREEK%%%%%%%%%%%%%%%%%%%%%%%%%%%%%%%%
\newcommand{\lam}{\lambda}
\newcommand{\sig}{\sigma}
\newcommand{\eps}{\epsilon}
\newcommand{\gam}{\gamma}
\newcommand{\ome}{\omega}
\newcommand{\Ome}{\Omega}
\newcommand{\al}{\alpha}
\newcommand{\del}{\delta}
\newcommand{\kap}{\kappa}
\newcommand{\pa}{\partial}
\newcommand{\Lam}{\Lambda}
\newcommand{\Del}{\Delta}
\newcommand{\Gam}{\Gamma}
\newcommand{\Sig}{\Sigma}
\newcommand{\Tht}{\Theta}
\newcommand{\tht}{\theta}
\newcommand{\vph}{\varphi}

%%%%%%%%%%%OPERATORNAME%%%%%%%%%%%%%
\renewcommand{\div}{\operatorname{div}}
\newcommand{\loc}{\operatorname{loc}}
\newcommand{\grad}{\operatorname{grad}}
\newcommand{\ess}{\operatorname{ess}}
\newcommand{\sgn}{\operatorname{sgn}}
\renewcommand{\ker}{\operatorname{Ker}}
\newcommand{\supp}{\operatorname{supp}}
\newcommand{\diam}{{\operatorname{diam}}}
\newcommand{\tr}{\operatorname{Tr}}
\newcommand{\mes}{\operatorname{mes}}
\newcommand{\tl}{\operatorname{TL}}
\newcommand{\km}{\operatorname{KM}}
\newcommand{\Supp}{\operatorname{Supp}}
\newcommand{\rs}{\operatorname{rs}}
\newcommand{\card}{\operatorname{card}}

%%%%%%%%%%%%%%%%%%ABBRS%%%%%%%%%%%%%%%%%%%%%%%%%%%%%
\newcommand{\lb}{\label}
\newcommand{\bigp}{\bigoplus}
\newcommand{\bigt}{\bigotimes}
\newcommand{\ti}{\tilde}
\newcommand{\bs}{\backslash}
\newcommand{\bi}{\bibitem}
\newcommand{\emt}{\emptyset}
\newcommand{\no}{\nonumber}
\newcommand{\lng}{\langle}
\newcommand{\rng}{\rangle}
\newcommand{\hmu}{{\hat{\mu}}}
\newcommand{\bone}{{(1)}}
\newcommand{\btwo}{{(2)}}
\newcommand{\btd}{\bigtriangledown}
\newcommand{\bia}{\bigcap}
\newcommand{\biu}{\bigcup}
\newcommand{\sube}{\subseteq}
\newcommand{\supe}{\supseteq}
%\documentstyle[12pt]{amsart}

%\setlength{\oddsidemargin}{0in}
%\setlength{\textheight}{9in}
%\setlength{\textwidth}{6.5in}
%\newcommand{\fmm}{\hspace*{5mm}}


%%%%%%%%%%%%%%%%%%%%%%%%NUMBERING%%%%%%%%%%%%%%%%%%%%%%%%
\newcommand{\bsection}[1]{\section{\normalsize\bf#1}}
\begin{document}
\title[Annular Hull Theorems]{The Annular Hull Theorems for the 
Kinematic Dynamo Operator\\ for an Ideally Conducting Fluid}
\author{C.~Chicone}
\thanks{carmen@@chicone.cs.missouri.edu  supported in part by
the NSF grant DMS-9303767.}
\author{Y.~Latushkin}
\thanks{mathyl@@mizzou1.missouri.edu supported  in part by the
NSF grant DMS-9400518.}
\author{S.~Montgomery-Smith}
\thanks{stephen@@mont.cs.missouri.edu supported in part by
the NSF grant  DMS-9201357.}
\address{Department of Mathematics, University of Missouri,
Columbia MO 65211}
\date{\today}
\keywords{magnetohydrodynamics,
ideally conducting fluid, operator semigroups,
spectral mapping theorem,
weighted composition operators}
\subjclass{ 76W05, 58F99, 58G25}
\maketitle
\begin{abstract}The group generated by the kinematic dynamo operator 
in the space of continuous divergence-free 
sections of the tangent bundle of a smooth manifold is studied.
As shown in previous work, 
if the underlying Eulerian flow is aperiodic, 
then the spectrum of this group is
obtained from the spectrum of its generator by exponentiation, but
this result does not hold for flows with an open set of  
periodic trajectories. In the present paper, we consider
Eulerian vector fields with periodic trajectories and prove
the following  annular hull theorems:  
The spectrum of the group
belongs to the annular hull of
the exponent of the spectrum of the kinematic dynamo operator, 
that is to the union of 
all circles centered at the origin and intersecting this set. 
Also, the annular hull of the spectrum of the group
on the space of divergence free vector fields
coincides with the smallest annulus, containing the spectrum
of the group on the space of all continuous vector fields.
As a corollary,  the spectral abscissa of the generator 
coincides with the growth bound for the group.


\end{abstract}
\section{Introduction}
In the present paper we continue the study, begun in \cite{clms}, of
the spectral properties of the group generated by the 
kinematic dynamo operator for an ideally conducting fluid.

Throughout the paper 
$X$ denotes a smooth compact $n$-dimensional, $n\geq 3$, 
Riemannian manifold  without boundary, $\calT X$ denotes the
tangent bundle of $X$, and $\calT_xX$ denote the tangent space at $x\in X$.
Also, $\bC=C(X,\calT X)$ denotes the Banach space of continuous
sections of $\calT X$ with $\sup$-norm and $\cnd=C_{ND}(X,\calT X)$
denotes the space of divergence free
vector fields on $X$ taken either as the closure in $\bC$
of the set of $C^\infty$-smooth vector fields $\bB$ on $X$ with
$\div \bB=0$ or as the set of $\bB\in\bC$ orthogonal to all 
$\grad g$ for $g$ a $C^\infty$-smooth real functions on $X$. 
The orthogonality here
is understood with respect to a Riemannian metric and volume on $X$.

For an operator $A$ on a Banach space $E$, we use
the notation $\sig(A)=\sig(A,E)$ for the spectrum
of $A$ on $E$. Also, for a set $S\subset \bbC$ 
we define $\calH(S)$,  the annular hull
of $S$, to be the union of all circles centered at the origin of the
complex plane that intersect $S$.


Let $u$ denote a smooth  divergence-free vector field on $X$ and
$\phi^t$ the flow generated by $u$.
If for each $x\in X$ and $t\in \bbR$ there is an operator 
$\Phi(x,t):\calT_xX\to\calT_{\phi^tx}X$ such that
$\Phi$ is smoothly dependent on both its arguments and if,
for all $x\in X$ and
$t,\tau\in \bbR$, 
\[
\Phi(x,t+\tau)=\Phi(\phi^tx,\tau)\Phi(x,t),\quad
\Phi(x,0)=I, 
\]
then $\Phi$ is called a smooth cocycle over $\phi^t$.
A cocycle $\Phi$ induces on $\bC$ a strongly continuous group 
$\{T^t\}$ defined by
\begin{equation}\lb{defTt}
T^t\bB(x):=\Phi(\phi^{-t}x,t)\bB(\phi^{-t}x), 
\quad t\in\bbR, \quad x\in X, \quad \bB\in\bC.
\end{equation}
We let $L$ denote the infinitesimal generator of this group,
$\left.L\bB=\frac{d}{dt}T^t\bB\right|_{t=0}$, so that
$T^t=e^{tL}$. Also, for the remainder of the paper we
will assume that the group $T^t$
preserves $\cnd$. 

Our objective is to relate the spectra of 
$T^t$ and $L$ on $\cnd$.
However, before stating the precise results we will prove,
we pause to explain the relationship of $L$ to the 
kinematic dynamo operator for an ideally conducting fluid.
For this, consider a steady incompressible conducting
Eulerian fluid with velocity $u=u(x)$,
for $x\in \bbR^3$, and let $\phi^t$ denote the corresponding flow.
The induction equation for 
a magnetic field ${\bH}=\bH(x,t)$ is given by
\begin{equation}\lb{kdeq}
 \dot{\bH} = \nabla\times (u\times {\bH}) + 
                   \varepsilon \Delta {\bH},\quad
 \div{\bH}  = 0,
\end{equation}
where $\varepsilon= {{\cal R}e_m}^{-1}$, and ${\cal R}e_m$ is the
magnetic Reynolds number (see, e.g., \cite[Ch.~6]{Moffatt}).
Recently, the spectral properties of the {\em kinematic dynamo operator}
$\calL_\varepsilon$, defined by the right hand side of \eqref{kdeq}, 
have been the subject of
intensive study \cite{Rafael,KY,Nunez,Vishik},  
especially in connection with
the famous ``fast'' dynamo problem, see
\cite{Arnold,AZRS,BC,FV,Vishik}
and the references therein.
 
For the ideally conducting fluid, $\varepsilon =0$,
the system \eqref{kdeq} given by
\begin{equation}\lb{kde}
\dot{\bH}= - (u,\nabla) {\bH} +
( {\bH}, \nabla ) u,
\quad \div{\bH}  = 0,
\end{equation}
has \cite{Moffatt} so-called Alfven solutions
\begin{equation}\lb{defpf}
{\bH}(x,t)=D\phi^t(\phi^{-t}x){\bB} (\phi^{-t}x),
\quad {\bH}(x,0)={\bB}(x),
\end{equation}
where $D\phi^t(x)$  denotes the differential of $\phi^t$.
These solutions are given by the group of ``push-forward
operators'' induced, as in \eqref{defTt},
by the cocycle $\Phi(x,t)=D\phi^t(x)$.
The infinitesimal generator
$\calL=\calL_0$ of this group is given by Lie differentiation
in the direction of $u$, that is,
\begin{equation}\lb{defL}
\calL: \bB\mapsto  (\bB,\nabla)u - (u,\nabla) \bB,\quad \div\bB=0.
\end{equation}
For $\varepsilon >0$, the operator 
$\calL_\varepsilon=\varepsilon\Delta+\calL$ is strongly elliptic
and generates an analytic semigroup. This implies (see, e.g., \cite{Pazy})
that the spectrum of the group $e^{t\calL_\varepsilon}$ is obtained 
from $\sigma(\calL_\varepsilon)$ by exponentiation, 
in fact,
$\sig(e^{t\calL_\varepsilon}) = \exp t\sig (\calL_\varepsilon)$ for $t\neq 0$.
The operator $\calL=\calL_0$ is not elliptic and, as we will show later,
this spectral mapping
result does not hold for $\calL$ with no additional assumptions on $u$. 


For a general flow on a compact manifold $X$,
we note that the operators in \eqref{defTt} belong to the 
class of so-called weighted
composition operators, see \cite{HPS,LS,Mather} for
connections with  dynamical system theory,
and \cite{LMS2,LMSFun} 
for connections with the theory of operator semigroups on Banach spaces. 
The semigroup \eqref{defTt} in the space of {\em all}
continuous vector fields was
studied in \cite{CS,J} where the spectral 
mapping theorem and the annular hull theorem (see precise statements below)
were proved. However,  the proofs of these results for
\eqref{defTt} on the space of {\em divergence-free} vector-fields
requires a new technique, 
developed in \cite{clms} and in the present paper.
This work was inspired by the
important paper \cite{Rafael} which initiated recent research 
on the spectral properties of the groups
\eqref{defTt} and \eqref{defpf} on the space of 
divergence-free vector-fields.

For notational convenience, for the semigroup $\{T^t\}$ in
\eqref{defTt} we define $T:=T^1$. Also, we let
$\calH_C$ denote the smallest  annulus, centered
at the origin of the complex plane, that contains $\sig(T,\bC)$.

The following {\em Spectral Mapping Theorem}
was proved in \cite{clms} under the additional assumption
that $u$ is a {\em nonvanishing} vector field. 
\begin{thm} \lb{t1}
Suppose   
the aperiodic trajectories of the flow $\phi^t$ are dense in $X$. 
If $L$ is the infinitesimal generator of a group $\{T^t\}$, 
$T^t=e^{tL}$, of weighted composition operators,  then
\[
\sig(e^{tL},\cnd) = \exp t\sig (L,\cnd),\quad t\neq 0.
\]
Moreover, $\sig(T^t,\cnd)$ is invariant with respect to
rotations centered at origin, and $\sig(L, \cnd)$ is
invariant with respect to translations along the imaginary axis.
Also, $\sig(T^t, \cnd)=\calH_C$.
\end{thm}

Below (cf. also \cite{CS}), we will give an example to show
the spectral mapping theorem does not hold in general if there
is an open set of periodic points. Also, the spectra of $T^t$ and $L$, 
generally, do not have the invariance property in this case.
The main results of this paper are the following 
{\em Annular Hull Theorems}. 
\begin{thm} \lb{t2}  
Suppose  $\phi^t$ is the flow of a 
smooth divergence free vector field on $X$. 
If $L$ is the infinitesimal generator of a group $\{T^t\}$ of weighted
composition operators as in \eqref{defTt}, then
\[
e^{t\sig(L,\cnd)}\subset \sig (e^{tL},\cnd) \subset 
 \calH (e^{t\sig (L,\cnd)}).
\]
\end{thm}
\noindent We note that $\calH(\sig(T, \bC))$, generally,
consists of several annuli centered at the origin.
Passing to the subspace $\cnd$ radically changes the picture. In fact,
$\calH(\sig(T, \cnd))$ is exactly one annulus: the gaps
in $\calH(\sig(T, \bC))$, if any, are filled.
\begin{thm}\lb{spectr}
If $\phi^t$ is the flow of a 
smooth divergence free vector field on $X$ and $\{T^t\}$ is
a group fo weighted composition operators as in \eqref{defTt},
then
\begin{equation}\lb{spechul}
\calH(\sig(T, \cnd))=\calH_C.
\end{equation}
\end{thm}
\noindent We remark  that there are nonnegative  numbers $r_{\pm}$ 
such that
$\calH_C=\{z\in\bbC: r_- \leq|z| \leq r_+\}$.
Hence, $\calH(\sig(T,\cnd))$
can be computed via exact Lyapunov-Oseledec
exponents for the cocycle $\Phi$
with respect to all 
$\phi^t$-ergodic measures on $X$ (see \cite{clms,LS} for details).

We stress that for Theorem~\ref{t2}--\ref{spectr} 
the flow of the vector field $u$ is allowed to have rest points
and periodic orbits and no assumption on the density of
aperiodic trajectories is made. In the present paper we also
show that Theorem~\ref{t1}
holds for a flow $\phi^t$ with rest points. 
We mention that, for a cocycle $\Phi$ in an infinite
dimensional Banach space, the annular hull theorem, generally, fails
\cite{LS,Mont}.

Recall the 
spectral abscissa $s$ and growth bound $\omega$ of the group $\{e^{tL}\}$
are defined by
\[
s(L):=\sup\{
\mbox{Re} z: z\in\sigma(L)\} \quad \mbox{ and }\quad
\omega(L):=\lim_{t\to\infty}t^{-1}\ln\|e^{tL}\|.
\] 
See \cite{Nagel} 
for a detailed discussion of these concepts and for
examples of groups with
$s$ {\em strictly} less than $\omega$. 
The Annular Hull Theorems imply the following:
\begin{cor}\lb{c1}
If the hypotheses of Theorem~\ref{t2} hold, then
$s(L)=\omega(L)$. 
\end{cor}

Of course, Theorems~\ref{t1}--\ref{spectr} and Corollary~\ref{c1} 
apply to the group $\{e^{t\calL}\}$. Since
the spectral mapping theorem is
valid for the semigroup generated by the kinematic dynamo operator
$\calL_\varepsilon$ with $\varepsilon>0$, 
one has
$s(\calL_\varepsilon)=\omega(\calL_\varepsilon)$ for $\varepsilon>0$. For
$\calL_0=\calL$,  Corollary~\ref{c1} shows that this equality is also
valid for $\varepsilon=0$.

This observation might be useful in connection with the so-called
``fast'' dynamo problem.
Recall (see, e.g., \cite{Arnold,AZRS,BC}) that the
kinematic dynamo is called
fast provided the limit
$\limsup_{\varepsilon\to 0}\omega(\calL_\varepsilon)$
is positive.
M.~Vishik \cite{Vishik} has shown that
\[\limsup_{\varepsilon\to 0}\omega(\calL_\varepsilon)\leq \omega(\calL_0).\] 
Thus, by the remarks above,
his theorem can be reformulated to state 
\[\limsup_{\varepsilon\to 0}s(\calL_\varepsilon)\leq s(\calL_0).\]
We stress that
the last assertion does not involve the construction of the group
$\{e^{t\calL_\varepsilon}\}$; it is given in the terms of generators only.

M.~Vishik \cite{Vishik} also proved that the dynamo is not fast 
provided the supremum over all Lyapunov {\it numbers}
of the cocycle $D\phi^t$ is nonpositive (in fact, $\omega(\calL_0)$
is equal to this supremum). 
We remark, that this condition for the absence of the fast dynamo
remains valid even if this supremum is taken 
over all {\it exact} Lyapunov-Oseledec
exponents for the cocycle $D\phi^t$
with respect to all 
$\phi^t$-ergodic measures on $X$. This remark holds
with no restrictions on $u$
(see \cite[Remark~3.8]{clms}
for the case when $\phi^t$ has  a dense set of aperiodic trajectories). 
\section{Annular Hull Theorems}
In this section we will prove 
the Annular Hull Theorems on $\cnd$. 
\subsection{Preliminaries}
Let $\sig_{ap}(A)=\sig_{ap}(A,E)$ denote the approximate point spectrum
of the operator $A$ on a Banach space $E$, that is, the set of approximate
eigenvalues $\lambda$ such that for every $\eps>0$ there exists
an approximate eigenvector 
$\bB$ with $\|(A-\lam I)\bB\|\leq\eps\|\bB\|$.

Since $\cnd$ is a $T^t$-invariant subspace of $\bC$, 
one has $\sig_{ap}(T,\cnd)\subset\sig_{ap}(T,\bC)$.
Using standard rescalings \cite[p.~14]{Nagel},
the Spectral Inclusion Theorem 
$\exp t\sig_{ap}(L)\subset\sig_{ap}(e^{tL})$,
and the spectral mapping theorems for point and residual spectra, see
\cite[pp.~84--85]{Nagel}, 
Theorem~\ref{t2} for $T=T^1$ as in \eqref{defTt}
can be derived from the following fact.
\begin{thm} \lb{t3}
If $1\in \sig_{ap} (T,\bC)$, then 
$\sig_{ap} (L,\cnd) \cap i\bbR \neq\emptyset$.
\end{thm}

The proof of  Theorem~\ref{t3}  
is based on several lemmas. 
The first lemma is an adaptation of a result of R.~Man\~e \cite{Mane}.
\begin{lem}\lb{l1}
For $1\in \sig_{ap} (T,\bC)$ it is necessary and sufficient
that there is a point $\bx \in X$ and a
vector $\bv\in \calT_\bx X$, so that $\|\bv\| =1$ and
\begin{equation}
\sup_{t\in\bbR} \| \Phi (\bx, t) \bv\| < \infty.
\lb{1}
\end{equation}
\end{lem}
\begin{pf} The proof of sufficiency is in \cite{Mane}.
We will prove the necessity.

Since $1\in \sig_{ap} (T,\bC)$, there is, for each natural number
$N$ 
(see, e.g., \cite[Lemma 2.6]{clms})
a section $f_N\in \bC$ with $\|f_N\|_\bC=1$ 
so that $\|T^kf_N\|_{\bC}\le 2$ for $|k|\le N$. Choose
$x_N\in X$ so that $\|f(x_N)\|=\|f_N\|_{\bC}=1$, and define
$v_N=f(x_N)$. Then, we have  
\[
\|\Phi(x_N,k)v_N\|\le \|T^kf_N\|_{\bC}\leq 2\mbox{ for  }
|k|\le N.
\]
Since $X$ and the unit sphere in $\bbR^n$ are compact,
we may assume that $(x_N,v_N)\to (\bx,\bv)$ for some $\bx\in X$, 
$\|\bv\|=1$. We claim that \eqref{1} holds for $(\bx,\bv)$.
Indeed, for 
fixed $k\in\bbZ$, one has 
$\|\Phi(\bx,k)\bv\|=\lim_{N\to\infty}\|\Phi(x_N,k)v_N\|\le 2$
and \eqref{1} follows from the estimate
\[\sup_{t\in\bbR}
\|\Phi(\bx,t)\bv\|
\le \sup_{k\in\bbZ}\sup_{\tau\in[0,1]}
\|\Phi(\phi^k\bx,\tau)\Phi(\bx,k)\bv\|.\]
\end{pf}
\begin{defn} A point $\bx$, respectively a  vector $\bv$, 
that satisfies \eqref{1} is called a {\em Man\~e point}, respectively
a {\em Man\~e vector}.
\end{defn}

Define the prime period of a point $x\in X$ by
$P(x) = \inf \{ t> 0: \phi^t x =x\}$ and note that $P$ is a lower
semi-continuous function. Naturally, we assign $P(x)=\infty$ for
aperiodic points $x\in X$.
Suppose $\bx$ is a Man\~e point.
There are two cases to consider:
\begin{description}
\item[\bf Case 1.] {\it Uniformly bounded periods.}
There is a constant $\bar{P}\in\bbR_+$ and a
neighborhood $U\ni \bx$ such that $P(y)\leq \bar{P}$ for all $y\in U$.
\item[\bf Case 2.] {\it Unbounded periods.}
 For every $\bar{P}\in\bbR_+$ and every neighborhood $U\ni \bx$
there is an $x\in U$ such that $P(x)\geq \bar{P}$.
\end{description}
These cases will be considered in subsections~2.2 and 2.3, respectively.
Our strategy in these subsections is as follows. 
Using the Man\~e point $\bx$ and the Man\~e vector $\bv$ as in 
\eqref{1}, we  will choose an appropriate
point $x_0\in X$ near $\bx$, and a vector $v\in\calT_{x_0}X$.
Then, we will define a divergence free vector field $w$, supported in a
sufficiently small neighborhood $D$ of $x_0$, with values $v$
in a smaller neighborhood $B\subset D$ of $x_0$. 
We will construct  a divergence-free
approximate eigenvector $\bB$ for $L$ that  corresponds to a pure
imaginary approximate eigenvalue $\lambda$ by integrating
$T^tw(x)$ along sufficiently 
long  segments of the trajectories of $\phi^t$
through the points $x\in D$. Finally, we will estimate 
$\|L\bB-\lambda\bB\|$
from above and $\|\bB\|$ from below to show that $\bB$ is indeed
an approximate eigenvector for $L$.

To construct $w$ as just described,  we will use
the following lemma from \cite{clms}:
\begin{lem}\lb{l3}
Suppose 
$x_0\in X$ is a point with $P(x_0)>0$,
and $v\in\calT_{x_0}X$. If 
numbers  $\eps > 0$ and $\delta>0$ are given, then
there exist neighborhoods $B\subset D$ of $x_0$ with  
$\diam {D} < \eps$  such that the following holds. 
There exists a smooth ``bump''-function
$\al:X\to [0,1]$ with $\al(x) = 1$ for $x\in B$ and 
$\al (x) =0$ for $x\not\in D$
and a continuously differentiable vector-field $w_0$ 
supported in $D\setminus B$, 
such that
\begin{enumerate}
\item
The vector-field $w(x) = \al(x) v+w_0 (x)$ is divergence-free and has
value $v$ in $B$;
\item
$\| w_0 \|_{\bC} \leq \delta$.
\end{enumerate}
\end{lem}
\begin{pf} See \cite[Lemma~2.3]{clms}.
The geometrical idea of the proof is as follows.
Take positive numbers $a<b<\eps$.
There is a thin and
long ellipsoid $B$ centered at $x_0$ 
whose longest axis  directed along $v$ has length $b/2$  while
all its other  axes have length $a/2$. The ellipsoid
$B$ is contained in a longer ellipsoid $D$ whose 
two longest axes, directed, respectively, along $v$
and along some perpendicular to $v$, both have length $b$,
while all its other axes have length $a$.
  
For a function $\al$ as in the statement of the lemma
one can compensate for the nonzero
divergence of $\al(\cdot)v$ by taking an appropriate
vector field
$w_0$ whose norm satisfies the estimate
$\|w_0\|\leq Ca/b$ for some constant  $C>0$  that does not
depend on $a$ or $b$.
To see this, imagine a flow of fluid, 
leaking out from the top of a thin vertical pipe $B$, 
then slowly recirculating to the bottom of $B$ 
through the much wider pipe $D$.
At any rate, the lemma is proved by taking  $a/b$ sufficiently small. 
\end{pf}

To carry out the estimates for $\bB$, mentioned above, we will
need to control
the sojourn time of a trajectory segment that passes through $D$ and $B$.
To do this, we will use the following definitions. 
For an open set $U$ containing $x\in X$ and $s> 0$, define
\begin{equation}
\Theta_{s,U} (x)  = \{ t\in \bbR: |t| \leq s,\,\phi^t x \in U
\},\quad
M_{s,U}(x)  = \mes \Theta_{s,U}(x).
\lb{thetam}
\end{equation}
\begin{lem} \lb{l5}
Suppose $n=\dim X\geq 3$.
There is a constant $K>0$ such that 
for every $x_0\in X$  with $P(x_0)>0$,
every $s>0$ with $s<P(x_0)/8$ and every
neighborhood $V\ni x_0$
there exist open sets $B\subset D$ as in Lemma~\ref{l3},
$B\subset D\subset V$, such that for every $x\in X$
the following inequality holds: 
\begin{equation}
M_{s,D}(x) \leq KM_{s,B} (x_0).
\lb{uniform}
\end{equation}
\end{lem}
\begin{pf} Use the semicontinuity of $P$ to choose 
 a neighborhood $U\subset V$ of $x_0$
so small, that $P(y)\geq P(x_0)/2$ for all $y\in U$. We will look
for $B\subset D\subset U$. Since $s\leq P(y)/4$, for any $y\in U$
one has $\phi^ty\not\in U$ provided $s<|t|\leq 2s$ (see 
\cite[Lemma~2.4]{clms} for the aperiodic and Lemma~\ref{l4} below for
the periodic $x_0$). Hence, $\Theta_{s,D}(y)=\Theta_{2s,D}(y)$ for each
$y\in D$. As in \cite[Lemma~2.4]{clms} this proves that
\eqref{uniform} holds for all $x\in X$, provided it holds
for $x\in D$.
The rest of the rigorous proof  is exactly as in
\cite[Lemma~2.5]{clms} provided  $s$ replaces $N$ in that lemma.
The geometrical idea is as follows. For $x\in D$, denote by
$l_{s,D}(x)$ the length of the segment of the trajectory
$\{\phi^tx:|t|\leq s\}\cap D$, by $t'_D$ the time 
between the moment when the segment of the trajectory through $x$
first enters $D$ and the moment when it {\it last} exits $D$, and by
$t''_B$ the time 
between the moment when the segment of the trajectory through $x_0$
first enters $B$ and the moment when it {\it first} exits $B$. Clearly,
$M_{s,D}(x)\leq t'_D$ and $M_{s,B}(x_0)\geq t''_B$.
 By the Mean Value Theorem
$l_{s,D}(x)=k_1t'_D$ and $l_{s,B}(x_0)=k_2t''_B$,
where $k_1\approx k_2\approx \|u(x_0)\|$ for small $U$. 
Since $B$ and $D$ are ellipsoids, 
described in the proof of Lemma~\ref{l3} above, one can prove that
$l_{s,D}(x)/l_{s,B}(x_0)$ is bounded.
\end{pf}

\subsection{Case 1. Uniformly bounded periods.}
Consider the Man\~e point $\bx$ as in \eqref{1} and  let the
open set $U\ni \bx$ be as in Case~1.  In particular, 
$\bx$ is periodic. 
Using the lower semicontinuity of $P$ we can and will choose $U$
sufficiently small so that one of the following 
alternatives holds:
\begin{description}
\item[\bf Subcase 1.1.]$P(x)=0$ for all $x\in U$;
\item[\bf Subcase 1.2.]$P(x)>0$ for all $x\in U$;
\item[\bf Subcase 1.3.]$P(\bx)=0$ but $P(x_n)>0$
for a sequence $x_n\to \bx$, $x_n\in U$.
\end{description}

For Subcase 1.1, every $x\in U$ is a rest
point: $\phi^tx=x$. Also, since $\Phi$ is a
cocycle, $\Phi(x,t)$ is a group for every $x\in U$. Let $A(x)$
denote the matrix that generates this group: $\Phi(x,t)=e^{tA(x)}$,
$t\in\bbR$, $x\in U$ and note that the matrix-valued function $A$ is
continuous on $U$. Also, for $w\in\cnd$ with $\supp w\subset U$,
we have $Lw(x)=A(x)w(x)$. Since, by \eqref{1},
$\sigma(e^{tA(\bx)})\cap\bbT\neq\emptyset$,
the spectral mapping theorem
for matrices implies there is $\xi\in\bbR$ so that
$i\xi\in\sigma(A(\bx))$. Define $x_0=\bx$, choose $v$ with 
$\|v\|=1$ so that $(A(x_0)-i\xi)v=0$, 
and fix $N\in\bbN$. Also, choose $\eps>0$ and
$\delta>0$ so small that for every neighborhood $V\ni x_0$
with $\diam {V}<\eps$
the following holds:
\[\sup\{\|(A(x)-i\xi)v\|: x\in V\} < 1/2N,\quad V\subset U,\quad
\delta\sup\{\|A(x)-i\xi\|: x\in U\}<1/2N.\]
Apply Lemma~\ref{l3} for these $\eps$ and $\del$ 
to find $B\subset D\subset U$.
Take $w=\al(\cdot)v+w_0$ as indicated in Lemma~\ref{l3}. Recall, 
that $\al(x)\leq 1$, $x\in X$.
Then, using the estimate (2) of the lemma, we find that:
\[\|(L-i\xi)w\|_{\cnd}=\sup_{x\in D}\|(A(x)-i\xi)w(x)\|\leq 1/N.\]
As a result, $i\xi\in\sigma_{{ap}}(L)$, and Theorem~\ref{t3} is
proved.

For Subcase 1.2, the neighborhood $U$ does not contain
rest points. By the semicontinuity of $P$
we can and will assume that there is a number $p_0>0$ such that
$P(x)\geq p_0$ for all $x\in U$. 
Moreover, we note that
$\sig (\Phi (\bx, P(\bx))) \cap \bbT \neq \emptyset$. 
Indeed, if $\phi^{P(\bx)}\bx=\bx$, then 
$\Phi(\bx,kP(\bx))=[\Phi(\bx,P(\bx))]^k$ for $k\in\bbZ$.
Since, by \eqref{1},
the sequence $\{\|[\Phi(\bx,P(\bx))]^k\bv\|\}$ is bounded, the matrix
$\Phi(\bx,P(\bx))$ can not be hyperbolic. For
a notational convenience, denote $x_0=\bx$.
Thus, there is some
$v\in \calT_{x_0}X$ and $\xi \in\bbR$, so that
\begin{equation}
\Phi (x_0, P(x_0)) v = e^{i\xi} v,\quad \|v\| =1.
\lb{2}
\end{equation}
We will use the choice of $x_0$ and $v$ as in \eqref{2}
in what follows.

For Subcase 1.3, the Man\~e point $\bx$ is a rest point 
and each $x\in U$ is a periodic point. 
We claim that $\Phi(x_n,P(x_n))$ can not be hyperbolic.
To see this, we assume that $x_n$ {\it is} hyperbolic.
By the Stable Manifold Theorem (see, e.g., \cite{HPS}),
the periodic orbit through
$x_n\in U$ has a stable or an unstable manifold, 
contrary to the fact that
$U$ consists entirely of periodic orbits.
For a nonhyperbolic $x_n$ we denote $x_0=x_n$, and
select $\xi\in\bbR$ and $v$ as in \eqref{2}.

Starting from the assumption that $1\in\sig_{{ap}}(T,\bC)$,
we have found in each subcase, 1.2 and 1.3,
a periodic point $x_0$ with period $P(x_0)>0$, 
a neighborhood $U$ of $x_0$ consisting entirely of
periodic points whose periods are
uniformly bounded and separated from zero, as well as
a number $\xi\in\bbR$ and a vector
$v$ as in \eqref{2}. We will prove
that $\lambda:=i\xi/P(x_0)$ is an approximate eigenvalue for $L$.

Define $p=P(x_0)$ and let $\mu_1 > 0$ be a constant  such that
\begin{equation}
P(y) \leq \mu_1 p,\quad y\in U.
\lb{per}
\end{equation}
Fix a natural number $N> \mu_1 p +1$.  
In what follows we use the letter $\mu$ (resp., $\nu$) 
with subscripts to denote
``big'' (resp., ``small'') constants that do not depend on $N$.  
Select (small) constants $\nu_i<1$, $i=0,\ldots, 5$. The
required values of these constants
will be determined later.  Define
$$
\mu_2 = 2\sup \{ \| \Phi (x_0, t) \| : |t| \leq \mu_1 p\},\quad R=N +
\mu_1 p.
$$
Also, fix $s>0$ so that $s < \min(p/8,1/4)$ and
\begin{equation}
\max_{|t| \leq s} \left| e^{-\frac{i\xi t}p} -1\right| \leq \nu_4,
\lb{S1}
\end{equation}
\begin{equation}
\sup_{|t| \leq s} \| \Phi (\phi^{-t} x_0, t) v-v\| \cdot \sup_{|n| \leq
3N/p} \| [ \Phi (x_0, p)]^n \| \leq \nu_2.
\lb{D3}
\end{equation}
\begin{lem}\lb{l4}
There exists $\eps > 0$ such that for every open set 
$D\ni x_0$ with $\diam {D} \leq \eps$ and 
every $y\in D$ the following implication holds:\newline
\centerline{If $\phi^t y \in D$ and $|t| \leq R$, then 
$t\in \bigcup_{n\in\bbZ} (-s + np, s + np)$.}
\end{lem}
\begin{pf}
Suppose the lemma is false. Choose $D_k\ni x_0$ with
$\diam {D_k}\to 0$ and $y_k\in D_k$ together with $t_k\in\bbR$, $|t_k|\le
R$, so that $\phi^{t_k}y_k\in D_k$, 
but $t_k\not\in\bigcup_{n\in\bbZ} (-s + np,
s + np)$. By compactness, we may assume $t_k\to t^*$, $|t^*|\le R$. By
continuity, $\phi^{t_k}y_k\to \phi^{t^*}x_0$, and, therefore,
$\phi^{t^*}x_0=x_0$. Since $t^*=np+\tau$ 
for some $n\in\bbZ$ and some $\tau$
with $s<\tau<p-s$, one has $\phi^\tau x_0=x_0$. 
Since $p=P(x_0)$ is the prime
period, in contradiction.
\end{pf}

Fix $\eps > 0$ as in Lemma~\ref{l4}. Choose a neighborhood $V\ni x_0$
with $\diam {V}<\eps$ so small that the following inequalities hold:
\begin{align}
\sup \{ \| \Phi (y,t)\| : |t| \leq \mu_1 p,\,y\in V\} \leq&\mu_2,
\lb{D21}\\
\sup_{y\in V} \sup_{|t| \leq 3N} \| \Phi (\phi^{-t} y, t) - \Phi
(\phi^{-t} x_0, t) \| \leq&\nu_1.
\lb{D1}
\end{align}
We note that $\Phi(x,0)=I$, for $x\in X$.
Fix $\delta$ so that
\begin{align}
\delta< & \nu_3,
\lb{D5}\\
\delta \max_{|t| \leq s} \| \Phi (\phi^{-t} x_0, t) \| \cdot
\sup_{|n| \leq 3N/p} \| [ \Phi (x_0, p)]^n \|\leq
&\nu_5.
\lb{D4}
\end{align}







Let $B\subset D$, $D\subset V$ be the open sets from Lemma~\ref{l5}
with the prescribed choice of $x_0$, $s$, $V$, $\eps$ and $\delta$.
We define $m=M_{s,B}(x_0)$. Consider $\al$ and
$w_0$ as indicated in Lemma~\ref{l3}.
Finally, choose a smooth function $\gam : \bbR\to [0,1]$ with $\supp \gam
\subset [-N, N]$ so that $|\gam' (t) |  \leq 2/{N}$ and
\begin{equation}
\gam (t) \geq \dfrac14 \mbox{ for } |t| \leq \dfrac{N}2.
\lb{G3}
\end{equation}

For $w=\al (\cdot) v + w_0$ as in Lemma~\ref{l3}, we define a
vector field $\bB$ as follows:
\begin{equation}\lb{defbB}
\bB(x) = \int_{-\infty}^\infty \gam(t) e^{-\frac{i\xi t}{p}} (T^t w)(x) \,
dt,\quad x\in X.
\end{equation}
Since $w$ is divergence-free, $\bB$ also is divergence-free.
We claim there is  a constant $\mu>0$, independent of $N$, such that
\begin{equation}
\left\| L\bB -  \dfrac{i\xi}{p} \bB \right\|_{\bC} 
\leq \dfrac{\mu}{N} \|\bB\|_{\bC}.
\lb{3}
\end{equation}
The inequality \eqref{3} guarantees $\sig_{ap}(L) \cap i\bbR \neq
\emptyset$, and proves Theorem~\ref{t3}.  We prove \eqref{3} in two
steps.

{\bf Step 1. The Upper Estimate .}
We will show there is a constant $\mu>0$ such that, for $m$ defined
above,
\begin{equation}
\left\|L\bB - \dfrac{i\xi}{p} \bB \right\|_{\bC} 
\leq \mu\cdot m.
\lb{5}
\end{equation}

A direct calculation shows:
\begin{equation}
(L\bB)(x)- \dfrac{i\xi}{p} \bB (x)
			=-\int_{-\infty}^\infty \gam' (t) e^{-\frac{i\xi t}p} 
			(T^t w)(x) \, dt.
\lb{4}
\end{equation}
Recall that $w(x)=\al(x)v+w_0(x)$ and, for $x\in X$, define
\begin{align*}
I(x) & = \int_{-\infty}^\infty |\gam' (t) | \, \|\Phi (\phi^{-t} x,
t)v \| \al (\phi^{-t} x)\, dt,\\
J(x) & = \int_{-\infty}^\infty |\gam' (t) | \, \| \Phi (\phi^{-t} x, t)
w_0 (\phi^{-t} x) \| \, dt.
\end{align*}
To prove \eqref{5} it suffices to show there is a constant $\mu>0$ 
such that, for all $x\in X$, 
\begin{equation}
\| I(x) \| \leq \mu \cdot m,\quad \| J(x) \| \leq \mu\cdot m.
\lb{6}
\end{equation}
Clearly, for 
$x\not\in \bigcup_{y\in D} \bigcup_{0 \leq t \leq P(y)} \phi^t (y)$,
since
$\al (\phi^{-t} x)=0$ and since $w(\phi^{-t}(x))=0$,
we have $I(x) = J(x) =0$. 
Fix $x\in \bigcup_{y\in D}\bigcup_{0 \leq t\leq P(y)} \phi^t (y)$ and
select $y\in D$ so that $x=\phi^\tau y$.  
In accordance with \eqref{per} we will
assume $0\leq \tau \leq \mu_1 p$.

To estimate $\| I (x)\|$ from above, we note first, that
$$
 \| I (x) \| \leq \int_{-\infty}^\infty | \gam' (t+ \tau)| \, \| \Phi
(\phi^{-t} y, t+ \tau) v\| \al (\phi^{-t} y) \, dt.
$$
Since $\tau\le \mu_1p$ and $\supp \gamma'\subset [-N,N]$,
the integration is unchanged if restricted to
$|t|\le R=\mu_1p+N\le 3N$. Also, $y\in D\subset V$.
We use \eqref{D1} and \eqref{D21} to obtain the following estimate:
\begin{align}
\| I(x) \| & \leq \int_{-\infty}^\infty |\gam' (t+ \tau) | \, \| \Phi
(y,\tau)\| 
\left(\|\Phi (\phi^{-t} x_0, t) v \|\right.\\
&+\left.\| \Phi (\phi^{-t} y, t) v -\Phi (\phi^{-t} x_0, t) v \|\right)
\lb{eestI}
\al (\phi^{-t} y) \, dt  
 \leq \mu_2 \left( I_1 (y) +\nu_1 I_2 (y)\right),\nonumber
\end{align}
where 
\begin{align*}
I_1 (y) & = \int_{-\infty}^\infty | \gam' (t+ \tau) | \, \, \| \Phi
(\phi^{-t} x_0, t) v \| \al (\phi^{-t} y) \, dt,\\
I_2 (y) & = \int_{-\infty}^\infty | \gam' (t+ \tau)| \al (\phi^{-t} y)
\, dt.
\end{align*}

We use Lemma~\ref{l4} to estimate $I_1(y)$ and $I_2(y)$.  Since $y\in D$
and $\supp \al \subset D$, the integration in $I_1(y)$ and $I_2(y)$ is
unchanged if restricted to
$t\in \bigcup_{n\in \bbZ} (-s +np, s+np)$.  Also, we have
$\Phi (\phi^{-t} x_0, t+np) = [\Phi (x_0, p)]^n \Phi (\phi^{-t} x_0, t)$.
As a result,
\begin{equation}\lb{lastin}
I_1 (y) 
 =  \sum_{n\in\bbZ} \int_{|t| \leq s} |\gam' (t+ np + \tau) | \, \|
[\Phi(x_0, p)]^n \Phi (\phi^{-t} x_0, t) v\| 
\cdot \al (\phi^{-t-np} y) \,dt.
\end{equation}
Since $\supp \gam' \subset [-N, N]$, the integral in the RHS of
\eqref{lastin} is equal to zero for $|n| > 2N/p$. To see this,
we recall that $N\geq \mu_1p+1$ and $s<1$. Since $|t|\leq s$ and
$|\tau|\leq \mu_1p$ in \eqref{lastin}, if $|n| > 2N/p$, then we have
$|t+\tau+np|\geq |n|p-s-\mu_1p\geq |n|p-N > N$.
Therefore, $I_1(y)$ does not exceed
\begin{multline*}
\sum_{|n|\leq \frac{2N}p} \int_{|t| \leq s} |\gam' (t+np + \tau)| 
( \| [\Phi (x_0, p)]^n v\| +\\ 
 \| [\Phi (x_0, p)]^n\cdot [ \Phi (\phi^{-t} x_0, t) v-v]\| ) 
\al(\phi^{-t- np} y) \, dt.
\end{multline*}
We use \eqref{2} and \eqref{D3} to estimate the last expression to obtain
\begin{equation}
I_1(y) \leq \left( 1+ \nu_2 \right) \sum_{|n|\leq 2N/p} \int_{|t|
\leq s}
|\gam'(t+np + \tau) | \al (\phi^{-t- np} y) \, dt.
\lb{Est1}
\end{equation}
Similarly, for $I_2(y)$, we have:
\begin{equation}
I_2 (y) \leq \sum_{|n|\leq 2N/p} \int_{|t| \leq s} 
| \gam' (t+ np + \tau) |
\al (\phi^{-t-np} y) \, dt.
\lb{Est2}
\end{equation}

Since $\supp\al\subset D$, 
for each integer $n$, the integration in \eqref{Est1}--\eqref{Est2}
can be restricted to $t\in\Theta_{s,D}(\phi^{-np}y)$, see
\eqref{thetam}.
Recall that, for $t\in \bbR$ and $x\in X$, 
we have $|\gam'(t)| \leq 2/N$ and 
$\al(x) \leq 1$. 
As a result, there is a constant $\mu'>0$ such that
\begin{equation}
\| I(x)\| \leq \dfrac{\mu'}N \sum_{|n| \leq 
\frac{2N}p} M_{s,D} (\phi^{-np}y).
\lb{7}
\end{equation}
Finally, we use estimate \eqref{uniform} with $x=\phi^{-np} y$
to obtain the
first inequality in \eqref{6}:
\begin{equation}\lb{eestIprime}
\| I(x) \| \leq \dfrac{\mu'}N \cdot \left[ \dfrac{2N}p\right] \cdot
KM_{s,B} (x_0) \leq \mu m,
\end{equation}
where the brackets denote the integer part.

To prove the second inequality in \eqref{6} we arrive, as 
in \eqref{eestI} above, to the
following estimate:
$$
J(x) \leq \mu_2 \left[ J_1(y) + \nu_1 J_2(y) \right],
$$
where
\begin{align}
J_1(y) &= \int_{-\infty}^\infty | \gam' (t+\tau) | \, \| \Phi (\phi^{-t}
x_0, t) w_0 (\phi^{-t} y) \| \, dt,
\lb{J1}\\
J_2(y) & = \int_{-\infty}^\infty | \gam' (t+\tau)| \, \| w_0
(\phi^{-t} y) \| \, dt.
\lb{J2}
\end{align}
We use the estimate 
$\| w_0 \|_{\bC} \leq \delta$ from Lemma~\ref{l3}.
Also, since $\gamma'$ is supported in $[-N,N]$
and in view of Lemma~\ref{l4}, the integration in $J_1(y)$ and $J_2(y)$ can
be restricted
as for $I_1(y)$ and $I_2(y)$
above, to
$t\in\bigcup_{|n|\leq 2N/p} (-s + np, s+np)$.  
Since $\supp w_0\subset D$, as in \eqref{lastin}--\eqref{Est2} above, 
we use the estimates
\eqref{D4} and \eqref{D5}
to obtain
\begin{align*}
J_1(y) &\leq  \nu_5\sum_{|n|\leq 2N/p} \int_{|t| \leq s} |\gam' (t+n p + \tau) |
\chi_D (\phi^{-t-np} y)  \, dt,\\
J_2(y) &\leq  \nu_3\sum_{|n|\leq 2N/p} \int_{|t| \leq s} |\gam' (t+n p + \tau) |
\chi_D (\phi^{-t-np} y)  \, dt,
\end{align*}
where $\chi_D(\cdot)$ is the characteristics function of $D$.  Using
\eqref{uniform}, as in
\eqref{eestIprime}, we have for some positive constants $\mu'$ and $\mu''$:
\begin{equation}
J_1(y) \leq \mu'\cdot m,\quad J_2(y) \leq \mu''\cdot m.
\lb{8}
\end{equation}
As a result, we obtain the second inequality in \eqref{6}.  
Therefore, \eqref{5} is proved.

{\bf Step 2.  The Lower Estimate.}
We will prove there is a constant $\nu>0$ such that
$$
 \| \bB \|_{\bC} \geq \| \bB (x_0)\| \geq \nu N  m.
$$
This, together with \eqref{5}, gives \eqref{3}.

Define
\begin{align*}
\ti I (x_0) &=  \left\|\int_{-\infty}^\infty \gam (t) e^{-\frac{i\xi t}p}
\Phi (\phi^{-t} x_0, t) v \cdot\al (\phi^{-t} x_0) \, dt \right\|,\\
\ti J (x_0) & = \left\| \int_{-\infty}^\infty \gam (t) e^{-\frac{i\xi
t}p} \Phi (\phi^{-t} x_0, t) w_0 (\phi^{-t} x_0)  \,dt \right\|
\end{align*}
and note that
\begin{equation}
\| \bB(x_0)\| \geq \ti I (x_0) - \ti J (x_0).
\lb{9}
\end{equation}
Recall that $\supp\gamma\subset [-N,N]$ and that, for $t\in \supp\gamma$,
we have $0\le\gamma(t)\leq 1$.
Using  arguments similar to those employed in \eqref{8} 
and \eqref{lastin}--\eqref{eestIprime}, only with the estimate
$|\gamma'(t)|\leq 2/N$ replaced by $\gamma(t)\leq 1$, 
together with an estimate 
based on the estimate
$\|w_0\|_{\bC}\leq \delta$ from Lemma~\ref{l3} and \eqref{D4},
we find there is a constant $\mu>0$ such that
\begin{align}
\ti J(x_0) & \leq
\sum\limits_{|n|\leq 2N/p}\int\limits_{|t|\leq s}
|\gamma(t+np)|\cdot \|[\Phi(x_0,p)]^n\|
\cdot \|\Phi(\phi^{-t}x_0,t)w_0(\phi^{-t}x_0)\|\,dt \nonumber\\
& \leq \nu_5
\sum\limits_{|n|\leq 2N/p}\int\limits_{|t|\leq s}
|\gamma(t+np)|\chi_D(\phi^{-t}x_0)\,dt
\leq\mu \nu_5 Nm.
\lb{10}
\end{align}
We use \eqref{2} to estimate $\ti I (x_0)$ from below as follows:
\begin{align*}
\ti I (x_0) & = \left\| \sum_{n\in\bbZ} \int_{|t| \leq s} \gam (t+ np)
e^{-\frac{i\xi t}p} [e^{-i\xi} \Phi (x_0, p) ]^n \Phi (\phi^{-t} x_0, t)
v \cdot \al (\phi^{-t} x_0) \, dt \right\|\\
& \geq \ti I_1(x_0) - \ti I_2 (x_0) - \ti I_3 (x_0),
\end{align*}
where
\begin{align*}
\ti I_1 (x_0) & = \left\| \sum_{n\in\bbZ} \int_{|t| \leq s} \gam (t+ np)
v \cdot \al (\phi^{-t} x_0) \, dt \right\|,\\
\ti I_2 (x_0) & = \left\| \sum_{n\in\bbZ} \int_{|t| \leq s} \gam (t+ np)
( e^{-\frac{i\xi t}p} -1) v \cdot \al (\phi^{-t} x_0) \, dt
\right\|,\\
\ti I_3 (x_0) & = \left\| \sum_{n\in\bbZ} \int_{|t| \leq s} \gam(t+np) [
e^{-i\xi} \Phi (x_0, p) ]^n (\Phi (\phi^{-t} x_0, t) v-v) \cdot \al
(\phi^{-t} x_0) \,dt \right\|.
\end{align*}


We will estimate $\ti I_1 (x_0)$ from below. 
Recall (see \eqref{G3}) that $\gam (\tau) \geq \frac14$ provided
 $|\tau| \leq \frac{N}2$.
If $|n| \leq \frac{N-1}{2p}$, then $|t+np| \leq s+
\frac{N-1}{2p} \cdot p \leq \frac{N}2$.  
Hence, $\gam (t+np) \geq \frac14$
for those $n$ and $|t|\leq s$. Also, $\| v\| =1$. Therefore,
\[\ti I_1 (x_0) = \sum_{n\in\bbZ} \int_{|t| \leq s} \gam (t+np) \cdot \al
(\phi^{-t} x_0) \, dt \geq
\frac{1}4\sum_{|n|\leq\frac{N-1}{2p}} \int_{|t| \leq s} \al
(\phi^{-t} x_0) \, dt.\]
Since  $\al
(\phi^{-t} x_0) =1$ for $t\in \Theta_{s,B} (x_0)$, one has:
\begin{equation}
\ti I_1 (x_0)  \geq \frac{1}4 \sum_{|n| \leq
\frac{N-1}{2p}}  
\int_{\Theta_{s,B} (x_0)}\, dt \geq \dfrac14 \cdot m \card
\left\{ n: |n| \leq \dfrac{N-1}{2p}\right\} \geq N m \nu_0
\lb{11}
\end{equation}
for some  $\nu_0 > 0$.

As in \eqref{10},  using \eqref{S1}, there is 
some $\mu'>0$ such that
\begin{equation}
\ti I_2(x_0) \leq \nu_4 \sum_{n\in \bbZ} \int_{|t| 
\leq s} \gam (t+np)\cdot \al
(\phi^{-t} x_0) \,
dt \leq \nu_4 \mu' N m.
\lb{12}
\end{equation}

Similarly,  using \eqref{D3}, we have
\begin{equation}
\ti I_3 (x_0 ) \leq \nu_2 \sum_{n\in\bbZ} \int_{|t|\leq s} \gam
(t+np) \cdot \al
(\phi^{-t} x_0)\, dt \leq \nu_2 \mu' N m.
\lb{13}
\end{equation}
Combining \eqref{9}--\eqref{13} we obtain
\[
\| \bB(x_0)\| \geq m N \left(\nu_0-\nu_5 \mu-\nu_2\mu'-\nu_4\mu' \right) 
\geq \nu N m
\]
for a constant $\nu=\nu_0-\nu_5 \mu-\nu_2\mu'-\nu_4\mu'>0$, 
provided $\nu_2, \nu_4, \nu_5$ are small enough.

We use \eqref{5} to obtain \eqref{3}, 
and the theorem for Case~1 is proved.
\quad $\Box$

\subsection{Case 2. Unbounded periods.}
Assume that, for the Man\~e point 
$\bx$ in \eqref{1}, the condition of Case~2 holds.
Also, note that this condition always holds 
provided the set of  aperiodic trajectories 
is dense in $X$. The latter situation was considered in \cite{clms}.
However, the proof of Lemma~2.7 in \cite{clms} contains, in fact, the 
following assertion with no assumption about the density of
aperiodic trajectories.
\begin{lem}\lb{aperc}
If for every
natural number $N$ there exist a point $x_0$, 
a vector $v\in\calT_{x_0}X$
with $\|v\|\geq 1/2$, and 
an open set $U\ni x_0$
such that the following holds:
\begin{itemize}
\item[{\bf (a)}]
 $\sup\{\|\Phi(y,t)v\|: y\in U, |t|\leq 8N\} < \mu$ for
a constant $\mu$, independent on $N$;
\item[{\bf (b)}]  $P(y)\geq 8N$ for all $y\in U$,
\end{itemize}
then $i\xi\in\sig_{ap}(L)$ for every $\xi\in\bbR$.
\end{lem}
\begin{pf} We will indicate how to modify 
the proof of 
\cite[Lemma~2.7]{clms} to obtain the lemma.
Fix $\xi\in\bbR$ and $N$. 
Take $x_0$, $U$ and $v$ as indicated in the statement of the lemma.
As in  \eqref{S1} and \eqref{D3}, there is a number
$s'>0$, so that for $|t|\leq s'$, 
\begin{equation}\lb{new2.18}
\|\Phi(\phi^{-t}x_0,t)v-v\| \mbox{ and  }
|e^{-i\xi t}-1| \mbox{  are small enough.}
\end{equation}
The conditions \eqref{new2.18}
must replace (2.18) and (2.19) in
\cite{clms}.
 Define $\gamma:\bbR\to [0,1]$ with $|\gamma'(t)|\leq 2/N$
and $\supp\gamma\subset [-N,N]$
such that $\gamma(t)=1$ for $|t|\leq s'$. 
Choose  $\delta$ so small that
$\delta\max_{|t|\leq 2N}\|T^t\|$ 
is small enough (cf. \eqref{D5}--\eqref{D4}).
As in Lemma~\ref{l4}, use condition (b) to find an $\eps>0$ such that
for every $V\ni x_0$ with $\diam {V}<\eps$
the segment of the trajectory
$\{\phi^ty: s'\leq |t|\leq N\}$
 does not intersect $V$ whenever $y\in V$.
Choose $V\ni x_0$ with $\diam {V}<\eps$
so that the LHS of \eqref{D1} is small enough.

Apply Lemma~\ref{l5} with $s=N$ and the prescribed choice of
$V$. Choose $B\subset D$, $D\subset V$ as in Lemma~\ref{l3} with the
 $\eps$ and $\delta$, selected above.
For $w=\al(\cdot)v+w_0$ from Lemma~\ref{l3}, we define
(cf. \eqref{defbB}):
\begin{equation}\lb{newB}
\bB(x)=\int\limits_{-\infty}^{\infty}e^{-i\xi t}
\gamma(t)T^tw(x)\,dt.
\end{equation}
As in \eqref{3} we show that
$\|L\bB-i\xi\bB\|_{\bC}
\leq \frac{\mu}N\|\bB\|_{\bC}$
for a constant $\mu$, independent on $N$. To this end,
we prove, that
\[\|(L\bB)(x)-i\xi\bB(x)\|=
\left\|\int\limits_{-\infty}^{\infty}e^{-i\xi t}
\gamma'(t)T^tw(x)\,dt\right\|\leq \frac{\mu}N M_{N,B}(x_0),\quad x\in X,\]
for some $\mu>0$, and
$\|\bB(x_0)\|\geq \nu M_{N,B}(x_0)$ for some $\nu>0$.

The arguments are similar (and easier) 
than the arguments in Step~1 and Step~2 above.
We note that the integration in \eqref{newB}
is unchanged if restricted to $t\in\Theta_{N,D}(x)$. 
Also, $\bB(x)=0$ provided $x\not\in\cup_{|t|\leq 2N}\phi^t(D)$.
Besides, for  $s'$ as above, the segment of the trajectory 
$\{\phi^ty: s'\leq |t|\leq N\}$,
 $y\in D$ does not intersect $D\subset V$. As a result, 
the segment of the trajectory $\{\phi^tx: |t|\leq 2N\}$, $x\in X$
spends not more than time $2s'$ in $D$, and
we do not need to replace the integral in 
\eqref{newB} by a sum as in \eqref{lastin}.
To obtain the desired estimates, we use 
condition~(a)  similarly to \eqref{D21} above. 
Finally, condition~(a) combined with 
\eqref{new2.18} is used in the estimates
as in \eqref{D3} and 
\eqref{D1} above.\end{pf}

To finish the proof of  Theorem~\ref{t3}, it remains to
show that in  Case~2 conditions (a) and (b) in Lemma~\ref{aperc}
hold. We use \eqref{1} to define 
$\mu=2\sup\{\|\Phi(\bx,t)\bv\|: t\in\bbR\}$.
Fix $N\in\bbN$. Let $\bar{P}=9N$, $v=\bv$, and use the continuity
of $\Phi$ to find $U\ni \bx$ such that (a) 
in Lemma~\ref{aperc} holds. Since we are in  Case 2,
for this $U$ and $\bar{P}$ there is a point 
$x_0\in U$ with $P(x_0)\geq 9N$.
The prime periods function $P$   is lower semicontinuous.
Hence, we can find a smaller neighborhood  $U'\subset U$ of $x_0$, 
so that (b)  in Lemma~\ref{aperc} holds. 
By Lemma~\ref{aperc}  $\sig_{ap} (L,\cnd) \cap i\bbR \neq \emptyset$,
and Theorem~\ref{t3} for Case~2 is proved.


\subsection{Proof of Theorem~\ref{spectr}.}
Recall that we must prove
\begin{equation}\lb{spechul2}
 \calH(\sig(T, \cnd))=\calH_C,
 \end{equation}
where $T=T^1$ and
$\calH_C=\{z\in\bbC: r_- \leq|z| \leq r_+\}$ denotes
the smallest  annulus containing $\sig(T, \bC)$. 
Using the notation
$\rs(A)=\rs(A,E)$ to denote the
spectral radius of an operator $A$ on a Banach space $E$, 
we clearly have
\[
r_+  =  \rs(T,\bC),\quad
r_-  =  \left(\rs(T^{-1}, \bC)\right)^{-1}=
\inf\{|\lam|:\lam\in\sig(T,\bC)\}.
\]

\begin{pf} Since  $\sig_{ap}(T,\cnd)\subset \sig_{ap}(T,\bC)$,
 obviously, $\calH(\sig_{ap}(T,\cnd))
\subset \calH(\sig_{ap}(T, \bC))$. We will show that, in fact,
\begin{equation}\lb{hulls}
\calH(\sig_{ap}(T,\cnd))
= \calH(\sig_{ap}(T, \bC)).
\end{equation}

The hull $\calH(\sig_{ap}(T, \bC))$ consists of a union of circles
centered at the origin of the complex plane. If
$\Gamma_r:=\{z\in\bbC:|z|=r\}\subset \calH(\sig_{ap}(T, \bC))$, then 
there is a point $\lam:=re^{i\theta}\in \sig_{ap}(T, \bC)$. 
Consider the group $\tilde{T}^t=r^{-t}e^{-i\theta t}T^t$ generated by
$\tilde{L}=-\ln r-i\theta+L$. 
Since $1\in\sig_{ap}(\tilde{T}, \bC)$,
Theorem~\ref{t3} applied to $\{\tilde{T}^t\}$ implies there is
some $\xi\in\bbR$ such that 
\[
i\xi\in\sig_{ap}(\tilde{L}, \cnd)=
                             -\ln r-i\theta+\sig_{ap}(L, \cnd).
\] 
By the Spectral Inclusion Theorem for $\{T^t\}$ on $\cnd$,
we have $re^{i(\xi+\theta)}\in\sig_{ap}(T,\cnd)$, and
$\Gamma_r\subset \calH(\sig_{ap}(T,\cnd))$. This proves
\eqref{hulls}.

Since the boundary of $\sig(T, \cnd)$ belongs to
$\sig_{ap}(T, \cnd)$, the inclusion
$\calH(\sig(T, \cnd))\subset\calH_C$ in \eqref{spechul2}
follows directly from \eqref{hulls}. To prove the inverse inclusion,
suppose $\Gamma_r\subset\calH_C$ but
$\sig(T, \cnd)\cap\Gamma_r=\emptyset$.
By rescaling, we can and will assume that $r=1$, that is $r_-<1<r_+$ and 
$\sig(T, \cnd)\cap\bbT=\emptyset$. Let $P_0$
denote the Riesz projection for the operator $T$ on $\cnd$
corresponding to the part of $\sig(T,\cnd)$ 
that lies inside of the unit disc $\bbD$. We {\it claim}
that both $\sig(T,\cnd)\cap\bbD$ and $\sig(T,\cnd)\cap(\bbC\setminus\bbD)$
are nonempty. Indeed, since $r_-<1<r_+$,
the set $\calH(\sig_{ap}(T, \bC))$ contains a circle
with radius greater than one and a circle
with the radius less than one. By \eqref{hulls}, 
$\calH(\sig_{ap}(T,\cnd))$ also contains these circles. 
This proves the claim.

As in \cite[Theorem~3.4]{clms},
one can show that  there is an extension of $P_0$ 
to the Riesz projection ${\cal P}$ 
for the operator $T$ on $\bC$. It follows, 
see \cite{CS,LS,Mather}, that this projector $\cal P$ has a form
${\cal P}\bB(x)=P_C(x)\bB(x)$, $\bB\in\bC$
where $P_C: X\to \mbox{ proj }(\calT_xX)$
is a continuous projection-valued function. Since ${\cal P}=P_0$
on $\cnd$, multiplication by $P_C(\cdot)$ preserves
the set of divergence-free vector fields. 
This, see \cite[Theorem~3.4]{clms},
leads to the fact that $P_C(x)$ is either the identity or the zero
operator, and, as a result, the same is true for the operator $P_0$,
in contradiction to the above claim.

To construct ${\cal P}$ on $\bC$ one 
can use the proof of \cite[Theorem~3.4]{clms}. The idea is 
to define $\calP$  on a dense subset of $\bC$ consisting of  
``step-sections''  of
the form $\bB(\cdot)=\sum\rho_k(\cdot)v_k$, where $\{\rho_k\}$ is
a partition of unity, and $v_k$ are given vectors.
For each $k$ we use Lemma~\ref{l3} to find $w_k\in\cnd$ such that 
$w_k(x)=v_k$ for $x\in\supp\rho_k$. We then 
define $\calP\bB=\sum\rho_k P_0w_k$. 
As in the proof of \cite[Theorem~3.4]{clms}, 
this definition does not depend on the choice of
$w_k$ and, moreover, the projection $\calP$ can be extended to a bounded
linear operator on $\bC$. In fact, 
the Steps~1,2,4,5 of the cited proof go through with no assumption
on density of the aperiodic trajectories of $\phi^t$. To modify Step~3
of the cited proof for the case when this assumption does not hold,
we use the sufficiency part of Lemma~\ref{l1} to conclude that
$1\in\sig_{ap}(T, \bC)$ instead of $\bbT\subset\sig_{ap}(T,\bC)$
as in \cite[Theorem~3.4, Step~3]{clms}. By \eqref{hulls}, this
contradicts the hyperbolicity of $T$ on $\cnd$, as required in
Step~3 of \cite[Theorem~3.4]{clms}.
\end{pf}

\subsection{Example.}
We give an example of an Eulerian vector-field $u$, 
so that the spectral mapping
theorem for the corresponding group $\{T^t\}$ does not hold. For this,
consider the three dimensional torus $\bbT^3$, viewed as $\bbR^3/\bbZ^3$
with coordinates $(x,y,z)$ in $\bbR^3$ modulo $2\pi$, and recall
Euler's equations are 
\begin{equation}\lb{Eul}
u_t+(u\cdot\nabla)u=-\frac{1}{\rho}\nabla {\bold p},\quad \div u=0,
\end{equation}
where $\rho$ is the fluid density and $\bold p$ is the pressure.
The constant
vector field $u$ given by $u(x,y,z)=(1,0,0)$ together with a constant
pressure provides a (steady state) solution for \eqref{Eul}.
Define $\calL$ to be the Lie derivative in the direction of $u$.
We will show that the spectral mapping theorem, considered in the
space $\cnd$ of divergence-free vector fields on $\bbT^3$,
does not hold for the 
corresponding group $T^t=e^{t\calL}$ generated by $\calL$, and 
that $\sig(T^t)$ and $\sig(L)$ are not invariant with respect to arbitrary
rotations and translations.

The tangent bundle of $\bbT^3$ is trivial, so we consider 
the elements of the Banach space $\cnd$ of divergence-free sections
as maps $f:\bbT^3\to\bbC^3$. With this representation,
we have
\[
(\calL f)(x,y,z)=\frac{\partial}{\partial x}f(x,y,z),
\quad (T^tf)(x)=f(x+t,y,z)
\]
where the operations are taken componentwise.

The spectrum of $\calL$ is $\{ik : k\in\bbZ\}$. To see this, 
note first that $F(x,y,z):=(0,e^{ikx},0)$
is a divergence-free eigenfunction of $\calL$ with eigenvalue $ik$.
On the other hand, if $\lambda$ is not of the form $ik$, then
$\calL-\lambda I$ is invertible. In fact, the inverse is given by
\[
(\calL-\lambda I)^{-1}g(x,y,z)=\frac{e^{2\pi\lambda}}{1-e^{2\pi\lambda}}
\int_0^{2\pi}e^{-\lambda s}g(s+x,y,z)\,ds.
\]
Thus, $\sig(L)=i\xi+\sig(L)$ does not hold
for all $\xi\in\bbR$

Let $t$ denote a real number that is incommensurate with $2\pi$.
Then $\sig(T^t)=\bbT$.
This follows from the Spectral Inclusion Theorem and the fact that
the spectrum of the bounded operator $T^t$ is closed. Since
$\exp t\sigma(\calL)$ is not
the whole circle, the spectral mapping theorem fails, as required. 
It is also easy to give a direct verification that $\sig(T^t)=\bbT$.
 For this,
choose a point $\lambda=e^{i\theta}\in\bbT$. 
If $\epsilon>0$, then there is a integer $k$ so that
$|e^{ikt}-e^{i\theta}|<\epsilon$. But, for $F$ defined above,
\[\|(T^tF)(x,y,z)-e^{i\theta}F(x,y,z)\|=|e^{ikt}-e^{i\theta}|<\epsilon.\] 
Thus, $e^{i\theta}\in\sig_{ap}(T^t)$.

If $t/2\pi$ is a rational number $q/p$, $q\in\bbZ$, $p\in\bbN$,
then $\sig(T)=\{z\in\bbC:z^p=1\}$  and, as required, 
is not  invariant with respect
to all rotations centered at origin. 
However, $\calH(\sig(T^t))=\bbT=\calH(\exp t\sig(L))$.


\begin{thebibliography}{XXXX}

\bi{Arnold} V.~I.~Arnold,
{\em Some remarks on the antidynamo theorem,}
Moscow University Mathem.~Bull., {\bf 6} (1982) 50--57.
 
\bi{AZRS} V.~I.~Arnold, Ya.~B.~Zel'dovich, A.~A.~Rasumaikin, and
D.~D.~Sokolov, {\em Magnetic field in a stationary flow with
stretching in Riemannian space,} Sov.~Phys.~JETP, {\bf 54} (6)
(1981) 1083--1086.
 
\bi{BC} B.~J.~Bayly and S.~Childress,
{\em Fast-dynamo action in unsteady flows and maps in three
dimensions,}
Phy.~Rev.~Let. {\bf 59} (14) (1987) 1573--1576.
 

\bi{clms}
C.~Chicone, Y.~Latushkin, and S.~Montgomery-Smith,
 {\it The spectrum of the
kinematic dynamic operator for an ideally conducting fluid\/}, Comm.\
Math.\ Phys., to appear.
 
\bibitem{CS} C. Chicone and R. Swanson,
{\em Spectral theory for linearization of dynamical systems,}
J.~Diff.~Eqns.,  {\bf 40} (1981) 155--167.

\bi{Rafael}  R.~de~la~Llave,
{\em Hyperbolic dynamical systems and
generation of magnetic fields by perfectly conducting fluids,}
Geophys. Astrophys. Fluid Dynamics,
{\bf 73} (1993) 123--131.
 
\bi{FV} S.~Friedlander and M.~Vishik,
{\em Dynamo theory methods for hydrodynamic stability},
J. Math. Pures Appl. {\bf 72} (1993) 145--180.

\bi{HPS} M.~Hirsch, C.~Pugh, and M.~Shub,
{\em Invariant Manifolds,}
Lect. Notes Math. {\bf 583} (1977).
 
\bi{J} R.~Johnson,
{\em Analyticity of spectral subbundles},
J.~Diff.~ Eqns., {\bf 35} (1980) 366--387.

\bi{KY} I.~Klapper and L.~S.~Young,
{\em Rigorous bounds on the fast dynamo growth rate involving
topological entropy,} Preprint (1994) 1--30.
 
\bi{LMS2} Y.~Latushkin and  S.~Montgomery-Smith,
{\em Lyapunov theorems for Banach spaces,}
Bull.~AMS {\bf 31} (1) (1994) 44--49.

\bi{LMSFun} Y.~Latushkin and  S.~Montgomery-Smith,
{\em Evolutionary semigroups and Lyapunov theorems in Banach Spaces,}
 
\bibitem{LS} Y.~Latushkin and A.~M.~Stepin,
{\em Weighted composition
operators and linear extensions of dynamical systems,}
Russian Math.~Surveys, {\bf 46}, no. 2, (1992) 95--165.
 
\bi{Mather} J.~Mather,
{\em Characterization of Anosov diffeomorphisms,}
Indag.~Math., {\bf 30} (1968) 479--483.
 
\bi{Mane} R.~Man\~e,
{\em Quasi-Anosov diffeomorphisms and hyperbolic manifolds,}
Trans. Amer. Math. Soc., {\bf 229} (1977) 351--370.
 
\bi{Moffatt} H.~K.~Moffatt,
{\em Magnetic Field Generation in
Electrically Conducting Fluids,} Cambridge Univ. Press, Cambridge,
1978.
 
\bi{MRS} S.~A.~Molchanov, A.~A.~Ruzmaikin, and D.~D.~Sokolov,
{\em Kinematic dynamo in random flow,} Sov.~Phys.~Usp. {\bf 28} (4)
(1985) 307--327.

\bi{Mont} S.~Montgomery-Smith,
{\em Stability and dichotomy of positive semigroups on $L_p$,}
preprint.

\bi{Nagel} R.~Nagel (ed.) {\em One-parameter Semigroups of
Positive Operators,}
Lect. Notes Math., {\bf 1184}, Springer-Verlag, NY, 1986.

\bi{Nunez} M.~Nu\'n\~ez, {\em Localized eigenmodes of the induction
equation,} SIAM J. Appl. Math., {\bf 54}, no. 5 (1994) 1254--1267.


%\bi{OselMET}  V.~I.~Oseledec,
%{\em A multiplicative ergodic
%theorem: Lyapunov characteristic numbers for dynamical systems,}
%Trans.~Moscow Math.~Soc., {\bf 19} (1968) 197--231.
% 
% 
%\bi{Osel1} V.~I.~Oseledec, $\Lambda$-entropy and the anti-dynamo
%theorem, In: {\it Proc. 6th Intern. Symp. on Information Theory,}
%Part III, Tashkent, (1984) 162--163.

\bi{Pazy} A.~Pazy,
{\em Semigroups of Linear Operators and Applications
to Partial Differential Equations,}
Springer-Verlag, N.Y./Berlin, 1983. 
 
\bibitem{Vishik} M.~M.~Vishik,
{\em Magnetic field generation by the motion of
a highly conducting fluid},
Geophys.~Astrophys.~Fluid Dynamics,
{\bf 48} (1989) 151--167.
 
\end{thebibliography}
\end{document}




Recall that, by the Multiplicative Ergodic Theorem \cite{OselMET}, 
for each ergodic measure $\nu\in\calE$,
there exists a set $X_\nu \subset X$ with $\nu(X_\nu)=1$ such that
for each $x\in X_\nu$ and $v\in\calT_xX$ 
there exist exact Lyapunov exponents for the cocycle $\Phi$:
\[\lam_\nu(x,v)=\lim_{t\to\pm\infty}\frac1t\ln\|\Phi(x,t)v\|.\]
For each $\nu$, there may exist $n'=n'(\nu)\leq n$
different Lyapunov exponents; we will denote them by
$\lam_\nu^1>\lam_\nu^1>\dots>\lam_\nu^{n'}$.
In \cite{LS} (see also \cite{Osel1}) the numbers $r_+$ and $r_-$ above 
were computed via exact Lyapunov exponents
 with respect to the set of $\phi^t$-ergodic
measures $\nu\in\calE$ as follows:
\begin{equation}\lb{lex}
\ln r_+=\sup\{\lam_\nu^1: \nu\in\calE\},\quad
\ln r_-=\inf\{\lam_\nu^{n'}: \nu\in\calE\},
\end{equation}
There exist measures such that $\sup$ and $\inf$ are attained.
As a result, formulae \eqref{lex}  allow one to compute
$\calH(\sig(T,\cnd)$ and the vertical strip consisting
of the line intersecting $\sig(L,\cnd)$.

Recall (see, e.g., \cite{Arnold,AZRS,BC}) that the
kinematic dynamo is called
``fast'' provided 
\[
\limsup_{\varepsilon\to\infty}\omega(\calL_\varepsilon)\quad \mbox{
is positive. }\]
M.~Vishik \cite{Vishik} gave the following sufficient condition for
the non-existence of a fast kinematic dynamo:
Define the Lyapunov numbers
\[
\bar{\lambda}(x,u)=\limsup_{t\to\infty}
\dfrac{1}{t}\ln\|D\phi^t(x)u\|.
\]
If
\begin{equation}\lb{vish}
\sup \{
\bar{\lambda}(x,u):
x\in X, u\in \calT_x X
\}\leq 0,
\end{equation}
then there is no fast kinematic dynamo.
Using \eqref{lex}, we can show that
the sufficient condition given by the inequality
\eqref{vish} can be replace by an analogous inequality where
the supremum is computed over exact Lyapunov exponents.



--PART-BOUNDARY=.19502011148.ZM28630.cs.missouri.edu--



