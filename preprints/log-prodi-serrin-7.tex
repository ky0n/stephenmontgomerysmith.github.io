\documentclass[12pt]{amsart}
\usepackage{amsmath,amssymb,url}
%\usepackage{setspace}
%\onehalfspacing

\newtheorem{thm}{Theorem}[section]
\newtheorem{prop}[thm]{Proposition}
\newtheorem{lemma}[thm]{Lemma}
\newtheorem{cor}[thm]{Corollary}
\newtheorem{defn}[thm]{Definition}

\theoremstyle{remark}
\newtheorem{rem}[thm]{Remark}

\newcommand{\N}{{\mathbb N}}
\newcommand{\E}{{\mathbb E}}
\newcommand{\R}{{\mathbb R}}
\newcommand{\C}{{\mathbb C}}
\newcommand{\T}{{\mathbb T}}

\newcommand{\modo}[1]{{\left|#1\right|}}
\newcommand{\normo}[1]{{\left\|#1\right\|}}
\newcommand{\smodo}[1]{{\mathopen|#1\mathclose|}}
\newcommand{\snormo}[1]{{\mathopen\|#1\mathclose\|}}

\DeclareMathOperator{\curl}{curl}
\DeclareMathOperator{\divergence}{div}

\begin{document}
\title[Regularity of Navier-Stokes]
{A condition implying regularity of the three dimensional Navier-Stokes
equation}
\author{Stephen Montgomery-Smith}
\makeatletter
\address{Department of Mathematics,
University of Missouri,
Columbia, MO 65211}
\email{stephen@math.missouri.edu}
\urladdr{\url{http://www.math.missouri.edu/~stephen}}
\thanks{The author was
partially supported
by an NSF grant.}
\keywords{Navier-Stokes equation, vorticity, 
Prodi-Serrin condition, Orlicz norm, stochastic methods}
\subjclass[2000]{Primary 35Q30, 76D05, Secondary 60H30, 46E30}

\begin{abstract}
It is shown that if $u$ is the solution to the
three dimensional Navier-Stokes equation, then a sufficient condition
for regularity is that 
$\int_0^T {\snormo{u(t)}_q^p}/{(1+\log^+ \snormo{u(t)}_q)}
 \, dt < \infty $,
for all $T>0$, and some
$2 < p < \infty$, $3 < q < \infty$,
$\frac2p+\frac3q=1$.  This represents a logarithmic
improvement over the usual Prodi-Serrin conditions.
\end{abstract}

\maketitle

\section{Introduction}

The version of the three dimensional
Navier-Stokes equation we will study is the
differential equation in
$u = u(t) = u(x,t)$, where $ t\ge 0$, and $x \in \R^3$:
$$ \frac{\partial u}{\partial t}
   = \Delta u - u \cdot \nabla u + \nabla P ,
   \quad \divergence u = 0 ,
   \quad u(0) = u_0.$$
We will also work with the vorticity form.  For the remainder of the paper
we denote $w = w(t) = w(x,t) = \curl u$.
Then
$$ \frac{\partial w}{\partial t}
   = \Delta w - u \cdot \nabla w + w \cdot \nabla u ,
   \quad w(0) = \curl u_0 .$$
A famous open problem is to prove regularity of the Navier-Stokes
equation, that is, if the initial data $u_0$ is in $L_2$ and is
regular (which in this paper we will define to mean that it is
in the Sobolev spaces $W^{n,q}$ for some $2\le q<\infty$ and all positive
integers $n$), then the solution $u(t)$ is regular for all
$t\ge0$.  Such regularity would also imply uniqueness of the solution
$u(t)$.  Currently the existence of weak solutions is known.
Also, it is known that for each regular $u_0$ that there exists
$t_0>0$ such that $u(t)$ is regular for $0\le t \le t_0$.
We refer the reader to 
\cite{cannone}, \cite{constantin-foias}, \cite{doering-gibbons},
\cite{lemarie-rieusset}, \cite{temam}.

In studying this problem various conditions that imply regularity
have been obtained.  For example, the 
Prodi-Serrin conditions (\cite{prodi}, \cite{serrin})
state that for some
$2 \le p < \infty$, $3<q\le\infty$ with
$\frac2p + \frac 3q \le 1$ that
$$ \int_0^T \snormo{u(t)}_q^p \, dt < \infty $$
for all $T>0$.
If $u$ is
a weak solution to the Navier-Stokes equation 
satisfying a Prodi-Serrin condition,
with regular initial
data $u_0$, then $u$ is regular (see \cite{sohr}).
(Recently Escauriaza, Seregin and Sver\'ak \cite{escauriaza et al}
showed that the condition when $q=3$ and $p=\infty$ is also sufficient.)
This is a long way from what is currently known for weak
solutions:
$$ \int_0^T \snormo{u(t)}_q^p \, dt < \infty $$
for $\frac2p + \frac 3q \ge \frac32$, $2 \le q \le 6$.
The purpose of this paper is to slightly reduce this rather
large gap as follows.

%Dimensions: $t = L^2$, $\smodo x = L$, $u = 1/L$, 
%$\snormo u_q = L^{3/q-1} = L^{-2/p}$, $\nabla^n u = 1/L^{n+1}$, 
%$\snormo{\nabla u}_2 = L^{1/2-n}$
%
%Usual Prodi-Serrin conditions
%have dimension $L^0$.
%What is known is that $\sup_t \snormo{u(t)}_2<\infty$ 
%which has dimension $L^{1/2}$.
%Need to close the gap between $L^0$ and $L^{1/2}$.  

\begin{thm} \label{main}
Let 
$2 < p < \infty$, $3<q < \infty$ with
$\frac2p+\frac3q=1$.
If $u$ is a solution to the Navier-Stokes equation satisfying
$$ \int_0^T \frac{\snormo{u(t)}_q^p}{1+\log^+\snormo{u(t)}_q}
   \, dt < \infty $$
for some $T>0$, 
then $u(t)$ 
is regular for $0 < t\le T$.
\end{thm}

The hypothesis of Theorem~\ref{main} imply that, 
given $\epsilon \in (0,T)$, there exists
$t' \in (0,\epsilon)$ with $u(t') \in L_q$.
Then by known results (for example Theorem~\ref{space analytic} below), 
it follows
that there exists $0 < T_0<\epsilon$ such that 
$\snormo{w(T_0)}_{r}$ is bounded for all $r \in [q,\infty]$.

Let $T^* > T_0$ be the first point of non-regularity for $u(t)$.
It is well known that 
in order to show that $T^* > T$,
it is sufficient to show an \emph{a priori} estimate, that is
$\sup_{T_0 \le t < \min\{T^*,T\}} \snormo{w(t)}_q < \infty$.
This is because it is then possible to extend the regularity beyond
$T^*$ if $T^* \le T$.
Without loss of generality, it is sufficient to consider the case
$T = T^*$ (so as to obtain a contradiction), and this we do
in the remainder of the paper.

Also, from now on, let us fix $p$ and $q$ satisfying the hypothesis of 
Theorem~\ref{main}, and allow all constants to implicitly
depend upon $p$ and $q$.

\section{A Stochastic Description}

This is a more rigorous formulation of the description
given in \cite{montgomery-smith-pokorny}.  As we have just stated, we are
supposing that $u(t)$ is regular for all $T_0 \le t < T$.

If $f\colon\R^3 \to \R$ is regular, and
$T_0 \le t_0 \le t_1 < T$,
define $A_{t_0,t_1} f(x) = \alpha(x,t_1)$, where 
$\alpha$ satisfies the transport equation
$$ \frac{\partial\alpha}{\partial t} = \Delta \alpha - u\cdot\nabla \alpha,
   \qquad
   \alpha(x,t_0) = f(x) .$$
Since $\divergence(u) = 0$, an easy integration by parts argument shows
that
$$ \frac\partial{\partial t} \int \alpha(x,t) \, dx = 0 ,$$
and hence if $f$ is also in $L_1$, then
$$ \int A_{t_0,t_1} f(x) \, dx = \int f(x) \, dx .$$
Since stochastic
differential equations traditionally move forwards in time, it will be 
convenient to consider a time reversed equation.
Let $b(t)$ be three dimensional Brownian motion.
For $T_0 \le t_0 \le t_1 < T_1$, define the random function
$\varphi_{t_0,t_1}\colon\R^3\to\R^3$ by
$\varphi_{t_0,t_1}(x) = X(-t_0)$, where $X$ satisfies the 
stochastic differential equation:
$$ dX(t) = u(X(t),t) \, dt + \sqrt2 \, db(t),
   \qquad
   X(-t_1) = x .$$
It follows by the Ito Calculus \cite{karatzas-shreve} that
if $T_0 \le t_0 \le t_1 < T$, then
$$ A_{t_0,t_1} f(x) = \E f(\varphi_{t_0,t_1}(x)) .$$
(Here as in the rest of the paper, $\E$ denotes expected value.)
Note that if $f$ is also in $L_1$, then
$$ \int \E f(\varphi_{t_0,t_1}(x)) \, dx = \int f(x) \, dx .$$
Applying the usual dominated and monotone convergence theorems, it
quickly follows that the last equality is also true if $f$ is any
function in $L_1$, or if $f$
is any positive function.

Now, we note that $w$ is the unique solution to the integral
equation
$$ w(t) = A_{T_0,t} w(T_0) + 
   \int_{T_0}^t A_{s,t} (w(s) \cdot \nabla u(s)) \, ds 
   \quad (T_0 \le t < T).$$
Uniqueness follows quickly by the usual fixed point argument
over short intervals, 
remembering that $u(t)$ is regular for $T_0 \le t < T$.

Consider also the random quantity
$\tilde w = \tilde w(x,t)$ as the solution to the integral equation
for $T_0 \le t < T$
$$ \tilde w(x,t) = w(\varphi_{T_0,t}(x),T_0) +
   \int_{T_0}^t \tilde 
   w(\varphi_{s,t}(x),s) \cdot \nabla u(\varphi_{s,t}(x),s) \, ds .$$
Again, 
it is very easy to show that a solution exists by using a fixed point
argument over short time intervals.

It is seen that $\E\tilde w$ satisfies the same equation as $w$, and
hence $\E \tilde w = w$.
By Gronwall's inequality, if $T_0 \le t < T$
$$ \smodo{\tilde w(x,t)}
   \le
   \exp\left(\int_{T_0}^t \smodo{\nabla u(\varphi_{s,t}(x),s)} \, ds\right)
   \smodo{w(\varphi_{T_0,t}(x),T_0)} .$$
(This is essentially the Feynman-Kac formula.)
The goal, then, is to find uniform estimates on the quantity
$$ \exp\left(\int_{T_0}^t \smodo{\nabla u(\varphi_{s,t}(x),s)} \, ds\right) .$$
This we will proceed to do in the next section.

\begin{rem}
Let us give a little motivation.  If instead we defined $\varphi_{t_0,t_1}(x)$
to be $X(-t_0)$, where $X$ satisfies
the equation
$$ dX(t) = u(X(t),t) \, dt,
   \qquad
   X(-t_1) = x ,$$
then $\varphi_{t_0,t_1}$ would be the ``back to coordinates map'' that
takes a point at $t=t_1$ to where it was carried from by the flow of the
fluid at time $t=t_0$.
See for
example \cite{constantin}.  Thus with the addition of Brownian
motion, $\varphi_{t_0,t_1}$ 
becomes a ``randomly perturbed back to coordinates map.''
In essence, we are considering a kind of Lagrangian approach to the
Navier-Stokes equation.
\end{rem}

\section{Inequalities}

Let us 
introduce some functions defined for $\lambda \ge 0$:
$$ \Phi(\lambda) = \left(\frac{e^\lambda-1}{e-1}\right)^q ,\qquad
   \Theta(\lambda) = \frac\lambda{2+\log^+\lambda} .$$
Notice that for any $a>0$, that
there exist constants $c_1,c_2>0$ such that
$c_1 \Theta(\lambda) \le \lambda/(1+\Phi^{-1}(\lambda^a)) 
\le c_2 \Theta(\lambda)$;
that $\Theta$ is a strictly increasing bijection on $[0,\infty)$;
and that $\Theta$ and $\Theta^{-1}$ obey a moderate growth condition, that is,
given $c_1>1$ there exists $c_2,c_3>1$ such that $c_2 \Theta(\lambda) \le
\Theta(c_1 \lambda) \le c_3 \Theta(\lambda)$.

We define the
$\Phi$-Orlicz norm on
the space of measurable functions by the formula
$$ \snormo f_\Phi = 
   \inf\left\{\lambda>0:
   \int \Phi(\smodo{f(x)}/\lambda) \, dx \le 1 \right\} .$$
The triangle inequality is a consequence of the fact that
$\Phi$ is convex (see \cite{kras-rutickii}).
We extend the definition of the Orlicz norm to random
functions $F$ as follows
$$ \snormo F_\Phi = 
   \inf\left\{\lambda>0:
   \int \E \Phi(\smodo{F(x)}/\lambda) \, dx \le 1 \right\} .$$
Using the notation from the previous section,
we see for $T_0 \le t_0 \le t_1 < T$ that
$\snormo{f\circ\varphi_{t_0,t_1}}_\Phi = \snormo f_\Phi$.

Let us fix the function
$$ M(\lambda) = 
   \int_{\{t\in[0,T] \colon \snormo{u(t)}_q \ge \lambda\}}
   \Theta(\snormo{u(t)}_q^p) \, dt .$$
The hypothesis of Theorem~\ref{main} tells us that
 $M(\lambda) \to 0$ as $\lambda \to \infty$.

The following result is very much related to the \emph{a priori} estimates
obtained in \cite{foias et al}.

\begin{lemma}
\label{localization to integral}
There are constants $c_1,c_2,c_3,c_4>0$ 
such that if
$\lambda>\max\{c_1,T^{-1}\}$,
then
$$ \int_{\{t \in [\lambda^{-1},T] \colon
         \snormo{\nabla u(t)}_\infty \ge c_2 \lambda\}}
   \snormo{\nabla u(t)}_{\Phi} \, dt \le c_3 M(c_4\lambda^{1/p}) .$$
\end{lemma}

Let us first show how to use this result.

\begin{proof}[Proof of Theorem~\ref{main}]
By Lemma~\ref{localization to integral}, there exists $\lambda>T_0^{-1}$ such
that
$$ \int_{B}
   \snormo{\nabla u(t)}_{\Phi} \, dt \le \frac1q .$$
where 
$B = \{t \in [T_0,T] \colon \snormo{\nabla u(t)}_\infty \ge c_2 \lambda\}$.
Thus for $T_0 \le t < T$, we have that $\smodo{\tilde w(x,t)}$ is bounded
by
$$ e^{c_2 \lambda (t-T_0)}
   \exp\left(\int_{B\cap[T_0,t]} 
   \smodo{\nabla u(\varphi_{s,t}(x),s)} \, ds\right)
   \smodo{w(\varphi_{T_0,t}(x),T_0)} .$$
Hence by Jensen's and
H\"older's inequalities, $\snormo{w(t)}_{q}^{q} \le 
\int \E\smodo{\tilde w(t)}^{q} \, dx \le 
e^{c_2 q \lambda (t-T_0)}(N_q^q + N_{qq'}^q \tilde N)$, where
$q' = q/(q-1)$, 
$$
   N_r
   =
   \left(\int \E \smodo{w(\varphi_{T_0,t}(x),T_0)}^r\,dx\right)^{1/r} 
   = \snormo{w(T_0)}_r \qquad (r \ge 1), $$
and
$$
   \tilde N
   =
   \int \E \left(
   \exp\left(q\int_{B \cap [T_0,t]}
   \modo{\nabla u(\varphi_{s,t}(x),s)} \, ds \right) - 1 \right)^q
   \, dx .
$$
Since the Orlicz norm satisfies the triangle inequality, 
$$ \normo{\int_{B\cap[T_0,t]} \smodo{\nabla u(\varphi_{s,t}(\cdot),s)} \, ds}
    _\Phi \le \frac1q ,$$
that is, $\tilde N \le (e-1)^q $.
Since $a^q + b^q \le (a+b)^q$ for $a,b \ge 0$,
we conclude that 
$$ \snormo{w(t)}_q \le \snormo{w(T_0)}_q + 
   (e-1) e^{c_2 \lambda (t-T_0)} \snormo{w(T_0)}_{qq'} ,$$
and the result follows.
\end{proof}

\begin{lemma}  \label{upper bound for L_Phi}
There is a constant $c>0$ such that if $f$ is a measurable
function, then
$$
  \snormo f_\Phi \le c
  \left(\snormo f_q + 
  \frac{\snormo f_\infty}{1+\Phi^{-1}(({\snormo f_\infty}/{\snormo f_q})^q)}
  \right) .$$
\end{lemma}

\begin{proof}
Let us assume that $\snormo f_\infty = 1$, 
and set $a = \snormo f_q$, $b = \Phi^{-1}(a^{-q})$ and
$n = a+1/(1+b)$.  
Let $f^*:[0,\infty]\to[0,\infty]$ be the non-increasing rearrangement
of $\modo{f}$, that is,
$$ f^*(t) =  
   \sup \{ \lambda>0 : \smodo{\{ x:\smodo{f(x)}>\lambda\}} > t \} ,$$
so $\int F(\modo{f(x)}) \, dx = \int_0^\infty F(f^*(t)) \, dt$ for any
Borel measurable function $F$.  Notice that $f^*(t) \le \min\{1,a t^{-1/q}\}$.

Let us first
consider the case $a \le 1$, so that $b \ge 1$, $2n \ge 1/b$, and
$n \ge a$.
Then
$$
   \int \Phi(\modo{f(x)}/2n) \, dx
   \le
   \int_0^\infty \Phi(f^*(t)/2n) \, dt .$$
We split this integral up into three pieces.  First,
$$ 
   \int_0^{a^q} \Phi(f^*(t)/2n) \, dt 
   \le
   \int_0^{a^q} \Phi(b) \, dt 
   = 1.
$$
Next, since $(\Phi(\lambda))^{1/2q}$ is convex for $\lambda \ge 1$,
\begin{align*}
   \int_{a^q}^{a^q b^q} \Phi(f^*(t)/2n) \, dt 
   &\le
   \int_{a^q}^{a^q b^q} \Phi(abt^{-1/q}) \, dt \\
   &\le 
   \int_{a^q}^{a^q b^q} \frac{a^{2q}\Phi(b)}{t^2} \, dt  \\
   &\le 1.
\end{align*}
Next, for $t \ge a^q b^q$, $f^*(t) \le 1/b \le 2n$, and
$\Phi(\lambda) \le \lambda^q$ for $0 \le \lambda \le 1$, so
$$
   \int_{a^q b^q}^\infty \Phi(f^*(t)/2n) \, dt 
   \le
   \int_{a^q b^q}^\infty (f^*(t)/2n)^q \, dt 
   \le 1.
$$
Since $\Phi(\lambda/3) \le \Phi(\lambda)/3$ for
$\lambda\ge0$,
$$
   \int \Phi(\modo{f(x)}/6n) \, dx
   \le 1 ,
$$
that is, $\snormo f_\Phi \le 6 n$.

The case $a \ge 1$ (so $b \le 1$ and $2n \ge 1+2a$) is simpler, as
it is easy to estimate
$$
   \int_0^\infty \Phi(f^*(t)/2n) \, dt 
   \le
   \int_0^1 \Phi(1) \, dt 
   +
   \int_1^\infty (f^*(t)/2n)^q \, dt 
   \le 2.
$$

\end{proof}

The following result is due to 
Gruji\'c and Kukavica \cite{grujic-kukavica}.

\begin{thm} \label{space analytic}
There exist constants $a,c>0$ and a function
$T:(0,\infty) \to (0,\infty)$, with $T(\lambda) \to \infty$ as $\lambda\to 0$,
with the following properties.  If $u_0 \in L_q(\R^3)$, then there is
a solution $u(t)$ $(0 \le t \le T(\snormo{u_0}_q))$
to the Navier-Stokes equation, with $u(0) = u_0$, and
$u(x,t)$ is the restriction of an analytic function 
$u(x+iy,t) + iv(x+iy,t)$ in the region
$\{x+iy \in \C^3 : \smodo y \le a \sqrt t\}$, and
$\snormo{u(\cdot+iy,t) + iv(\cdot+iy,t)}_q \le c \snormo{u_0}_q$ for
$\smodo y \le a \sqrt t$.
\end{thm}

This next result is related to Scheffer's Theorem \cite{scheffer} 
that states that the Hausdorff dimension
of the set of $t$ for which the solution $u(t)$ is not regular is
$1/2$.

\begin{lemma}  \label{localization}
There exists constants
$0<c_5<1<c_6$, such
that if $0<r<1$, and
$u(t)$ is a regular
solution
to the Navier-Stokes equation with
$$ \smodo{\{t \in [t_0-r^2,t_0] : \snormo{u(t)}_q \ge c_5 r^{-2/p}\}} 
   < c_5 r^2 ,$$
then
$$ \snormo{\nabla u(t_0)}_{\Phi}
   + \Theta(\snormo{\nabla u(t_0)}_\infty)
   < c_6 \Theta(r^{-2}) .$$
\end{lemma}

\begin{proof}
Let us first consider the case when $t_0 = 0$ and $r = 1$.
By hypothesis, we see that there exists
$t \in [-1,-1+c_5]$
with
$ \snormo{w(t)}_2 < c_5$.  
By Theorem~\ref{space analytic} and the appropriate Cauchy integrals, 
if $c_5$
is small enough, then there exists a constant $c_7>0$ such that
$\snormo{\nabla u(0)}_q < c_7$  
and
$\snormo{\nabla u(0)}_\infty < c_7$  

Next, by replacing 
$u(x,t)$
by
$r^{-1} u(r^{-1}x,r^{-2}(t-t_0))$, we can relax the restriction $r=1$ and
$t_0=0$, and we see that
$\snormo{\nabla u(t_0)}_q < c_7 r^{-2/p-1}$
and
$\snormo{\nabla u(t_0)}_{\infty} < c_7 r^{-2}$.
Finally, the result follows
from Lemma~\ref{upper bound for L_Phi},
and some simple estimates.
\end{proof}

\begin{proof}[Proof of Lemma~\ref{localization to integral}]
Let $c_5$ and $c_6$ be as in Lemma~\ref{localization}.
Define the sets
$$ A_\mu = \{t \in [1/\Theta^{-1}(\mu),T]:
   \snormo{\nabla u(t)}_\Phi + 
   \Theta(\snormo{\nabla u(t)}_\infty) \ge 
   c_6 \mu \} ,$$
and
$$ B_\mu 
   =
   \{t \in [0,T]:
   \Theta(\snormo{u(t)}_q^p) \ge \mu\} .$$
Notice that
$$
   \int_{\{t \in [0,T] \colon
         \snormo{u(t)}_q \ge \lambda^{1/p}\}}
   \Theta(\snormo{u(t)}_q^p) \, dt
   =
   \Theta(\lambda) \smodo{B_{\Theta(\lambda)}}
   +
   \int_{\Theta(\lambda)}^\infty \smodo{B_\mu} \, d\mu ,
$$
and similarly if $c_2>0$ is chosen to always exceed 
$\mu^{-1} \Theta^{-1}(c_6 \Theta(\mu))$ for $\mu > 0$, then
$$
   c_6^{-1} \int_{\{t \in [\lambda^{-1},T] \colon
         \snormo{\nabla u(t)}_\infty \ge c_2 \lambda\}}
   \snormo{\nabla u(t)}_{\Phi} \, dt
   \le 
   \Theta(\lambda) \smodo{A_{\Theta(\lambda)}}
   + 
   \int_{\Theta(\lambda)}^\infty \smodo{A_\mu} \, d\mu .
$$
Thus it is sufficient to show the existence of constants $c_8,c_9>0$ such
that for $\mu > 1$ that $\modo{A_\mu} < c_8 \smodo{B_{c_9\mu}}$.
Define $r$ by the relation $\mu = \Theta(r^{-2})$.  Note that
$0 < r < 1$.  Then
$$ A_\mu = \{t \in [r^2,T]:
   \snormo{\nabla u(t)}_\Phi + 
   \Theta(\snormo{\nabla u(t)}_\infty) \ge 
   c_6 \Theta(r^{-2}) \} ,$$
and for sufficiently small $c_9$
$$ B_{c_9\mu} \supset \{t \in [0,T]:\snormo{u(t)}_q \ge c_5 r^{-2/p}\} .$$
It is trivial to find a finite collection $t_1,\dots,t_N$
in the closure of $A_r$
such that the disjoint sets $(t_n-r^2,t_n]$ cover $A_r$.
Furthermore, by Lemma~\ref{localization}
$$ \smodo{\{t \in [t_n-r^2,t_n] : \snormo{u(t)}_q \ge c_5 r^{-2/p}\}} 
   < c_5 r^2 .$$
Hence
$$ \smodo{A_\mu}
   \le
   N r^2
   <
   c_5^{-1} \sum_{n=1}^N \smodo{[t_n-r^2,t_n] \cap B_{c_9\mu}}
   \le
   c_5^{-1} \smodo{B_{c_9\mu}} .
$$
\end{proof}

\section*{Acknowledgments}

The author wishes to extend his sincere gratitude to Michael Taksar for
help with understanding stochastic processes.


\begin{thebibliography}{999}

\bibitem{cannone}
M.~Cannone,  Ondelettes, paraproduits et Navier-Stokes, (French) 
[Wavelets, paraproducts and Navier-Stokes], 
with a preface by Yves Meyer, 
Diderot Editeur, Paris, 1995.

\bibitem{constantin}
P.~Constantin,
An Eulerian-Lagrangian approach to the Navier-Stokes equations, 
Comm. Math. Phys.  216  (2001), 663--686.

\bibitem{constantin-foias}
P.~Constantin and C.~Foia\c s, 
Navier-Stokes equations, 
Chicago Lectures in Mathematics,
University of Chicago Press, Chicago, IL, 1988.

\bibitem{doering-gibbons}
C.R.~Doering and J.D.~Gibbon,
Applied analysis of the Navier-Stokes equations,
Cambridge Texts in Applied Mathematics,
Cambridge University Press, Cambridge, 1995.

\bibitem{escauriaza et al}
L.~Escauriaza, G.~Seregin and V.~Sver\'ak,
On 
$L_{3,\infty}$-solutions to the Navier-Stokes equations and backward 
uniqueness,
preprint, \url{http://www.ima.umn.edu/preprints/dec2002/dec2002.html}.

\bibitem{foias et al}
C.~Foia\c s, C.~Guillop\'e and R.~Temam,
New a priori estimates for Navier-Stokes equations in dimension $3$,
Comm. Partial Differential Equations 6 (1981), no. 3, 329--359.

\bibitem{grujic-kukavica}
Z.~Gruji\'c and I.~Kukavica,
Space analyticity for the Navier-Stokes and related 
equations with initial data in $L\sp p$,
J. Funct. Anal.  152  (1998), 447--466.

\bibitem{karatzas-shreve}
I.~Karatzas and S.E.~Shreve,
Brownian motion and stochastic calculus,
second edition. Graduate Texts in Mathematics, 113,
Springer-Verlag, New York, 1991.

\bibitem{kras-rutickii}
M.A.~Krasnosel'ski{\u\i} and Ja.B.~Ruticki\u\i,
Convex functions and Orlicz spaces,
translated from the first Russian edition by Leo F. Boron, 
P. Noordhoff Ltd., 
Groningen 1961.

\bibitem{lemarie-rieusset}
P.G.~Lemari\'e-Rieusset,
Recent developments in the Navier-Stokes problem,
Chapman and Hall/CRC, 2002.

\bibitem{montgomery-smith-pokorny}
S.J.~Montgomery-Smith and M.~Pokorn\'y,
A counterexample to the smoothness of the solution to an 
equation arising in fluid mechanics,
Comment. Math. Univ. Carolin.  43  (2002),  61--75.

\bibitem{prodi}
G.~Prodi,
Un teorema di unicit\`a per le equazioni di Navier-Stokes,
Ann. Mat. Pura Appl. (4)  48  (1959) 173--182.

\bibitem{scheffer}
V.~Scheffer,
Turbulence and Hausdorff dimension, 
Turbulence and Navier-Stokes equations
(Proc. Conf., Univ. Paris-Sud, Orsay, 1975), 174--183,
Lecture Notes in Math., Vol. 565, Springer, Berlin, 1976. 

\bibitem{serrin}
J.~Serrin,
On the interior regularity of weak solutions of the Navier-Stokes equations,
Arch. Rational Mech. Anal.  9  (1962), 187--195.

\bibitem{sohr}
H.~Sohr,
Zur Regularit\"atstheorie der instation\"aren Gleichungen von Navier-Stokes, 
Math. Z.  184  (1983),  no. 3, 359--375.

\bibitem{temam}
R.~Temam,
Infinite-dimensional dynamical systems in mechanics and physics,
second edition, Applied Mathematical Sciences, 68,
Springer-Verlag, New York, 1997.

\end{thebibliography}

\end{document}
