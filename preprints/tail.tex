\magnification=\magstep1
\baselineskip =5.5mm
\lineskiplimit =1.0mm
\lineskip =1.0mm

\def\sqr{\vcenter {\hrule height.3mm
\hbox {\vrule width.3mm height 2mm \kern2mm
\vrule width.3mm } \hrule height.3mm }}

\def\Proof:{\medskip\noindent{\bf Proof:}\ \ \ }
\def\endproof{\hfill$\sqr$\medskip}

\def\Bbb{\bf}
\def\N{{\Bbb N}}

\def\ds#1{{\displaystyle{#1}}}
\def\ts#1{{\textstyle{#1}}}

\def\list#1{#1_1$, $#1_2,\ldots}
\def\lists#1,#2{#1_1$, $#1_2,\ldots,$\ $#1_{#2}}
\def\xlist{\list x}
\def\epsilonlist{\list\varepsilon}
\def\gammalist{\list\gamma}

\def\seqo#1{(#1_n)_{n=1}^\infty}
\def\seqx{\seqo x}

\def\recipo#1{{1\over #1}}
\def\smallrecipo#1{\ts{\recipo{#1}}}
\def\half{{\recipo2}}
\def\smallhalf{\smallrecipo2}
\def\smallthird{\smallrecipo3}

\def\invo#1{#1^{-1}}
\def\invc{\invo c}

\def\modo#1{\left| #1 \right|}
\def\modxn{\modo{x_n}}

\def\normo#1{\left\| #1 \right\|}
\def\normx{\normo x}

\def\sumt{\sum_{m=1}^t}
\def\sumtt{\sum_{m=1}^{t^2}}
\def\sumBm{\sum_{n\in B_m}}
\def\sumnm{\sum_{n=n_m+1}^{n_{m+1}}}

\def\rmss_#1^#2#3{\left(\sum_{#1}^{#2} #3^2\right)^\half}
\def\rms#1{\rmss_{}^{}{#1}}
\def\rmsBm#1{\rmss_{n\in B_m}^{}{#1}}
\def\rmsnm#1{\rmss_{n=n_{m-1}+1}^{n_m}{#1}}

\def\Kotto#1{K_{1,2}(#1,t)}
\def\Jitto#1{J_{\infty,2}(#1,t)}
\def\Jitinvto#1{J_{\infty,2}(#1,\invo t)}
\def\Kottx{\Kotto x}
\def\Jittx{\Jitto x}
\def\Jitinvtx{\Jitinvto x}


\def\Pt{{P(t)}}
\def\Ptt{{P(t^2)}}

\def\Prepsilon#1{\Pr\left(\sum \varepsilon_n x_n > #1 \right)}
\def\Prgamma#1{\Pr\left(\sum \gamma_n x_n > #1 \right)}

\centerline{\bf The Distribution of Rademacher Sums}

\bigskip
\centerline{\vbox{\halign{\hfil#\hfil\cr
S.J.~Montgomery-Smith,\cr
Department of Mathematics,\cr
University of Missouri at Columbia,\cr
Columbia, MO 65211, U.S.A.\cr}}}
\bigskip

\beginsection Abstract

We find upper and lower bounds for $\Pr\left(\sum \pm x_n \ge t\right)$,
where $\xlist$\ are real numbers. We express the answer in terms of the
$K$-interpolation norm from the theory of interpolation of Banach spaces.

\beginsection Acknowledgements

All the work in this paper appears in my Ph.D.\ thesis [6], taken at
Cambridge University under the supervision of Dr.~D.J.H.~Garling, to
whom I would like to express my thanks. I would also like to express my
gratitude to the Science and Engineering Research Council, who financed
my studies in Cambridge.

\beginsection Primary A.M.S.\ Classification

60C05

\beginsection Secondary A.M.S.\ Classification

60G50, 60J15

\beginsection Key Words and Phrases

Rademacher Sum, Holmstedt's formula.

\bigskip\bigskip

\centerline{Revised August 30, 1989}

\vfill
\eject

\beginsection Introduction

Throughout this paper, we let $\epsilonlist$\ be independent Bernoulli
random variables (that is, $\Pr(\varepsilon_n=1) =
\Pr(\varepsilon_n=-1) = 1/2$).
We are going to look for upper and lower bounds for
$\Prepsilon t$, where $\xlist$\ is a sequence of real numbers such
that $x = \seqx \in l_2$.

Our first upper bound is well known (see, for
example, Chapter~II, \S59 of [5]):
$$ \Prepsilon{t\normx_2} \le e^{-\half t^2} . \eqno(1)$$
However, if $\normx_1 < \infty$, this cannot also provide a good
lower bound, because then we have another upper bound:
$$ \Prepsilon{\normx_1} = 0 . \eqno(2)$$
To look for lower bounds, we might first consider using some version of
the central limit theorem. For example, using Theorem~7.1.4 of [2], it
can be shown that for some constant $c$\ we have
$$ \modo{ \Prepsilon{t\normx_2}
          - \ts{\recipo{\sqrt{2\pi}}} \int_t^\infty e^{-\half s^2} \,ds }
   \le c \, \ts{\left({\normx_3\over\normx_2}\right)^3} .$$
Thus, for some constant $c$\ we have that if
$t\le \invc \bigl(\log\normx_3/\normx_2\bigr)^\half$, then
$$ \Prepsilon{t\normx_2}
   \ge \invc \int_t^\infty e^{-\half s^2} \,ds
   \ge {c^{-2} e^{-\half t^2}\over t} .$$
However, we should hope for far more. From $(1)$\ and $(2)$, we could
conjecture something like
$$ \Prepsilon{\invc \inf\{\normx_1,t\normx_2\}} \ge \invc e^{-ct^2} .$$
Actually such a conjecture is unreasonable---one should not take infimums
of norms, but instead one should consider the following quantity:
$$ K(x,t;l_1,l_2) =
   \Kottx = \inf \{\, \normo{x'}_1 + t \normo{x''}_2 : x',x''\in
                      l_2, \, x'+x''=x \} . $$
This norm is well known to the theory of interpolation of Banach spaces
(see, for example [1] or [3]). For small $t$, this norm looks a
lot like $t\normx_2$, but as $t$\ gets much larger, it starts to look
more like $\normx_1$. In fact, there is a rather nice approximate
formula due to T.~Holmstedt (Theorem~4.1 of [3]):
\def\ttaa_#1{\modo{x_#1}}%
if we write $\seqo{x^*}$\ for the sequence $\seqo{\ttaa}$\
rearranged into decreasing order, then
$$ \invc \Kottx
   \le \sum_{n=1}^{\lfloor t^2 \rfloor} x_n^* + t \rmss_{n=\lfloor
   t^2 \rfloor + 1}^\infty {(x_n^*)}
   \le \Kottx ,$$
where $c$\ is a universal constant.

\bigskip

In this paper, we will prove the following result.

\proclaim Theorem. There is a constant $c$\ such that for all
$x\in l_2$\ and $t>0$\ we have
$$ \Prepsilon{\Kottx} \le e^{-\half t^2} $$
and
$$ \Prepsilon{\invc\Kottx} \ge \invc e^{-c t^2} .$$

An interesting example is
$x=(n^{-1})_{n=1}^\infty$. Then $\invc \log t \le \Kottx \le c \, \log
t$, and hence
$$ \invc \exp(-\exp(ct))
   \le \Pr\left(\sum\varepsilon_n n^{-1} > t \right)
   \le c \, \exp(-\exp(\invc t)) .$$
This is quite different behaviour then that which we might have expected
from the central limit theorem.

We might also consider $x=(n^{-{1\over p}})_{n=1}^\infty$, where $1 < p <
2$. This example leads us to deduce Proposition~2.1 of [7]. More
involved methods allow us to rederive the results of [8] (which include
the above mentioned result from [7]). We do not go into details.

We also deduce the following
corollary.

\proclaim Corollary. There is a constant $c$\ such that for all
$x\in l_2$\ and $0<t\le{\normx_2/\normx_\infty}$\ we have
$$ \Prepsilon{\invc t \normx_2} \ge \invc e^{-c t^2} .$$

\Proof: It is sufficient to show that there is a constant $c$\ such
that if $0 < t \le {\normx_2/\normx_\infty}$, then
$$ \Kottx \le t \normx_2 \le c \, \Kottx .$$
The left hand inequality follows straight away from the definition of
$\Kottx$. The right hand side
follows easily from Holmstedt's formula; obviously if $t<1$, and
otherwise because
$$ \sum_{n=1}^{\lfloor t^2 \rfloor} x_n^* \ge
   \lfloor t^2 \rfloor {\normx_2 \over t} \ge \ts{t\over2} \normx_2
   .$$
\endproof

\beginsection Proof of Theorem

In order to prove the theorem, we will need some new norms on $l_2$, and
a few lemmas.

\proclaim Definition. For $x\in l_2$\ and $t>0$, define the norm
$$ J(x,t;l_\infty,l_2) =
   \Jittx = \max \{ \normx_\infty ,\, t \normx_2 \} .$$

\proclaim Lemma 1. For $t>0$, the spaces
$\bigl(l_2,\Kotto{\,\cdot\,}\bigr)$\ and
$\bigl(l_2,\Jitinvto{\,\cdot\,}\bigr)$\ are dual to one another,
that is, for $x\in l_2$\ we have
$$ \Kottx = \sup \left\{\, \sum x_n y_n : y\in l_2 ,\,
   \Jitinvto y \le 1 \right\} .$$

\Proof: This is elementary (see, for example Chapter~3, Exercise
1--6 of [1]).
\endproof

\proclaim Definition. For $x\in l_2$\ and $t\in\N$, define the norm
$$ \normx_\Pt = \sup \left\{ \sumt \rmsBm \modxn \right\} ,$$
where the supremum is taken over all disjoint subsets, $\lists B,t
\subseteq \N$.

\proclaim Lemma 2. If $x\in l_2$\ and $t^2\in\N$, then
$$ \normx_\Ptt \le \Kottx \le \sqrt2 \, \normx_\Ptt .$$

\Proof: To show the first inequality, note that we have
$$ \normx_\Ptt \le \normx_1 \quad\hbox{and}\quad
   \normx_\Ptt \le t \normx_2 .$$
Hence
$$ \eqalignno{
   \Kottx
   &= \inf \{\, \normo{x'}_1 + t \normo{x''}_2 : x'+x'' = x \} \cr
   &\ge \inf \{\, \normo{x'}_\Ptt + \normo{x''}_\Ptt : x'+x''=x \} \cr
   &\ge \normx_\Ptt ,\cr}$$
where the last step follows by the triangle inequality.

For the second inequality, we start by using Lemma~1. For any
$\delta>0$, let $y\in l_2$\ be such that
$$ (1-\delta) \Kottx \le \sum x_n y_n
   \hbox{\quad and\quad}
   \Jitinvto y = 1 .$$
Now pick numbers $n_0$, $\lists n,{t^2} \in \{0$, $1$, $2,\ldots,$\
$\infty\}$ by induction as follows: given $0=n_0<n_1<\ldots<n_m$, let
$$ n_{m+1} = 1 + \sup \left\{\, \nu :
   \sum_{n=n_m+1}^\nu \modo{y_n}^2 \le 1 \right\} .$$
Since $\normo y_\infty \le 1$, we have that $\sumnm \modo{y_n}^2 \le 2$.
Also, as $\normo y_2 \le t$, it follows that $n_{t^2} = \infty$.
Therefore
$$ \eqalignno{
   (1-\delta) \Kottx
   &\le \sum x_n y_n \cr
   &\le \sumtt \rmsnm{\modo{y_n}} \rmsnm \modxn \cr
   &\le \sqrt2 \, \normx_\Ptt .\cr}$$
Since this is true for all $\delta>0$, the result follows.
\endproof

The following lemma is due to Paley and Zygmund.

\proclaim Lemma 3. If $x\in l_2$, then given $0<\lambda<1$\ we have
$$ \Prepsilon{\lambda\normx_2} \ge \smallthird (1-\lambda^2)^2 .$$

\Proof: See Chapter~3, Theorem~3 of [4].
\endproof

\bigskip

Now we proceed with the proof of the theorem. First we will show that
$$ \Prepsilon{\Kottx} \le e^{-\half t^2} .$$
Given $\delta>0$, let $x'$, $x''\in l_2$\ be such that $x'+x''=x$, and
$$ (1+\delta) \Kottx > \normo{x'}_1 + t\normo{x''}_2 .$$
Then
$$ \eqalignno{
   \Prepsilon{(1+\delta) \Kottx}
   &\le \Pr\left(\sum \varepsilon_n x'_n > \normo{x'}_1 \right) \cr
   &\quad
   + \Pr\left(\sum \varepsilon_n x''_n > t\normo{x''}_2 \right) \cr
   &\le 0+e^{-\half t^2} ,\cr}$$
where the last inequality follows from equations $(1)$\ and $(2)$\ above.
Letting $\delta \to 0$, the result follows.

Now we show that for some constant $c$\ we have
$$ \Prepsilon{\invc \Kottx} \ge \invc e^{-c t^2} .$$
First, let us assume that $t^2\in\N$.
Given $\delta>0$, let $\lists B,{t^2} \subseteq \N$\ be disjoint subsets
such that $\bigcup_{m=1}^{t^2} B_m = \N$\ and
$$ \normx_\Ptt \le (1+\delta) \sumtt \rmsBm \modxn .$$
Then
$$ \Prepsilon{\smallhalf\Kottx}
   \ge \Prepsilon{\smallrecipo{\sqrt2} \normx_\Ptt} $$
$$ \ge \Pr \left( \sumtt \sumBm \epsilon_n x_n
   \ge \smallrecipo{\sqrt2} (1+\delta) \sumtt \rmsBm \modxn \right) $$
$$ \ge \prod_{m=1}^{t^2} \Pr \left( \sumBm \varepsilon_n x_n \ge
       \smallrecipo{\sqrt2} (1+\delta) \rmsBm \modxn \right) $$
$$ \ge \left( \smallthird \left( 1 - \smallhalf (1+\delta)^2
   \right)^2 \right)^{t^2} ,$$
where the last step is from Lemma~3. If we let $\delta\to0$, then we see
that
$$ \Prepsilon{\smallhalf\Kottx} \ge \exp\bigl(-(\log 12) \,
   t^2\bigr) .$$
This proves the result for $t^2 \in \N$. For $t\ge1$, note that
$$ \Kottx \le K_{1,2}(x,\lceil t \rceil)
\hbox{\qquad and \qquad}
\lceil t \rceil^2 \le 4t^2 ,$$
and hence the result follows (with $c=4\,\log 12$). For $t<1$, the
result may be deduced straightaway from Holmstedt's formula and Lemma~3.
\endproof

\beginsection References

\halign{#\hfil & \quad\vtop{\hsize=5.5 truein%
\parindent=0pt\hangindent=1em \strut#\strut\smallskip}\cr
[1] & C.~Bennett and R.~Sharpley,\sl\ Interpolation of Operators,\rm\
Academic Press.\cr
[2] & K.L.~Chung,\sl\ A Course in Probability Theory (2nd.\ Ed.),\rm\
Academic Press.\cr
[3] & T.~Holmstedt,\rm\ Interpolation of quasi-normed spaces,\sl\ Math.\
Scand.\ {\bf 26} (1970), 177--199.\cr
[4] & J-P.~Kahane,\sl\ Some Random Series of Functions,\rm\ Cambridge
studies in advanced mathematics 5.\cr
[5] & P-A.~Meyer,\sl\ Martingales and Stochastic Integrals~I,\rm\
Springer-Verlag 284.\cr
[6] & S.J.~Montgomery-Smith,\sl\ The Cotype of Operators from $C(K)$,\rm\
Ph.D.\ thesis, Cambridge, August 1988.\cr
[7] & G.~Pisier,\sl\ De nouvelles caract\'erisations des ensembles de
Sidon,\rm\ Mathematical Analysis and Applications, Advances in Math.\
Suppl.\ Stud., 7B (1981), 686--725.\cr
[8] & V.A.~Rodin and E.M.~Semyonov,\rm\ Rademacher series in symmetric
spaces,\rm\ Analyse Math.\ {\bf 1} (1975), 207--222.\cr
}

\bye
