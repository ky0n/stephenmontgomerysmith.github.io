\documentclass[12pt]{amsart}

\newtheorem{thm}{Theorem}
\newtheorem{cor}{Corollary}
\newtheorem{lem}{Lemma}

\begin{document}
\title[Extension to a Martingale Inequality]{An Extension to the Tangent Sequence Martingale Inequality}
\author{Stephen Montgomery-Smith}
\address{Department of Mathematics,\\
University of Missouri\\
Columbia, Missouri 65211, USA.\\
\tt stephen@math.missouri.edu}
\author{ Shih-Chi Shen }
\address{Department of Mathematics,\\
University of Missouri\\
Columbia, Missouri 65211, USA.\\
\tt mathgr75@math.missouri.edu} \maketitle

\begin{abstract}
For each $1<p<\infty$, there exists a positive constant $c_p$, 
depending only on $p$, such
that the following holds.  Let $(d_k)$, $(e_k)$ be real-valued martingale
difference sequences. If for for all bounded nonnegative predictable 
sequence $(s_k)$ and all positive integers $k$ we have
\[E[s_k \vee |e_k|]\leq E[s_k \vee |d_k|]\]
then for all positive integers $n$ we have
\[ \left\| \sum_{k=1}^n e_k \right\|_p 
\leq  c_p \left\| \sum_{k=1}^n d_k \right\|_p .\]
\end{abstract}
\maketitle

\section{Introduction}
Let $(\Omega ,\mathcal{F},P)$ be a probability space, and let
$(\mathcal{F}_k)$ be a filtration on $(\Omega ,\mathcal{F},P)$. 
(We will suppose that $\mathcal{F}_0 = \{\emptyset,\Omega\}$.)
If an adapted sequence $(d_k)$
is a real-valued martingale difference sequence, Burkholder's
inequality \cite{B2} shows that for any $1<p<\infty$ that there exists
a positive constant $c_p$ depending
only on $p$ such that such that
for all $\varepsilon _k \in \{1,-1\}$ and all positive integers $n$ that
\[ \left\| \sum_{k=1}^n \varepsilon _k d_k \right\|_p 
\leq c_p \left\| \sum_{k=1}^n d_k \right\|_p .\]
More generally, if $(v_k)$ is a predictable sequence bounded
in absolute value by 1, then
\[ \left \| \sum_{k=1}^n v_k d_k \right\|_p 
\leq c_p \left\| \sum_{k=1}^n d_k \right\|_p .\]

A different approach to this inequality was proposed by
Kwapie\'n and Woyczinski \cite{KW1} (see also \cite{KW2}).
Two adapted sequence $(f_k)$ and $(g_k)$ are said to be tangent if
for each $k\geq 1$, we have that the law of $f_k$ conditionally on
$ \mathcal{F}_{k-1} $ is the same as the law of $g_k$
conditionally on $\mathcal{F}_{k-1}$, that is,
\[P(f_k<\lambda | \mathcal{F}_{k-1})= P(g_k<\lambda | \mathcal{F}_{k-1})\]
for all real numbers $\lambda $. 
Answering a conjecture of Kwapie\'n and Woyczinski \cite{KW1},
it was proved by Hitczenko \cite{H1} (see also \cite{Z})
that for $1<p<\infty$ that there exists a positive constant $c_p$, 
depending only
on $p$, such that if $(d_k)$ and
$(e_k)$ are martingale difference sequences and $(d_k)$, $(e_k)$
are tangent, then for all positive integers $n$
\begin{equation}
\label{ek<=dk}
\left\|\sum_{k=1}^n e_k \right\|_p \leq c_p \left\| \sum_{k=1}^n
d_k \right\|_p.
\end{equation}

The purpose of this paper is to provide a common generalization to these
two results.

\begin{thm}
\label{t ek<=dk sk}
For each $1<p<\infty$, there exists a positive constant $c_p$, 
depending only on $p$, such
that the following holds.  
Let $(d_k)$, $(e_k)$ be real-valued martingale
difference sequences. If for for all bounded nonnegative predictable 
sequence $(s_k)$ and all positive integers $k$ we have
\begin{equation} 
\label{skek<=skdk}
E[s_k\vee |e_k|]\leq E[s_k \vee |d_k|]
\end{equation}
then for all positive integers $n$ we have
equation~(\ref{ek<=dk}).
\end{thm}

This is essentially equivalent to another result, which concerns martingales
in a specific situation.  We will consider the probability space 
$[0,1]^\mathbf{N}$ equipped with the product Lebesgue measure $\mathcal L$,
and consider the filtration $(\mathcal{L}_k)$, where $\mathcal L_k$
is the minimal $\sigma$-field
for which the first $k$ coordinate functions of $[0,1]^{\mathbf{N}}$ are
measurable.  Then two sequences $(d_k)$ and $(e_k)$ are tangent
if 
\[ e_k (x_1,...,x_k)=d_k(x_1,...,x_{k-1},\phi_k(x_1,...,x_k))\]
where $(\phi_k:[0,1]^k\rightarrow [0,1])$ is a sequence of
measurable functions such that $\phi_k(x_1,...,x_{k-1},\cdot)$ is
a measure preserving map for almost all $x_1$, ..., $x_{k-1}$. 

We will consider a more general situation.  Suppose we have a
sequence of linear operators $(T_k(x_1,...,x_{k-1}))$, 
depending measurably upon $(x_k) \in [0,1]^{\mathbf{N}}$,
that are bounded operators on both 
$L_1[0,1]$ and $L_\infty [0,1]$ with norm 1.  Then consider the
condition
\begin{equation}
\label{ek=Tk(dk)}
e_k(x_1,...,x_{k-1}, \cdot
)=[T_k(x_1,...,x_{k-1})]d_k(x_1,...,x_{k-1}, \cdot ).
\end{equation}

\begin{thm}
\label{t ek<=dk T}
For each $1<p<\infty$, there exists a positive constant $c_p$, 
depending only on $p$, such
that the following holds.
If $(d_k)$, $(e_k)$ and $(T_k)$ be as above satisfying (\ref{ek=Tk(dk)}),
then for all positive integers $n$ we have
equation~(\ref{ek<=dk}).
\end{thm}

We will also need the following intermediate result.
For any random variable $f$, 
let $f^{\#}$ be the decreasing rearrangement of $|f|$, that is,
\[f^{\#}(t)=\sup\{ s\in \mathbf{R}: P(|f|<s)<t\}.\]

\begin{thm}
\label{t ek<=dk sharp}
For each $1<p<\infty$, there exists a positive constant $c_p$, 
depending only on $p$, such
that the following holds.
Let $(d_k)$, $(e_k)$ be martingale difference sequences on
$[0,1]^\mathbf{N}$ with respect to $(\mathcal{L}_k)$. 
Suppose that for each positive integer $k$
\[\int_0^t (e_k(x_1,...,x_{k-1},\cdot))^{\#}(s)ds 
\leq \int_0^t  (d_k(x_1,...,x_{k-1},\cdot))^{\#}(s)ds\]
for all $t\in [0,1]$ and almost all $x_1$,...,$x_{k-1}$.
Then for all positive integers $n$ we have
equation~(\ref{ek<=dk}).
\end{thm}

\section{The Discrete Type Case}

In this section we will prove Theorems~\ref{t ek<=dk sk}, \ref{t ek<=dk T}
and~\ref{t ek<=dk sharp} in a special discrete situation, which we now describe.
For any positive integer $N$, let $\Sigma $ be the $\sigma
$-field generated by the partition $\{[\frac{i-1}{N},\frac{i}{N});
i=1,2,...,N\}$. Define a filtration on $[0,1]^{\mathbf N}$
by $(\mathcal{F}_k)$ by
$\mathcal{F}_k=\mathcal{L}_{k-1}\times \Sigma $. Suppose $(d_k)$,
$(e_k)$ are $(\mathcal{F}_k)$-adapted.  Then for each $k$ and for
each $x_1$, ..., $x_{k-1}$, we see that $d_k(x_1,...,x_{k-1},\cdot)$ and
$e_k(x_1, ...,x_{k-1},\cdot)$ are $\Sigma$-measurable simple
functions on $[0,1)$. Therefore $d_k$ and $e_k$ can be written as
$N$-dimensional vectors and $T_k(x_1, ...,x_{k-1})$ can be
represented by a $N\times N$ matrix, that is,
\begin{equation}
\label{[e]=[a][d]}
\left[ \begin{array}{c} e_k(1) \\ e_k(2) \\ \vdots \\ e_k(N)
\end{array} \right ]
= \left[\begin{array}{ccc} a_k(1,1), & ... & ,a_k(1,N) \\
a_k(2,1), & ... & ,a_k(2,N) \\ \vdots & & \vdots \\ a_k(N,1), &
... & ,a_k(N,N)\end{array}\right] \left[ \begin{array}{c} d_k(1)
\\ d_k(2) \\ \vdots \\ d_k(N) \end{array} \right]
\end{equation} 
where
\[d_k(i)=d_k(x_1,...,x_{k-1},i)=d_k(x_1,...,x_{k})
\mbox{ if $x_k \in [\frac{i-1}{N},\frac{i}{N})$}\]
\[e_k(i)=e_k(x_1,...,x_{k-1},i)=e_k(x_1,...,x_{k})
\mbox{ if $x_k \in [\frac{i-1}{N},\frac{i}{N})$}\]
\[T_k=T_k(x_1, ...,x_{k-1})=\left[(a_k(x_1, ...,x_{k-1}))(i,j)\right]_{N\times N}=\left[a_k(i,j)\right]_{N\times N}\]
The condition of being martingale difference sequences implies that
\[ \sum_{i=1}^N d_k(i)= \sum_{i=1}^N e_k(i)=0\]
We will prove Theorem~\ref{t ek<=dk T} to in this discrete setting.

\begin{thm}
\label{t ek<=dk T disc}
Theorem~\ref{t ek<=dk T} holds in the case that
$(d_k)$ and $(e_k)$ are
adapted to the filtration $(\mathcal F_k)$ described above.
\end{thm}

In this discrete case, the boundedness of
$\|T_k\|_{L_1}$ and $\|T_k\|_{L_\infty }$ by 1 is equivalent to
the condition that $\sum_{j=1}^N |a_k(i,j)|\leq 1 $ for all $i$
and $\sum_{i=1}^N |a_k(i,j)|\leq 1$ for all $j$. We claim that
without loss of generality, we can assume that every row sum and
column sum of $T_k$ is 0, that is,
\[ \sum_{j=1}^N a_k(i,j)= \sum_{i=1}^N a_k(i,j)=0\]
for all $i$ and $j$. 
Suppose the $i^{th}$ row sum $\sum_{j=1}^N
a_k(i,j)=R_k(i)$. Let $T'_k$ be the liner operator defined by
\[T'_k=\left[a_k(i,j)-\frac{R_k(i)}{N} \right]_{N\times N}\]
It is clear that every row sum of $T'_k$ is 0 and
\begin{eqnarray*}
(T'_kd_k)(i)&=& \sum_{j=1}^N \left(a_k(i,j)-\frac{R_k(i)}{N} \right)d_k(j)\\
            &=& \sum_{j=1}^N a_k(i,j)d_k(j)-\frac{R_k(i)}{N} \sum_{j=1}^N d_k(j)\\
            &=& e_k(i)
\end{eqnarray*}

Now we can assume that every row sum of $T_k$ is 0. Similarly
suppose the $j^{th}$ column sum $\sum_{i=1}^N a_k(i,j)=C_k(j)$.
Let $T''_k$ be the linear operator defined by
\[T''_k=\left[a_k(i,j)-\frac{C_k(j)}{N} \right]_{N\times N}\]
Again it is clear that every row sum and column sum of $T''_k$ is
0 and
\begin{eqnarray*}
(T''_kd_k)(i)&=& \sum_{j=1}^N \left(a_k(i,j)-\frac{C_k(j)}{N} \right)d_k(j)\\
            &=& \sum_{j=1}^N a_k(i,j)d_k(j)-\frac{1}{N} \sum_{j=1}^N C_k(j)d_k(j)\\
            &=& e_k(i)
\end{eqnarray*}
since
\[\sum_{i=1}^N e_k(i)= \sum_{j=1}^N C_k(j)d_k(j)=0\]
After adjusting $T_k$, it is easy to check that the norms of $T_k$
may be enlarged up to 4. Of course, we can pick up $T_k /4 $
instead and absorb the 4 into the constant $c_p$.

A nonnegative real matrix is said to be \emph{doubly stochastic} if each
of its row and column sum is 1. A sub-doubly stochastic matrix
means that each of its row and column sum is less than or equal to
1. Therefore we can change the assumption in Theorem~\ref{t ek<=dk T disc} 
to be that:
``for almost all $x_1$, ..., $x_{k-1}$, every row sum and
column sum of the matrix from $T_k$ is 0, 
and the matrix from $|T_k|$ is sub-doubly stochastic for each positive
integer $k$''

 One of the fundamental results in the theory of doubly
stochastic matrices was introduced by Birkhoff (see for example
\cite[p.~117]{M1}).

\begin{thm}
\label{t double stoch}
If $M$ is a doubly stochastic matrix, then
\[ M=\sum_{i=1}^S \theta_iP_i\]
where $P_i$ are permutation matrices, and the $\theta _i$ are
nonnegative numbers satisfying $\sum_{i=1}^S \theta _i =1$.
\end{thm}
\bigskip

\begin{lem}
\label{l 2nx2n}
If $M$ is a $n\times n$ sub-doubly stochastic matrix, 
then there exists a
$2n\times 2n$ doubly stochastic matrix such that its upper left
$n\times n$ sub-matrix is $M$.
\end{lem}

\begin{proof} Suppose that $R(i)$ is the $i^{th}$ row sum of $M$, $C(j)$ is the $j^{th}$ column sum and $S$
is the sum of all entries. Let
\[A=\left[\begin{array}{ccc} \frac{1-R(1)}{n}, & ... & ,\frac{1-R(1)}{n} \\
\vdots & & \vdots \\ \frac{1-R(n)}{n}, & ... &
,\frac{1-R(n)}{n}\end{array}\right]_{n\times n}\]
\[B=\left[\begin{array}{ccc} \frac{1-C(1)}{n}, & ... & ,\frac{1-C(n)}{n} \\
\vdots & & \vdots \\ \frac{1-C(1)}{n}, & ... &
,\frac{1-C(n)}{n}\end{array}\right]_{n\times n}\]
\[C= \mbox{Diag}\left[\begin{array}{ccc} \frac{S}{n}, & ... & ,\frac{S}{n}
\end{array}\right]_{n\times n}\]
Then define
\[M'=\left[\begin{array}{cc} M & A \\
B & C \end{array}\right]_{2n\times 2n}\] It is easy to check that
$M'$ is a doubly stochastic matrix.
\end{proof}

\begin{lem}
\label{l M+N}
If $M$ is a sub-doubly stochastic matrix, then there exists a sub-doubly
stochastic matrix $N$ such that $M+N$ is doubly stochastic.
\end{lem}

\begin{proof}
Let $M'$ be the $2n\times 2n$ doubly stochastic matrix such that
its upper left $n\times n$ sub-matrix is $M$. By Theorem~\ref{t double stoch},
\[ M'=\sum_{i=1}^S \theta_iP'_i\]
where $P'_i$ are $2n\times 2n$ permutation matrices and
$\sum_{i=1}^S \theta _i =1$. Suppose that $P_i$ is the upper left
$n\times n$ sub-permutation matrix of $P'_i$, then
\[ M=\sum_{i=1}^S \theta_iP_i\]
Let $Q_i$ be a $n\times n$ sub-permutation matrix such that
$P_i+Q_i$ is a permutation matrix, say $R_i$. Define
\[N=\sum_{i=1}^S \theta _iQ_i\]
thus \[M+N =\sum_{i=1}^S \theta _i R_i\] which is a doubly
stochastic matrix.
\end{proof}

\begin{thm}
\label{t sub stoch}
Let $M$ be an $n\times n$ matrix. If every row sum and column sum of
$M$ is 0 and $|M|$ is sub-doubly stochastic, then
\[M=\sum_{i=1}^S \theta _iP_i\] where $P_i$ are permutation matrices,
$\sum_{i=1}^S \theta _i=0$ and $\sum_{i=1}^S |\theta _i|=1$
\end{thm}

\begin{proof}
Let
\[A=\frac{|M|+M}{2}\]
\[B=\frac{|M|-M}{2}\]
so $A$ and $B$ are nonnegative, and $2A$ and $2B$ are sub-doubly
stochastic.
By Lemma~\ref{l M+N}, there exists a sub-doubly stochastic matrix $C$
such that $2(A+C)$ is a doubly stochastic. But $A$ and $B$ have
the same row sums and column sums, and hence $2(B+C)$ is also a doubly
stochastic. By applying Theorem~\ref{t double stoch}, we have
\[2(A+C)=\sum_{i=1}^m \lambda _iQ_i\]
\[2(B+C)=\sum_{i=1}^{m'} \lambda '_iQ'_i\]
where $Q_i$, $Q'_i$ are permutation matrices, and the $\lambda
_i$, $\lambda'_i$  are nonnegative numbers satisfying
$\sum_{i=1}^m \lambda_i =\sum_{i=1}^{m'} \lambda'_i =1$.
Then the result follows because
\[M=(A+C)-(B+C)=\sum_{i=1}^m \frac{\lambda _i}{2}Q_i -\sum_{i=1}^{m'} \frac{\lambda '_i}{2}Q'_i\]
\end{proof}

\begin{proof}[Proof of Theorem~\ref{t ek<=dk T disc}]
From Theorem~\ref{t sub stoch}, we know that for each $k\geq1$ and almost all
$x_1$, ..., $x_{k-1}$
\[T_k(x_1,...,x_{k-1})=\sum_{i_k=1}^{S_k} \theta _{k,i_k}(x_1,...,x_{k-1})\cdot P_{k,i_k}(x_1,...,x_{k-1})\]
where $P_{k,i_k}$ are permutation matrices, $\sum_{i=1}^{S_k}
\theta _{k,i_k}=0$, and $\sum_{i=1}^{S_k} |\theta _{k,i_k}|=1$. 
Let
\begin{equation}
\label{h=Pd}
h_{k,i_k}(x_1,...,x_{k-1},\cdot)=[P_{k,i_k}(x_1,...,x_{k-1})]d_k(x_1,...,x_{k-1},\cdot).
\end{equation}
Then
\begin{eqnarray*}
e_k& =& \left[\sum_{i_k=1}^{S_k} \theta _{k,i_k}P_{k,i_k}\right]d_k\\
&=&\sum_{i_k=1}^{S_k} |\theta _{k,i_k}|(\varepsilon _{k,i_k}h_{k,i_k})
\end{eqnarray*}
where $\varepsilon_{k,i_k}= \mbox{sgn}(\theta _{k,i_k})$.

Now we need to consider the probability space 
$\Omega_1 \times \Omega_2$, where 
$\Omega_1 = \Omega_2 = [0,1]^{\mathbf N}$.
We consider all of the previous random variables considered as
random variables on this new probabiltiy space, depending only upon
the first coordinate $\omega_1$.  We define a filtration 
$(\mathcal G_k)$ where $\mathcal G_k = \mathcal F_k \otimes \mathcal F_k$.

We define a sequence of random variables $(I_k)$ so that for
each $\omega_1 \in \Omega_1$, the random variable
$I_k(\omega_1,\cdot)$ takes the value $i$ with probability
$|\theta_{k,i}(\omega_1)|$.  Then we see that
\[ e_k = E(\epsilon_{k,I_k} h_{k,I_k} | 
   \mathcal L \otimes \{\emptyset,\Omega_2\}) .\]
Hence, since conditional expectation is a contraction on $L_p$
\[ 
\left\| \sum_{k=1}^n e_k \right\|_p
\le
\left\| \sum_{k=1}^n \epsilon_{k,I_k} h_{k,I_k} \right\|_p .
\]
Now we see that $(\epsilon_{k,I_k})$ is a predictable sequence
bounded by $1$.  Hence by Burkholder's inequality, we see
that
\[
\left\| \sum_{k=1}^n \epsilon_{k,I_k} h_{k,I_k} \right\|_p
\le
c_p
\left\| \sum_{k=1}^n h_{k,I_k} \right\|_p .
\]
Next, observing (\ref{h=Pd}), since $P_{k,i_k}$ are permutation
matrices, for each $k\geq1$, $i_k=1,2,...,S_k$, $h_{k,i_k}$ is
just an $x_k$-rearrangement of $d_k$. that is
\[h_{k,i_k}(x_1,...,x_{k-1},j)=d_k(x_1,...,x_{k-1},\pi_{k,i_k}(j))\]
for some permutation $\pi _{k,i_k}$. Thus for any sequence
$(i_k)$ we have that
$(h_{k,i_k})$ and
$(d_k)$ are tangent sequences.
But then we see that $(h_{k,I_k})$ and $(d_k)$ are tangent 
sequences.  Hence there
exists a positive constant $c_p$ such that
\[\left\|\sum_{k=1}^n h_{k,I_k}\right\|_p\leq
c_p \left\|\sum_{k=1}^n d_k\right\|_p . \] 
The result follows.
\end{proof}

\begin{thm}
\label{t ek<=dk sharp disc}
Theorem~\ref{t ek<=dk sharp} holds in the case that
$(d_k)$ and $(e_k)$ are
adapted to the filtration $(\mathcal F_k)$ described above.
\end{thm}

This will follow immediately from the following 
well-known result \cite[p.~124]{LT}.
\begin{thm}
\label{t fsharp<=gsharp}
$f=(f_1,f_2,...,f_N)$, $g=(g_1,g_2,...,g_N)$ are $N$-dimensional
real-valued vectors. $f^\#=(f_1^\#,f_2^\#,...,f_N^\#)$ is the
decreasing rearrangement of $|f|=(|f_1|,|f_2|,...,|f_N|)$. Then
\[ \sum_{k=1}^n g_k^\#\leq \sum_{k=1}^n f_k^\# \]
for all $n=1,2,...,N$ if and only if there exists a matrix
$T=[a_{ij}]_{N \times N}$ such that $Tf=g$, $\sum_{i=1}^N
|a_{ij}|\leq 1$ and $\sum_{j=1}^N |a_{ij}|\leq 1$.
\end{thm}









\section{The General Case}
The following theorem was proved by Crowe, Zweibel and Rosenbloom
\cite{CZR}.
\begin{thm}
\label{t CZR}
Suppose $f$, $g$ are random variables on $[0,1]$, then for $1\leq
p\leq \infty $,
\[\left\|f^\# -g^\# \right\|_p\leq \|f -g\|_p\]
\end{thm}

\begin{proof}[Proof of Theorem \ref{t ek<=dk sharp}]
We will prove this theorem by using the discrete case. For
each $k\geq 1$, we approximate
$d_k$ and $e_k$ by functions $d'_k(x_k)$ and
$e'_k(x_k)$ such that $(d'_k)$ and $(e'_k)$ are adapted
to $(\mathcal{L}_{k-1}\times\Sigma)$, keep the martingale
property, and 
\[\int_0^t (e'_k(x_1,...,x_{k-1},\cdot))^{\#}(s)ds \leq \int_0^t  (d'_k(x_1,...,x_{k-1},\cdot))^{\#}(s)ds\]
for all $t\in [0,1]$. Then we apply
Theorem~\ref{t ek<=dk sharp disc}.

In what follows, we will fix $x_1$,...,$x_{k-1}$, and regard the
functions below as functions as functions of only one variable $x_k$ on
$[0,1]$. Since we only consider
$L_p$-norm approximation, without loss of generality, we can
assume that $d_k$ and $e_k$ are simple functions. Suppose
\[d_k=\sum_{i=1}^S \alpha _{k,i}\chi _{A_{k,i}}\]
\[e_k=\sum_{i=1}^S \beta _{k,i}\chi _{B_{k,i}}\]
For $1\leq p<\infty$, each $A_{k,i}$, $B_{k,i}$ can be
approximated by $A'_{k,i}$, $B'_{k,i}$ which are the finite unions
of disjoint intervals $[a,b)$ with rational endpoints such that
for any $\gamma >0$,
\[\mu \left(\cup_{i=1}^S (A_{k,i}\bigtriangleup A'_{k,i})\right)< \gamma ^p \|d_k\|_1^p\]
\[\mu \left(\cup_{i=1}^S (B_{k,i}\bigtriangleup B'_{k,i})\right)< \gamma ^p \|e_k\|_1^p\]
Let $N$ be the least common denominator of these rational
endpoints. Hence $[\frac{i-1}{N},\frac{i}{N})$ is either contained
in some $A'_{k,j}$ or disjoint to all $A'_{k,j}$. Let
$\alpha'_{k,i}=\alpha_{k,j}$ if
$[\frac{i-1}{N},\frac{i}{N})\subset A'_{k,j}$ for some $j$,
$\alpha'_{k,i}=0$ otherwise. $\beta '_{k,i}$ are likewise.
\[\left(\int_{\cup _{i=1}^S (A_{k,i}\cap A'_{k,i})} +
 \int_{\cup _{i=1}^S (A_{k,i} \bigtriangleup A'_{k,i})}\right)
\left|\sum_{i=1}^N \alpha '_{k,i}\chi _i-d_k \right|^p \leq
\gamma^p \|d_k\|_\infty ^p \|d_k\|_1^p \] where $\chi _i=\chi
_{[\frac{i-1}{N},\frac{i}{N})}$.
\begin{equation}
\label{approx-1}
\left\|\sum_{i=1}^N \alpha' _{k,i}\chi _i-d_k \right\|_p \leq
\delta\|d_k\|_1
\end{equation}
where $\delta = \gamma \|d_k\|_\infty$. Similarly, for $\beta
'_{k,i}$, we have
\[\left\|\sum_{i=1}^N \beta' _{k,i}\chi _i-e_k \right\|_p \leq
\epsilon \|e_k\|_1  \] where $\epsilon = \gamma \|e_k\|_\infty$.
Since
\begin{equation}
\label{approx-2}
\left( \sum_{i=1}^N \alpha' _{k,i}\chi _i\right)^\# = \sum_{i=1}^N
\varepsilon_{\sigma (i)}\alpha' _{k,\sigma (i)}\chi _i
\end{equation}
for some permutation $\sigma $,  where $\varepsilon
_j=\mbox{sgn}(\alpha '_{k,j})$. By Theorem~\ref{t CZR},
\begin{equation}
\label{approx-3}
\left\| \sum_{i=1}^N \varepsilon_{\sigma (i)}\alpha' _{k,\sigma
(i)}\chi _i-d^\#_k \right\|_p <
 \delta\|d_k\|_1
\end{equation}
Define
\begin{equation}
\alpha''_{k,i}=N \cdot \int_{\frac{i-1}{N}}^{\frac{i}{N}} d_k^\#
\end{equation}
Note that
\begin{equation}
\int_0^t \sum_{i=1}^N \alpha''_{k,i}\chi _i=\int_0^t d_k^\#
\end{equation}
if $t=\frac{i}{N}$ for some $i=0,1,2,...,N$.

Since $d_k^\#$ is monotone decreasing, thus Remain integrable, if
$N$ is large enough ( if not, pick the multiple of $N$ ), we can
get
\begin{equation}
\label{approx-4}
\left\|\sum_{i=1}^N \alpha''_{k,i}\chi_i -d^\#_k \right\|_p <
\delta\|d_k\|_1
\end{equation}
From (\ref{approx-3}) and (\ref{approx-4}),
\begin{equation}
\left\|\sum_{i=1}^N \alpha''_{k,i}\chi_i- \sum_{i=1}^N
\varepsilon_{\sigma (i)}\alpha' _{k,\sigma (i)}\chi _i \right\|_p
< 2\delta\|d_k\|_1
\end{equation}
By doing the reverse process of taking decreasing rearrangement of
$|\sum_{i=1}^N \alpha' _{k,i}\chi _i|$,
\begin{equation}
\label{approx-5}
\left\|\sum_{i=1}^N \varepsilon_{\sigma^{-1}(i)}\alpha''_{k,\sigma
^{-1}(i)}\chi_i- \sum_{i=1}^N \alpha' _{k,i}\chi _i \right\|_p <
2\delta\|d_k\|_1
\end{equation}
From (\ref{approx-1}) and (\ref{approx-5}),
\begin{equation}
\label{approx-6}
\left\|\sum_{i=1}^N \varepsilon_{\sigma^{-1}(i)}\alpha''_{k,\sigma
^{-1}(i)}\chi_i-d_k \right\|_p < 3\delta\|d_k\|_1
\end{equation}
Repeat the same discussion from (\ref{approx-2}) to (\ref{approx-6}) 
for $\beta'_{k,i}$, we
can obtain a permutation $\tau $ and a sequence $(\beta
''_{k,\tau^{-1}(i)})$ such that
\[\left\|\sum_{i=1}^N \varepsilon_{\tau^{-1}(i)}\beta''_{k,\tau ^{-1}(i)}\chi_i-e_k \right\|_p <
3\epsilon \|e_k\|_1\] Here we realize that the integer $N$ should
be chosen large enough for approximating both $d^\#_k$ and
$e^\#_k$. For shorter notations, we set
\[\left(\varepsilon_{\sigma^{-1}(i)}\alpha''_{k,\sigma^{-1}(i)}\right)
_{\varepsilon_{\sigma^{-1}(i)}=1}^N=\left(\hat{\alpha}_{k,j}\right)_{j=1}^N\]
\[\left(\varepsilon_{\tau^{-1}(i)}\beta''_{k,\tau ^{-1}(i)}\right)
_{\varepsilon_{\tau^{-1}(i)}=1}^N=\left(\hat{\beta
}_{k,j}\right)_{j=1}^N\]
\[E\left[\sum_{j=1}^N \hat{\alpha }_{k,j}\chi_j\right]=\zeta\]
\[E\left[\sum_{j=1}^N \hat{\beta }_{k,j}\chi_j\right]=\eta \]
For $t=\frac{j}{N}$, $j=0,1,2,...,N$, it is clear that
\[\int_0^t \left( \sum_{j=1}^N \hat{\alpha }_{k,j}\chi _j \right)^\# =
\int_0^t \sum_{j=1}^N \alpha'' _{k,j}\chi _j=\int_0^t d_k^\#,\]
\[\int_0^t \left( \sum_{j=1}^N \hat{\beta }_{k,j}\chi _j \right)^\# =
\int_0^t \sum_{j=1}^N \beta'' _{k,j}\chi _j=\int_0^t e_k^\#,\] and
$|\zeta | \leq 3\delta\|d_k\|_1 $, $|\eta|\leq
3\epsilon\|e_k\|_1$.
\begin{eqnarray}
\label{int ineq seq}
& &\int_0^t \left(\sum_{j=1}^N \left(\hat{\beta}_{k,j}-\eta \right)\chi _j \right)^\# \\
& \leq &\int_0^t\left(\sum_{j=1}^N\left(|\hat{\beta}_{k,j}|+|\eta |\right)\chi _j \right)^\#\nonumber\\
& \leq &\int_0^t \left( \sum_{j=1}^N \hat{\beta}_{k,j}\chi _j
\right)^\# +
3\epsilon \|e_k\|_1\cdot t\nonumber\\
& \leq & \int_0^t d_k^\# + 3\epsilon \|d_k\|_1\cdot t\nonumber\\
& \leq & (1+3\epsilon)\int_0^t d_k^\# \nonumber
\end{eqnarray}
On the other hand,
\begin{eqnarray}
\label{int approx}
\int_0^t \left(\sum_{j=1}^N\left(\hat{\alpha}_{k,j}-\zeta
\right)\chi _j \right)^\#
&\geq & \int_0^t d_k^\# - 3\delta\|d_k\|_1\cdot t\\
&\geq & (1-3\delta)\int_0^t d_k^\#\nonumber
\end{eqnarray}
Finally, we are ready to define $d'_k$ and $e'_k$. Let
\[d'_k= (1+3\epsilon) \sum_{j=1}^N (\hat{\alpha }_{k,i}-\zeta )\chi_i \]
\[e'_k= (1-3\delta) \sum_{j=1}^N (\hat{\beta }_{k,i}-\eta )\chi _i\]
where $E[d'_k]=E[e'_k]=0 $ is obvious. 
Combining (\ref{int ineq seq}) and (\ref{int approx}), we
have
\[ \int_0^t (e'_k)^\# \leq \int_0^t (d'_k)^\# . \]
By Theorem~\ref{t ek<=dk sharp disc}, 
there exist a positive constant $c_p$ such
that
\[ \left\|\sum_{k=1}^n e'_k \right\|_p \leq c_p \left\| \sum_{k=1}^n d'_k \right\|_p\]
This is true for almost all $x_1$, ..., $x_{k-1}$, thus it is not
hard to check that
\[ \left\|\sum_{k=1}^n  d_k-\sum_{k=1}^n d'_k \right\|_p \leq \sum_{k=1}^n \left\|d_k -d'_k \right\|_p \leq
6\gamma\left(\sum_{k=1}^n \|d_k\|_\infty\|d_k\|_1\right)\]
\[ \left\|\sum_{k=1}^n e_k-\sum_{k=1}^n e'_k \right\|_p \leq \sum_{k=1}^n \|e_k -e'_k\|_p \leq
6\gamma \left(\sum_{k=1}^n \|e_k\|_\infty\|e_k\|_1\right)\]
Therefore
\[ \left\|\sum_{k=1}^n e_k \right\|_p \leq c_p \left\| \sum_{k=1}^n d_k \right\|_p +
12\gamma \left(\sum_{k=1}^n \|d_k\|_\infty \|d_k\|_1 +\sum_{k=1}^n
\|e_k\|_\infty\|e_k\|_1\right)\] for arbitrary $ \gamma$.
\end{proof}

\begin{proof}[Proof of Theorem~\ref{t ek<=dk T}]
If $f$ is a random variable on $(\Omega ,\mathcal{F},P)$, $1\leq
p<\infty$, $0\leq t\leq 1$, we define the $K$-functional by
\[K(t,f;L_p,L_\infty)=\inf_{f_0+f_1=f}\{\|f_0\|_p+t\|f_1\|_\infty \}.\]
J. Peetre \cite{P} has shown that
\[K(t,f;L_1,L_\infty) = \int_0^t f^{\#} (s)ds .\]
Hence it follows that if $T$ is an operator on both $L_1([0,1])$ 
and $L_\infty([0,1])$ with norm bounded by $1$, then for $t \ge 0$
\[ \int_0^t (Tf)^{\#}(s) ds \le \int_0^t f^{\#}(s) ds . \]
Thus the result follows from
Theorem~\ref{t ek<=dk sharp}.
\end{proof}

\begin{lem}
\label{l Mg<=Mf}
Let $f$ and $g$ are real-valued random variables on
$(\Omega ,\mathcal{F},P)$.  Then 
\begin{equation}
\label{Mg<=Mf}
E \left[ M \vee |g| \right]\leq E\left[M \vee |f| \right]
\end{equation}
for all nonnegative number $M$  if and
only if 
\[\int_0^t g^{\#}(s) ds\leq \int_0^t f^{\#}(s) ds \]
for all $t\in [0,1]$.
\end{lem}

\begin{proof}
Equation~(\ref{Mg<=Mf}) is equivalent to
$E \left[ M \vee g^\#  \right] \leq E \left[M \vee f^\# \right]$.
For the "if" part, let
\[\alpha =\sup \left\{t:f^\# (t) \geq M \right\}\]
\[\beta =\sup \left\{t:g^\# (t)\geq M \right\}\]
Note that
\[E \left[ M \vee f^\#\right] =\int_0^\alpha f^\#  + (1-\alpha)M\]
\[E \left[ M \vee g^\# \right]=\int_0^\beta  g^\#  + (1-\beta )M\]
If $\alpha \geq \beta$,
\begin{eqnarray*}
\int_0^\alpha f^\# + (1-\alpha )M
&=& \int_0^\beta f^\# +\int_\beta ^\alpha f^\# +(1-\alpha )M\\
&\geq & \int_0^\beta g^\# +\int_\beta ^\alpha M + (1-\alpha )M\\
&=& \int_0^\beta g^\# + (1-\beta )M
\end{eqnarray*}
If $\alpha <\beta$,
\begin{eqnarray*}
& &\left[ \int_0^\alpha f^\# + (1-\alpha )M
\right] -
\left[\int_0^\beta g^\# + (1-\beta )M\right]\\
&=&\int_0^\alpha \left[f^\# - g^\#
\right] -
\int_\alpha ^\beta g^\# + (\beta -\alpha )M\\
&=&\left\{\int_0^\beta \left[f^\# - g^\# \right] - 
\int_\alpha ^\beta \left[f^\#
- g^\# \right]  \right\}-
 \int_\alpha ^\beta  g^\# + (\beta -\alpha )M\\
&=&\int_0^\beta \left[f^\# - g^\# 
\right] + \left[(\beta -\alpha )M - \int_\alpha ^\beta  f^\# \right] \geq  0
\end{eqnarray*}
Conversely, for any $\alpha \in [0,1]$, let 
\[M=f^\#(\alpha)\]
\[\beta =\inf \left\{t:g^\# (t)\geq M \right\}\]
\begin{eqnarray*}
\int_0^\alpha f^\# + (1-\alpha )M
&=& E \left[ M \vee f^\# \right]\\
&\geq & E \left[M \vee g^\# \right]\\
&= & \int_0^\beta g^\# + (1-\beta )M\\
&\geq &\int_0^\alpha g^\# +(1-\alpha )M
\end{eqnarray*}
Thus
\[\int_0^\alpha g^{\#}(s) ds\leq \int_0^\alpha f^{\#}(s) ds \]
\end{proof}

Given a random variable $f$ and a sigma field $\mathcal{G}$, we
will say that $f$ is nowhere constant with respect to
$\mathcal{G}$ if $P(f=g)=0$ for every $\mathcal{G}$ measurable
function $g$. The following theorem \cite{M2} shows a concrete
representation of a sequence of random variables.

\begin{thm}
\label{t concrete}
Let $(f_n)$ be a sequence of random variables takeing values in a
separable sigma filed $(S,\mathcal{S})$. Then there exists a
sequence of measurable functions $(g_n:[0,1]^{n}\rightarrow S)$
that has the same law as $(f_n)$. If further we have that
$f_{n+1}$ is nowhere constant with respect to
$\sigma(f_1,...,f_n)$ for all $n \geq 0$, then we may suppose that
$\sigma(g_1,...,g_n)=\mathcal{L}_n$ for all $n \geq 0$.
\end{thm}

\begin{proof}[Proof of Theorem~\ref{t ek<=dk sk}]
First we claim that (\ref{skek<=skdk}) is equivalent to
\begin{equation}
\label{skekFk-1<=skdkFk-1}
E[(s_k \vee |e_k|)|\mathcal{F}_{k-1}]\leq E[(s_k \vee
|d_k|)|\mathcal{F}_{k-1}]
\end{equation}
For any $A_k \in \mathcal{F}_{k-1}$, $\lambda \geq 0$, $(\lambda
\chi_{A_k} \vee s_k)$ is predictable.
\[E[(\lambda \chi_{A_k} \vee s_k ) \vee |e_k|]\leq E[(\lambda \chi_{A_k} \vee s_k) \vee
|d_k|]\] When $\lambda$ intends to infinity,  \[E[( s_k \vee
|e_k|)\chi_{A_k^c}]\leq E[(s_k \vee |d_k|)\chi_{A_k^c}]\] which is
equivalent to (\ref{skekFk-1<=skdkFk-1}).

Let's consider the map $D_k=(d_k,e_k,f_k):\Omega \times
[0,1]^\mathbf{N} \rightarrow \mathbf{R}^3$ by
$(\omega,(x_k))\mapsto (d_k(\omega),e_k(\omega),x_k)$. It is clear
that $D_k$ is nowhere constant with respect to
$\sigma(D_1,D_2,...,D_{k-1})$. Apply the previous theorem to get
$\widetilde{D}_k=(\widetilde{d}_k,\widetilde{e}_k,\widetilde{f}_k):[0,1]^k
\rightarrow \mathbf{R}^3 $ such that $(\widetilde{D}_k)$ has the
same law as $(D_k)$ and
$\sigma(\widetilde{D}_1,\widetilde{D}_2,...,\widetilde{D}_k)=\mathcal{L}_k$.
Let $\phi : [0,1]^{k-1} \rightarrow [0,\infty)$ be any
bounded nonnegative measurable function. There exists a bounded
Borel measurable function $\theta : \mathbf{R}^{3(k-1)}
\rightarrow \mathbf{R}$ such that
$\phi=\theta(\widetilde{D}_1,\widetilde{D}_2,...,\widetilde{D}_{k-1})$
almost everywhere in $[0,1]^{k-1}$. For particular $v_k=M$,
\begin{eqnarray*}
\int_{[0,1]^k} (M \vee |\widetilde{e}_k|) \cdot\phi
&=&\int_{[0,1]^k} (M\vee |\widetilde{e}_k|) \cdot \theta
(\widetilde{D}_1,...,\widetilde{D}_{k-1})\\
&=& \int_\Omega (M \vee |e_k|) \cdot \theta(D_1,...,D_{k-1})\\
&=& \int_\Omega E[(M \vee |e_k|)|\mathcal{F}_{k-1}] \cdot
\theta(D_1,...,D_{k-1})\\
&\leq & \int_\Omega E[(M \vee |d_k|)|\mathcal{F}_{k-1}] \cdot
\theta(D_1,...,D_{k-1})\\
&=& \int_\Omega (M \vee |d_k|) \cdot
\theta(D_1,...,D_{k-1})\\
&=& \int_{[0,1]^k} (M \vee |\widetilde{d}_k|) \cdot
\theta(\widetilde{D}_1,...,\widetilde{D}_{k-1})\\
&=&\int_{[0,1]^k} (M \vee |\widetilde{d}_k|) \cdot\phi
\end{eqnarray*}
Since this is true for all such $\phi\geq 0$, thus $E[(M \vee
|\widetilde{e}_k|)|\mathcal{L}_{k-1}] \leq E[(M \vee
|\widetilde{d}_k|)|\mathcal{L}_{k-1}]$. Also if $\phi :
[0,1]^{k-1} \rightarrow \mathbf{R}$ is any bounded measurable
function, there exists a bounded Borel measurable function $\theta
: \mathbf{R}^{3(k-1)} \rightarrow \mathbf{R}$ such that
$\phi=\theta(\widetilde{D}_1,\widetilde{D}_2,...,\widetilde{D}_{k-1})$
almost everywhere in $[0,1]^{k-1}$. Similarly,

\[\int_{[0,1]^k} \widetilde{d}_k \cdot\phi= \int_\Omega E[d_k|\mathcal{F}_{k-1}] \cdot
\theta(D_1,...,D_{k-1})=0.\] Thus
$E[\widetilde{d}_k|\mathcal{L}_{k-1}]=0$. i.e.
$(\widetilde{d}_k)$, $(\widetilde{e}_k)$ are martingale difference
sequences with respect to $(\mathcal{L}_{k})$ and the result
follows from Lemma~\ref{l Mg<=Mf} and Theorem~\ref{t ek<=dk sharp}.
\end{proof}

{\bf Acknowledgments.} We would like to mention the help of Jim
Reeds and David Boyd in obtaining the argument about doubly
stochastic matrices and other useful remarks.

\bibliographystyle{amsplain}
\begin{thebibliography}{10}
\bibitem {B1} D. L. Burkholder, \textit{Distribution function inequalities for martingales.} Ann.
Probability \textbf{1} (1973), 19--42.

\bibitem {B2} D. L. Burkholder, \textit{A geomtrical characterization of banach spaces in which
marti difference sequences are unconditional.} Ann. Probability
\textbf{9} (1981), 997--1011.

\bibitem {CZR} L. A. Crowe, J. A. Zweibel and P. C. Rosenbloom, \textit{Rearrangements of functions},
Journal of functional analysis \textbf{66} (1986), 43291--438.

\bibitem {H1} P. Hitczenko, \textit{Comparison of moments for tangent sequences of random variables.}
Probab. Theory Related Fields \textbf{78} (1988), 223--230.

\bibitem {H2} P. Hitczenko, \textit{On a domination of sums of random variables by sums of conditionally
independent ones.} Ann. Probability \textbf{22} (1994), 453--468.

\bibitem {H3} P. Hitczenko and S. J. Montgomery-Smith, \textit{Tangent sequences in Orlicz and
rearrangement invariant spaces.} Math. Proc. Camb. Phil. Soc.
\textbf{119} (1996), 91--101.

\bibitem {K} P. Kree, \textit{Interpolation d'espaces qui ne sont mi normes, ni complets.} Applications,
Seminaire Lions-Schwartz, semestre 1964-1965, Secretariat Mathematatique, 11 rue Pierre Curie, Paris 5e.

\bibitem {KW1} S. Kwapie\'n and W. A. Woyczynsi, \textit{Semimartingale integrals
via decoupling inequalities and tangent processes.} Case Western Reserve University, preprint, (1986).

\bibitem {KW2} S. Kwapie\'n and W. A. Woyczynsi, \textit{Random series and stochastic integrals. Single and
 multiple.} Birkhauser, Boston. (1996).

\bibitem {LT} J. Lindenstrauss ans L. Tzafriri, 
\textit{Classical Banach spaces II.} Springer-Verlag. (1979).

\bibitem {M1} H. Minc, \textit{Nonnegative matrices.} Wiley Interscience. (1988).

\bibitem {M2} S. J. Montgomery-Smith, \textit{Concrete representation of
martingales.} Electronic J. Probab. \textbf{3}, (1998), paper 15

\bibitem {P} J. Peetre, \textit{Espaces d'interpolation, generalisations,
applications.} Rend. Sem. Mat. Fis. Milano \textbf{34} (1964),
83-92.

\bibitem {Z} J. Zinn, \textit{Comparison of martingale difference sequences.}
Probability in Banach spaces, V (Medford, Mass., 1984), 453--457, Lecture Notes
in Math., 1153, Springer, Berlin, 1985.

\end{thebibliography}

\end{document}


If we define that
\[H_{(p)}f(t)= \left(\int_0^{t^p} \left(f^{\#} (s)\right)^pds\right)^{1/p},\]
the result above can be generalized by Kree \cite{K} to yield
\begin{thm}
\label{t kree}
$1\leq p<\infty $, $K(t,f;L_p,L_\infty )$ is equivalent to
$H_{(p)}f(t)$. i.e. there exists two positive constants $k_p$,
$k'_p$ such that
\[k_p H_{(p)}f(t)\leq K(t,f;L_p,L_\infty )\leq k'_p H_{(p)}f(t)\]
\end{thm}

One of the directions of the result in Theorem~\ref{t fsharp<=gsharp} 
can be extended
immediately.
\begin{cor}
\label{c kree for p}
$1\leq p < \infty $. $f$, $g$ are real-valued random variables on
$(\Omega ,\mathcal{F},P)$. Then $H_{(p)}g(t)\leq H_{(p)}f(t)$ for
all $t\in [0,1]$ if there exists a linear operator $T$ in
$L_p(\Omega )$ and $L_\infty (\Omega )$ with norm bounded by
$k_p/k'_p$ such that $Tf=g$.
\end{cor}
\begin{proof}
Whenever $f_0 + f_1 =f$, let $g_0=Tf_0$, $g_1=Tf_1$. By Theorem~\ref{t kree},
\begin{eqnarray*}
H_{(p)}f(t)
& \geq & \frac{1}{k'_p} \inf_{f_0+f_1=f}\{\|f_0\|_p + t\|f_1\|_\infty \}\\
& \geq & \frac{1}{k'_p} \inf_{f_0+f_1=f}\{\frac{k'_p}{k_p}
|g_0\|_p + t \frac{k'_p}{k_p} \|g_1\|_\infty \}\\
& \geq & \frac{1}{k_p} \inf_{g_0+g_1=g }\{\|g_0\|_p + t\|g_1\|_\infty \}\\
& \geq & H_{(p)}g(t)
\end{eqnarray*}
\end{proof}
Note that at the special case $p=1$, $k_1/k'_1=1$.
