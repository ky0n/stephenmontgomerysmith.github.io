\input amstex
\documentstyle{amsppt}
\overfullrule=0pt
\document


\topmatter
\title Bounds on the Tail Probability of 
U-Statistics and Quadratic Forms \endtitle
\author Victor H. de la Pe\~na and S.~J.~Montgomery-Smith \endauthor
\address  Department of Statistics, Columbia University, 
New York, New York 10027
\endaddress
\email vp\@wald.stat.columbia.edu
\endemail
\address Department of Mathematics, University of Missouri, Columbia, Missouri
65211
\endaddress
\email stephen\@mont.cs.missouri.edu
\endemail
\thanks Both authors were supported by NSF grants. \endthanks
\subjclass AMS 1991 subject classifications:  Primary 60E15. 
\indent Secondary 60D05,
60E05 \endsubjclass
\keywords
U-statistics, Quadratic Forms, Decoupling \endkeywords

\endtopmatter

It is very common for expressions of the form:
$$ \sum_{1\le i_1 \ne \dots \ne i_k \le n}
   f_{i_1 \dots i_k}(X_{i_1},\dots,X_{i_k}) $$
to appear in Probability theory.  Here $\{X_i\}$\ is a 
sequence of independent
random variables taking values in a measurable space $(S, {\Cal S})$,
and $\{f_{i_1\dots i_k}\}$ is a sequence of measurable functions
from  $S^k$ into a 
Banach Space $(B,
\| \cdot \|)$.  Special cases of this type of random variable appear, 
for example, in Statistics in
the form of U-statistics and quadratic forms.
Throughout we will refer to them as generalized U-statistics.

There is great interest in decoupling such quantities, that is, by replacing
the above quantity by the expression
$$ \sum_{1\le i_1 \ne \dots \ne i_k \le n}
   f_{i_1 \dots i_k}(X_{i_1}^{(1)},\dots,X_{i_k}^{(k)}),$$
where $\{X_i^{(1)}\}$, $\{X_i^{(2)}\},\dots$,
$\{X_i^{(k)}\}$ are $k$\ independent
copies of $\{X_i\}$.  

Decoupling inequalities allow one to compare expressions of the first kind
with expressions of the second kind.  
Such results permit the almost direct transfer of results for sums of
independent random variables to the case of generalized U-statistics. 
The reason being that 
conditionally on $\{X_i^{(2)}\},...,\{X_i^{(k)}\}$,
the second sum above is a sum 
independent random variables. 
It is important to remark that such results have 
led to the development of several optimal 
results in the functional theory of U-Statistics (cf. [1]
and [7]) and various other areas
including the study of the invertibility of large matrices
(cf. [2]),  
stochastic integration (cf. [10]),
the study of integral operators on Lebesgue-Bochner spaces 
(cf. a result of T. R. McConnell and D. Burkholder found in
[3]). Aside from those directly cited in this paper, other
important contributors to the area of
decoupling inequalities include
A. de Acosta,  
P. Hitczenko, 
J. Jacod, 
A. Jakubowski, 
O. Kallenberg, 
M. Klass,
W. Krakowiak,
G. Pisier,
J. Rosinski, and
J. Szulga.
Due to space restrictions, we refer the reader to 
[10] for a more complete account.

In this paper, we announce a result, which allows one to compare
the tail probabilities of the above quantities.
In particular, this inequality 
represents the definitive generalization 
of the decoupling inequalities for multilinear forms of McConnell
and Taqqu [11] and the more general decoupling inequalities for
expectations of convex functions of U-statistics introduced in 
[4]. 

\proclaim{Theorem 1} There is a constant $C_k>0$, depending only on $k$, 
such that
for all $n\ge k$, 
$$ P(\|\sum_{1\le i_1 \ne \dots \ne i_k \le n}
   f_{i_1 \dots i_k}(X_{i_1},\dots,X_{i_k}) \| \ge t) \qquad \qquad 
\qquad\qquad$$
$$\qquad \qquad\le
   C_k P(C_k\|\sum_{1\le i_1 \ne \dots \ne i_k \le n}
   f_{i_1 \dots i_k}(X_{i_1}^{(1)},\dots,X_{i_k}^{(k)}) \| \ge t), \hbox
{ for all $t>0$.}\leqno(1)$$
Moreover, the reverse inequality holds if 
the functions satisfy the condition
$$ f_{i_1 \dots i_k}(X_{i_1},\dots,X_{i_k}) =
   f_{i_{\pi(1)} \dots i_{\pi(k)}}(X_{i_{\pi(1)}},\dots,X_{i_{\pi(k)}}) $$
for all permutations $\pi$\ of $\{1,\dots,k\}$. 
That is, 
$$ P(\|\sum_{1\le i_1 \ne \dots \ne i_k \le n}
   f_{i_1 \dots i_k}(X_{i_1},\dots,X_{i_k}) \| \ge t) 
\qquad\qquad
\qquad\qquad$$
$$ \qquad\qquad\ge
  {1\over C_k} P({1\over C_k}\|\sum_{1\le i_1 \ne \dots \ne i_k \le n}
   f_{i_1 \dots i_k}(X_{i_1}^{(1)},\dots,X_{i_k}^{(k)}) \| \ge t),  \hbox
{ for all $t>0$.}\leqno(2)$$
Note that
the expression $  i_1 \ne \dots \ne i_k $\ means that $i_r \ne i_s$\ for all
$1\le r \ne  s \le k$.
\endproclaim

An example to illustrate this result can be found in the study of random
graphs (see also [5]).  
Given a sequence of independent random points $\{X_i\}$\
in $R^N$, we might consider a measure of clustering
$$ D_1 = \sum_{1\le i \ne j \le n} d(X_i,X_j) ,$$
where $d(x,y)$\ denotes the distance between $x$\ and $y$.
The above result allows us to compare $D_1$, which measures the distance
`within' the graph formed by the random cluster of points 
$\{X_i\}$, to a quantity $D_2$, which is a  measure of the distance `between'
the two independent clusters $\{X_i\}$ and $\{\tilde X_i \}$,
$$D_2 = \sum_{1\le i \ne j \le n} d(\tilde X_i, X_j), $$
\noindent where $\{\tilde X_i\}$ is an independent copy of $\{X_i\}$. 
Then we have for all $t>0$ that,
$$ C_2^{-1} P(|D_1| \ge C_2 t) \le P(|D_2| \ge t)
   \le C_2 P(|D_1| \ge C_2^{-1} t) .$$

Other examples where U-statistics are used in graph theory may be found
in [8]. 

\bigskip

We will prove the Theorem in the special case that $k=2$. 
For ease of notation, let us suppose that $ \tilde X_i = X_i^{(1)}$,
and denote $ X_i = X_i^{(2)}$.
The proof
of the more general result will appear elsewhere.
We will use a sequence of lemmas. 
Following [6], 
our point of departure is equation (4) below which
provides a partial decoupling result and focuses attention on a
polarized version of the U-statistic kernel as the key element in 
the development of a solution of the problem at hand. 
Let 
$$T_n =
\sum_{1\le i\ne j\le n} \{f_{ij}(X_i, X_j)+
f_{ij}(X_i, \tilde X_j)+   
f_{ij}(\tilde X_i, X_j)
+ f_{ij}(\tilde X_i,\tilde X_j)\}, \leqno (3)$$ then by using the
triangle inequality one obtains that,
$$\eqalign {\noindent P(\|\sum_{1\le i\ne j\le n} f_{ij}(X_i, X_j) 
+ f_{ij}(\tilde X_i,\tilde X_j)\| \ge  t )\le & \cr  
P(\| T_n 
\| \ge { t \over 3} ) 
+
2P(\|\sum_{1\le i\ne j\le n} f_{ij}(X_i, \tilde X_j)\| \ge { t \over 3} ).
} \leqno (4) $$
This observation reduces the proof (1) to the problem  of 
obtaining the bounds 
$$P(\|\sum_{1\le i\ne j\le n} f_{ij}(X_i, X_j)\| \ge t) \le
cP(c\|\sum_{1\le i\ne j\le n} f_{ij}(X_i, X_j) +
f_{ij}(\tilde X_i,\tilde X_j)\| \ge  t )
, \hbox{ and} \leqno (5)$$
$$  P(\|T_n
\| \ge { t } ) 
 \le  
cP(c\|\sum_{1\le i\ne j\le n} f_{ij}(\tilde X_i,X_j)\|  \ge  t).\leqno (6)$$ 
We obtain (5) by means of Lemma 1 
(possibly of independent interest).
The proof of (6) is somewhat involved. In obtaining it, we used 
(conditionally) an extension of the Paley-Zygmund inequality  found in
[10] in combination with a symmetrization identity 
similar to the one introduced in [12].

\proclaim{Lemma 1} Let X, Y be two i.i.d. random variables.
Then
$$
P(\|X\|\ge t)  \le 
3P(\|X+Y\|\ge {2t\over 3}). \leqno (7) $$
\endproclaim

\demo{Proof}
Let $X$, $Y$\ and $Z$ be i.i.d. random
variables. Then
$$\eqalign{P(\|X \|& \ge t)  \cr
&=
P(\|(X+Y) + (X+Z) - (Y+Z) \ge 2t) \cr
& \le
P(\|X+Y\| \ge {2t\over 3})+P(\|X+Z\| \ge {2t\over 3})+
P(\|Y+Z\| \ge {2t\over 3}) \cr
& =
3P(\|X+Y\| \ge {2t\over 3}). \cr }$$
\enddemo

It is to be remarked that the
very desirable `Universal Symmetrization Lemma', 
$P(\|X\|\ge t)  \le 
c P(c \|X-Y\|\ge t)$\ is not true.  This makes the above result all the more
surprising.

The following is an observation found in Section 6.2 of
[10] will be used in combination with 
Lemma 2 in proving Theorem 1.

\proclaim{Proposition 1}
Let $Y$ be any mean zero random variable with values in a
Banach Space $(B, \|\cdot \|)$. Then, for all $a\in B$,
$P(\|a+Y\| \ge \|a\|) \ge {\kappa \over 4}$, where,
$\kappa = \inf_{x'\in B'}
 {(E|x'(Y)|)^2 \over E(x'(Y))^2}$.
\endproclaim

As a consequence of the above we obtain the following lemma.

\proclaim{Lemma 2}
 Let $x, a_i, b_{ij}$ all belong to a Banach space $ (B, \|\cdot\|)$,
with $b_{ii} =0$. Let $\{\epsilon_i\}$ be a sequence of independent and 
symmetric
Bernoulli
random variables, that is, $P(\epsilon_i = 1) = P(\epsilon_i =-1) 
={1\over 2}$.
Then,
for a universal constant $c > 0 $,
$$P(\|x+\sum_{i=1}^n a_i \epsilon_i + 
\sum_{1\le i\ne j \le n} b_{ij} \epsilon_i \epsilon_j \| \ge \|x\|)
\ge c^{-1}.$$
\endproclaim

\demo{Proof} Suppose that $a_i, b_{ij}$ are in $R$, then it follows
easily from (1.4) of [9] (see also sections 6.2 and
6.5 of [10]) that
$$ (E|\sum_{i=1}^n a_i\epsilon_i + 
   \sum_{1\le i \ne j \le n} b_{ij} \epsilon_i \epsilon_j |^4)^{1\over 4} 
   \le c (
   E|\sum_{i=1}^n a_i\epsilon_i + 
   \sum_{1\le i \ne j \le n} b_{ij} \epsilon_i \epsilon_j |^2)^{1/2} ,$$ 
for some constant $c>0$.
Next, observe that $\|\xi \|_4 \le c \|\xi \|_2 $ 
implies that 
$\|\xi \|_2 \le c^2 \|\xi \|_1 $ (since $E(\xi )^2 \le 
(E|\xi |)^{2/3}\cdot (E(\xi )^4)^{1/3}$) . 
The result then follows by
Proposition~1.
\enddemo

\demo{Proof 
of Theorem 1}
We first transform the problem of proving (6) into a problem dealing  
(conditionally) with a non-homogeneous binomial 
in Bernoulli random variables.
Let 
$\{\epsilon_i \}$ be a sequence of independent
and symmetric
Bernoulli random variables independent of $\{X_i\}$, $\{\tilde X_i\}$.  
Let $(Z_i, \tilde Z_i) = (X_i, \tilde X_i) $ if $\epsilon_i =  1$ 
and $(Z_i, \tilde Z_i) = (\tilde X_i,  X_i) $ if $\epsilon_i = -  1$.
Then, 
$$\eqalign{ 4f_{ij}(\tilde Z_i, Z_j) & = \{(1-\epsilon_i)(1+\epsilon_j)
f_{ij}(X_i,X_j) 
+ (1+\epsilon_i)(1+\epsilon_j)
f_{ij}(\tilde X_i,X_j) \cr
&
+ (1-\epsilon_i)(1-\epsilon_j)
f_{ij}( X_i,\tilde X_j) 
+ (1+\epsilon_i)(1-\epsilon_j)
f_{ij}( \tilde X_i,\tilde X_j) \} .\cr }\leqno (8)$$
Setting ${\Cal G} = \sigma (X_i, \tilde X_i ; i=1,...,n)$
we get 
$$4E(f_{ij}(\tilde Z_i, Z_j)| {
\Cal G})  =\{ 
f_{ij}(X_i,X_j) + 
f_{ij}(\tilde X_i,X_j) +
f_{ij}( X_i,\tilde X_j) +
f_{ij}( \tilde X_i,\tilde X_j)\}.\leqno (9) $$
>From Lemma 2, (3), (8) and (9), and letting $x=T_n$,
it follows that for some $c>0$,
$$P(4\|\sum_{1\le i\ne j\le n} f_{ij}(\tilde Z_i, Z_j) \| \ge 
\|T_n\|
\big| {\Cal G} ) \ge c^{-1} .
$$
Integrating over the set
$\{\| T_n
\|  \ge t\}$
we get
$$ {1 \over c}P(\|
T_n\|  \ge t) \le
P(4\|\sum_{1\le i\ne j\le n} f_{ij}(\tilde Z_i,  Z_j) \|  \ge t) =  
P(4\|\sum_{1\le i\ne j\le n} f_{ij}(\tilde X_i,  X_j) \| \ge  t),  $$ 
\noindent since the sequence $\{(X_i,\tilde X_i), i=1,...,n\}$
has the same distribution as
$\{(Z_i,\tilde Z_i), i=1,...,n\}.$ The proof is completed by using this
inequality along with (4) and (5).

The proof of (2) is similar and  uses an analogue of (8) concerning
$4f(Z_i,Z_j)$. 
In obtaining this bound, one does not need to use Lemma 1. Instead
one 
uses the symmetry condition on the functions $f_{ij}$, introduced after (1) 
and equation (3) to get,
$$P(\|\sum_{1\le i \ne j \le n} f_{ij}(\tilde X_i,X_j) \| \ge t) =
P(\|\sum_{1\le i \ne j \le n} f_{ij}(X_i,\tilde X_j) +
f_{ij}(\tilde X_i,X_j) \| \ge 2t)  $$
$$ \le P(\| T_n
  \| \ge {2\over 3}t)  
+ 2P(\|\sum_{1\le i \ne j \le n} f_{ij}(X_i,X_j) \| \ge {2\over 3}t). $$

\enddemo

\Refs

\ref \no 1 \by M.~Arcones and E.~Gin\'e \paper Limit theorems for U-processes
\jour Ann. Probab. \vol 21 (3) \pages 1495--1592 \yr 1993 \endref

\ref \no 2 \by J.~Bourgain and L.~Tzafriri \paper 
Invertibility of ``large'' submatrices with
applications to the geometry of Banach spaces and
harmonic analysis
\jour Israel J. Math. \vol 57 (3) \pages 137--224 \yr 1987 \endref

\ref \no 3 \by D.~Burkholder \paper A geometric condition that implies the 
existence
of certain singular integrals of Banach-space-valued functions
\inbook Conference on Harmonic Analysis in Honor of A. Zygmund
\eds W.~Beckner, A.P.~Calderon, R.~Fefferman, P.~Jones
\publ Wadsworth \pages 270--286 \yr 1983 \endref

\ref \no 4 \by V.H.~de~la~Pe\~na \paper Decoupling and Khintchine's 
inequalities
for U-statistics \jour Ann. Probab. \vol 20 (4) \pages 1877--1892 \yr 1992
\endref

\ref \no 5 \by V.H.~de~la~Pe\~na \paper Nuevas desigualdades para 
U-estad\'isticas y gr\'aficas aleatorias \inbook Proceedings of the fourth
Latin American Congress of Probability and Math. Stat. (CLAPEM),
M\'exico City, September 1990 \yr 1992 \endref

\ref \no 6 \by V.H.~de~la~Pe\~na, S.J.~Montgomery-Smith and
J.~Szulga \paper Contraction and decoupling inequalities for  
multilinear forms and U-statistics \jour Preprint \yr 1992 \endref

\ref \no 7 \by E.~Gin\'e and J.~Zinn
\paper A remark on convergence in
distribution of U-statistics
\jour Ann. Probab. (to appear) \yr 1992 \endref

\ref \no 8 \by S.~Janson and K.~Nowicki \paper The asymptotic distributions of 
generalized U-statistics with applications to random graphs
\jour Probab. Theory Relat. Fields \vol 90 \pages 341--375 \yr 1991 \endref

\ref \no 9 \by S.~Kwapien and J.~Szulga \paper Hypercontraction methods in
moment inequalities for series of independent random variables in
normed spaces \jour Ann. Probab. \vol 19 (1)  \pages 369--379 \yr 1991 \endref

\ref \no 10 \by S.~Kwapien and W.~Woyczynski 
\book Random series and stochastic
integrals: simple and multiple \publ Birkhauser, NY \yr 1992 \endref

\ref \no 11 \by T.R.~McConnell and M.S.~Taqqu 
\paper Decoupling inequalities
for multilinear forms in independent symmetric random variables
\jour Ann. Probab. \vol 14 (3) \pages 943--954 \yr 1986 \endref

\ref \no 12 \by D.~Nolan and D.~Pollard D 
\paper U-processes: rates of convergence
\jour Annals of Stat. \vol 15 (2) \pages 780--799 \yr 1987 \endref

\endRefs

\enddocument



\item{1.} Arcones, M. and Gin\'e, E. (1993). Limit theorems for U-processes.
{\it Ann. Probab.}, Vol. {\bf 21} (3), 1494-1542.

\item{2.} Bourgain, J. and Tzafriri, L. (1987).
Invertibility of ``large'' submatrices with
applications to the geometry of Banach spaces and
harmonic analysis. {\it Israel J. Math.} {\bf 57} (3)
137-224. 

\item{3.} Burkholder, D. (1983). A geometric condition that implies the existenceof certain singular integrals of Banach-space-valued functions,
{\it Conference on Harmonic Analysis in Honor of A. Zygmund}, W. Beckner, A.P. 
Calderon, R. Fefferman, P. Jones, Eds., Wadsworth, pp. 270-286. 

\item {4.} de la Pe\~na, V. H. (1992a). Decoupling and Khintchine's 
inequalities
for U-statistics. {\it Ann. Probab.}, Vol. {\bf 20} (4) 1877-1892.

\item {5.} de la Pe\~na, V. H. (1992b). Nuevas desigualdades para 
U-estad\'isticas y gr\'aficas aleatorias. {\it Proceedings of the fourth
Latin American Congress of Probability and Math. Stat. (CLAPEM)},
M\'exico City, September 1990.

\item {6.} de la Pe\~na V. H., Montgomery-Smith, S. J.\ and
Szulga, J.\ (1992).  Contraction and decoupling inequalities for  
multilinear forms and U-statistics. {\it Preprint.}

\item{7.} Gin\'e, E. and Zinn, J. (1992). A remark on convergence in
distribution of U-statistics. To appear in {\it Ann. Probab.}

\item{8.} Janson, S. and Nowicki, K. (1991). The asymptotic distributions of 
generalized U-statistics with applications to random graphs.
{\it Probab. Theory Relat. Fields} {\bf 90}, 341-375.

\item{9.} Kwapien, S, and Szulga, J. Hypercontraction methods in
moment inequalities for series of independent random variables in
normed spaces. {\it Ann. Probab.} {\bf 19} (1), 369-379.

\item {10.} Kwapien, S, and Woyczynski, W. (1992). 
Random series and stochastic
integrals: simple and multiple. Birkhauser, NY.

\item {11.} McConnell, T.R. and Taqqu, M.S. (1986).  Decoupling inequalities
for multilinear forms in independent symmetric random variables.
{\it Ann. Probab.} {\bf 14} (3) 943-954.

\item{12.} Nolan D. and Pollard D. (1987). U-processes: rates of convergence.
{\it Annals of Stat.} {\bf 15} (2) 780-799.



