\documentclass[12pt]{amsart}
\usepackage{amsmath,amssymb}
%\usepackage{fullpage}

%\usepackage{setspace}

\newtheorem{thm}{Theorem}
\newtheorem{prop}[thm]{Proposition}
\newtheorem{lemma}[thm]{Lemma}
\newtheorem{cor}[thm]{Corollary}

\newcommand{\N}{{\mathbb N}}
\newcommand{\E}{{\mathbb E}}
\newcommand{\R}{{\mathbb R}}

\newcommand{\modo}[1]{{\left|#1\right|}}
\newcommand{\normo}[1]{{\left\|#1\right\|}}
\newcommand{\smodo}[1]{{\mathopen|#1\mathclose|}}
\newcommand{\snormo}[1]{{\mathopen\|#1\mathclose\|}}

\newcommand{\measure}{\text{\rm measure}}

\begin{document}
\title[R.i. norms of symmetric sequence norms]
{Rearrangement Invariant Norms of Symmetric Sequence Norms
of Independent Sequences of Random Variables
}

\author{Stephen Montgomery-Smith}
\makeatletter
\address{Department of Mathematics\\
University of Missouri\\
Columbia, MO 65211}
\email{stephen@math.missouri.edu}
\urladdr{http://www.math.missouri.edu/\~{}stephen}
\thanks{The author was 
partially supported
by NSF grant DMS 9870026, and a grant from the Research Office of the
University of Missouri.}
\keywords{Independent random variables, symmetric sequence space,
rearrangement invariant space}
\subjclass{60G50, 46B45, 46E30}

\begin{abstract}
\noindent
Let $X_1$, $X_2,\dots,$\ $X_n$ be a sequence of independent random
variables, let $M$ be a rearrangement invariant space on the underlying
probability space, and let $N$ be a symmetric sequence space.  This paper
gives an approximate formula for the quantity
$\snormo{\snormo{(X_i)}_N}_M$ whenever $L_q$ embeds into $M$ for some
$1 \le q < \infty$.
This extends recent work of Gordon, Litvak, Sch\"utt and Werner who obtained
similar results for Orlicz spaces.
\end{abstract}

\maketitle

\section{Introduction}

Let $M$ be a rearrangement invariant space on $[0,1]$, or equivalently
on any probability space, and let $N$ be a symmetric sequence space.
Let $X_1$, $X_2, \dots,$\ $X_n$ be positive independent random variables.
Define a function $Y:[0,n] \to [0,\infty]$ to be a non-increasing
function such that
\[ \measure\{Y > t \} = \sum_{i=1}^n \Pr(X_i > t) . \]
Notice that $Y$ has the same law as the function 
$t \mapsto X_{[t]+1}(t-[t])$ where $[t]$ denotes the integer part of $t$.
The purpose of this paper is to investigate conditions under which
the following approximation holds:
\begin{equation}
\label{e main}
   \snormo{\snormo{(X_i)}_N}_M \approx
   \snormo{Y|_{[0,1]}}_M + \snormo{(Y(i))}_N .
\end{equation}
Here $A \approx B$ means that the ratio of $A/B$ is bounded below
and above by constants.  In the case of equation~(\ref{e main}), the
constants of approximation may be allowed to depend upon $M$.

There is much support to make such a conjecture.  First, Rosenthal's
inequality \cite{rosenthal} can be interpreted (see for example 
\cite{carothers-dilworth})
as the truth of this in the case that $N = \ell_1$ or $N = \ell_2$,
and $M = L_p$ for $1 \le p < \infty$.  This was extended by
Carothers and Dilworth \cite{carothers-dilworth} to the case when
$M$ is a Lorentz space $L_{p,q}$, $1 \le p < \infty$, $1 \le q \le \infty$,
and then by Johnson and Schechtman \cite{johnson-schechtman} to the case
when $M$ is any rearrangement invariant space, even including the case
when $M$ is a quasi-Banach space, as long as $M$ satisfies certain restrictions
on the Boyd indices that disallow the case $M = L_\infty$.
It is not hard to extend this last result to also allow $N = \ell_p$ for any
$1 \le p < \infty$.

The next step was taken in a paper by 
Gordon, Litvak, Sch\"utt and Werner \cite{gordon et al}.
They found a formula for $\snormo{\snormo{(X_i)}_N}_M$ in the case that
$M = L_1$ and that $N$ is an Orlicz space, and that $X_i = a_i \xi_i$,
where $a_1$, $a_2,\dots,$\ $a_n$ are real numbers, and $\xi_1$, 
$\xi_2,\dots,$\ $\xi_n$ are identically distributed random variables.
Their paper was a major inspiration for this work, and there 
is quite some overlap with the techniques.

In this paper we will assume that all the vector spaces in question 
satisfy the triangle inequality.  It seems quite likely that many
of the formulae will extend to at least some quasi-Banach situations,
but we do not explore this possibility here.  We will normalize the
spaces so that $\snormo 1_M = \snormo{(1,0,\dots,0)}_N = 1$.

\begin{thm}
\label{t main}
Equation~(\ref{e main}) is true if there exists $1\le q < \infty$ such that 
$L_q$ embeds continuously via
the natural embedding into $M$.  In that case, the constants of approximation
in equation~(\ref{e main})
depend only upon $q$ and the constant of embedding.
\end{thm}

It is clear that at least some restriction must be placed upon $M$.  
For example,
if $M = L_\infty$ then $\snormo{\snormo{(X_i)}_N}_M = 
\snormo{(\snormo{X_i}_\infty)}_N$, and so equation~(\ref{e main})
does not necessarily hold.

The author would like to express his sincere appreciation to Mark Rudelson for
bringing this problem and reference \cite{gordon et al} to his attention,
and also to Joel Zinn for pointing out the reference \cite{marcus-zinn}.

\section{Proof of Main Theorem}

If $(x_i)$ is a sequence we will denote its non-increasing rearrangement
by $(x^*_i)$.  If $f$ is a function or random variable, we will denote
its non-increasing rearrangement by $f^\#$.

\begin{lemma}
\label{l L_1 l_infty}
Equation~(\ref{e main}) is true if $M = L_1$ and $N = \ell_\infty$.
\end{lemma}

\begin{proof}
This follows because 
\begin{equation}
\label{e max in Pr}
    {\textstyle\frac12}\,\measure\{ Y|_{[0,1]} > t\}
    \le
    \Pr(\max_i X_i > t)
    \le
    \measure\{ Y|_{[0,1]} > t\} .
\end{equation}
This has an elementary proof --- 
see for example \cite[Proposition 2.1]{hitczenko-montgomery-smith} or
\cite{gine-zinn}.  Thus
\begin{equation}
\label{e L_1 l_infty}
   \snormo{\snormo{(X_i)}_\infty}_1 \approx \int_0^1 Y(t) \, dt .
\end{equation}
\end{proof}

For each integer $1\le m\le n$, let $k_m$ denote the sequence space
$\snormo{(x_i)}_{k_m} = \sum_{i=1}^m x^*_i$.

\begin{lemma}
\label{l L_1 k_m}
For each positive integer $m$, 
equation~(\ref{e main}) is true if $M = L_1$ and $N = k_m$, with
constants of approximation independent of $m$.
\end{lemma}

\begin{proof}
Let $I_1$, $I_2,\dots,$\ $I_n$ be $\{0,1\}$-valued
independent random variables that
are also independent of $(X_i)$, where $\Pr(I_i = 1) = 1/m$.  
Applying equation~(\ref{e L_1 l_infty})
to the sequence $(I_i X_i)$ we obtain
\begin{equation}
\label{e Ii}
   \snormo{\snormo{(I_i X_i)}_\infty}_1 
   \approx \int_0^1 Y(mt) \, dt
   \approx \frac1m \left( \int_0^1 Y(t) \, dt + 
   \snormo{(Y(i)}_{k_m} \right).
\end{equation}
Next, let $\mathcal M$ denote the $\sigma$-field generated by $(I_i)$.
Then applying equation~(\ref{e Ii}), we see that
\[
   \E(\snormo{(I_i X_i)}_\infty | \mathcal M)
   \approx \frac1m \snormo{(X_i)}_{k_m} . \]
Thus we also obtain that
\[
   \snormo{\snormo{(I_i X_i)}_\infty}_1 
   = \snormo{ \E(\snormo{(I_i X_i)}_\infty | \mathcal M) }_1
   \approx
   \frac1m \snormo{\snormo{(X_i)}_{k_m}}_1 .\]
The result follows.
\end{proof}

Let $P$ denote the space of functions $f$ on $[0,n]$ for which its quasi-norm
\[
   \snormo f_P = \snormo{f^\#|_{[0,1]}}_M + \snormo{(f^\#(i))}_N
\]
is finite.  In fact this quasi-norm is equivalent to a norm, \emph{viz},
$\snormo f_{P'} = \snormo{f^\#|_{[0,1]}}_M 
    + \normo{\left(\int_{i-1}^i f^\#(t) \, dt\right)}_N$ (see for example
\cite[Section~7]{montgomery-smith-semenov}).  However we will content ourselves
with proving the following statement.

\begin{lemma}
\label{l dilate P} For any function $f$ on $[0,n]$ we have
$\snormo{f(\cdot/100)}_P \le 200 \snormo f_P$.
\end{lemma}

\begin{proof}
First, since $M$ satisfies the triangle inequality, it follows
that 
$\snormo{f(\cdot/100)^\#|_{[0,1]}}_M \le 100\snormo{f^\#|_{[0,1/100]}}_M$.  
Next, since $f^\#(i/100) \le f^\#([i/100])$, where $[t]$ denotes the integer
part of $t$, we see that
\begin{eqnarray*}
   \snormo{(f^\#(i/100))}_N 
   &\le& 100 \snormo{(f^\#(i))}_N + \sum_{i=1}^{99} f^\#(i/100) \\
   &\le& 100 \snormo{(f^\#(i))}_N + 100 \int_0^1 f^\#(t) \, dt \\
   &\le& 100 \snormo f_P .
\end{eqnarray*}
\end{proof}

Finally we need to cite a couple of results 
\cite[Theorem 6.1 and Theorem 7.1]{hitczenko-montgomery-smith}.
These concern maximal sums of vector valued random variables
$U = \max_k \normo{\sum_{i=1}^k Z_i}$, where $Z_1$, $Z_2,\dots,$\
$Z_n$ are Banach-valued independent random variables.
Let $V:[0,1]\to[0,\infty]$ be defined so that
\[
   \measure\{V>t\} = \min\left\{ 1 ,
   \sum_{i=1}^n \Pr(\snormo{Z_i} > t) \right\} .
\]

\begin{thm}
\label{t lp}
If
$p \ge 1$, 
then
$ \snormo U_p \approx U^\#(e^{-p}/4) + \snormo V_p $.
\end{thm}

\begin{thm}
\label{t r.i.}
Suppose that $L_q$ embeds continuously into $M$ via the natural embedding,
where $1 \le q < \infty$.  Then
$ \snormo U_M  \approx \normo U_1 + \normo V_M $,
where the constant of approximation depends only upon $q$ and the embedding
constant.
\end{thm}

\begin{proof}[Proof of Theorem~\ref{t main}]
Let us first show the lower bound.  Here the proof is very similar
to the proof of \cite[Theorem 27]{montgomery-smith-semenov}.
We know that
\begin{eqnarray*}
   \snormo{(x_i)}_N
   &=&
   \sup_{\snormo y_{N^*} \le 1}
   \sum_{i=1}^n x_i^* y_i^*  \\
   &=&
   \sup_{\normo y_{N^*} \le 1}
   \sum_{m=1}^n (y_m^*-y_{m+1}^*) \snormo{(x_i)}_{k_m} ,
\end{eqnarray*}
where by convention $y^*_{n+1} = 0$, and $N^*$ denotes the dual space
to $N$.  From this, we immediately see
that
\begin{eqnarray*}
   \E \snormo{(X_i)}_N
   &\ge& 
   \sup_{\normo y_{N^*} \le 1}
   \sum_{m=1}^n (y_m^*-y_{m+1}^*)
   \E\snormo{(X_i)}_{k_m} \\
   &\approx&
   \sup_{\normo y_{N^*} \le 1}
   \sum_{m=1}^n (y_m^*-y_{m+1}^*)
   \left(
   \int_0^1 Y(t) \, dt
   +
   \snormo{(Y(i))}_{k_m}
   \right)  \\
   &\approx&
   \int_0^1 Y(t) \, dt
   +
   \snormo{(Y(i))}_N 
\end{eqnarray*}
since $y_1^* \le 1$ whenever $\normo y_{N^*} \le 1$.
To finish the lower bound, we see that
\[ 2\snormo{ \snormo{(X_i)}_N }_M
   \ge \snormo{\snormo{(X_i)}_\infty}_M + \E \snormo{(X_i)}_N , \]
and the result follows by equation~(\ref{e max in Pr}).

Now let us focus on the upper bound.  Really the first part of this proof
follows by an inequality obtained independently by van Zuijlen 
\cite{van zuijlen 1}, \cite{van zuijlen 2}, \cite{van zuijlen 3},
and Marcus and Pisier \cite{marcus-pisier}.  But we
shall provide a self contained proof that is essentially a copy of 
the proof of this same result that may be found in
\cite[Theorem 5.1]{marcus-zinn}.  From Lemma~\ref{l dilate P},
it follows that $ \normo{Y(\cdot/100)}_P \le 200 \normo Y_P$.
We have that
\begin{eqnarray*}
   \Pr( \snormo{(X_i)}_N > 200 \normo Y_P )
   &\le&
   \Pr( \snormo{(X_i)}_N > \normo{Y(\cdot/100)}_P ) \\
   &\le&
   \Pr( \snormo{(X_i)}_N > \normo{(Y(i/100))}_N ) \\
   &\le&
   \Pr( \exists i : X_i^* > Y(i/100) ) \\
   &\le&
   \sum_{i=1}^n
   \Pr( X_i^* > Y(i/100) ) \\
   &\le&
   \sum_{i=1}^n
   \sum_{j_1<j_2<\cdots<j_i}
   \prod_{k=1}^i \Pr( X_{j_k} > Y(i/100) ) \\
   &\le&
   \sum_{i=1}^n
   \frac1{i!} \left(\sum_{j=1}^n \Pr(X_j > Y(i/100))\right)^i \\
   &\le&
   \sum_{i=1}^n
   \frac{i^i}{100^i i!} \\
   &\le&
   \frac1{4e} .
\end{eqnarray*}
Now we may apply Theorems~\ref{t lp} and~\ref{t r.i.} to 
$Z_i = X_i e_i \in N$, where $e_i$ denotes the $i$th unit vector.
In that case we see that $U = \snormo{(X_i)}_N$, and $V = Y|_{[0,1]}$,
and the result follows.
\end{proof}

\section{Application to Orlicz Spaces}

In this section we will recover some of the results of
Gordon, Litvak, Sch\"utt and Werner \cite{gordon et al}.
Suppose that $M$ and $N$ are Orlicz spaces constructed
from Orlicz functions $\Phi$ and $\Psi$ respectively.  Then
it can be seen that $P$ is equivalent to an Orlicz space whose
Orlicz function is given by 
\[
   \Theta(x) = \left\{
   \begin{array}{cl}
     \Psi(x) & \text{if }0\le x \le 1\\
     \Phi(x) & \text{if }x \ge 1.
   \end{array} \right.
\]
(Note that because of the normalisation on $M$ and $N$ that 
$\Phi(1) = \Psi(1) = 1$.  Note also that actually
$\Theta$ need not be an Orlicz function, but it is equivalent
to an Orlicz function:
$\tilde\Theta(x) = \int_0^x \frac{\Theta(t)} t \, dt $.)

Now let us consider the special case when $X_i = a_i \xi_i$ as described
in the introduction.  In that case it follows from the definitions that
there is an Orlicz function 
$\Lambda(t) = \Pr(\Theta(t \xi_1))$ such that
\[
   \snormo{(a_i)}_{L_\Lambda} \approx \snormo{\snormo{(a_i \xi_i)}_N}_M .
\]

Finally let us finish with a remark.  In \cite{gordon et al}, the authors
showed in the case that $M = L_1$
that their upper bound held even if the random variables were
not independent.  This can also hold in our more general case.
In \cite{montgomery-smith-semenov} was introduced the concept of what
it means for a rearrangement invariant space to be $D^*$-convex.
This property is held, for example, by all Orlicz spaces.
Following the proof of \cite[Theorem 27]{montgomery-smith-semenov},
it can be shown that equation~(\ref{e main}) holds even if the
sequence $(X_i)$ is not necessarily independent, as long as 
$M = L_1$ and $P$ is $D^*$-convex.  
It is easy to see from the definition that the condition that
$P$ be $D^*$-convex cannot be dropped.
We leave the details to the
interested reader.

\begin{thebibliography}{999}

\bibitem{carothers-dilworth} 
Carothers,~N.L.; Dilworth,~S.J. Inequalities for sums
of
independent random variables, {\em Proc. Amer. Math. Soc.} {\bf 104} (1988),
221--226.

\bibitem{gine-zinn} 
Gin\'e, E.; Zinn, J. 
Central limit theorems and weak laws of large numbers in certain Banach spaces. 
{\em Z. Wahrsch. Verw. Gebiete, {\bf 62} (1983), 323--354.}

\bibitem{gordon et al}
Gordon, Y.; Litvak, A.; Sch\"utt, C.; Werner, E.
Orlicz norms of sequences of random variables,
{\em preprint}.

\bibitem{hitczenko-montgomery-smith}
Hitczenko P.; Montgomery-Smith S. 
Measuring the magnitude of sums of independent random variables,
{\em to appear in Annals of Probab.}

\bibitem{johnson-schechtman} 
Johnson, W.B.; Schechtman, G. Sums of independent
random variables in rearrangement invariant function spaces, {\em
Ann. Probab.} {\bf 17} (1989), 789--808.

\bibitem{marcus-pisier}
Marcus, M.B.; Pisier, G. 
Characterizations of almost surely continuous $p$-stable random Fourier 
series and strongly stationary processes.
{\em Acta Math. {\bf 152} (1984), 245--301.}

\bibitem{marcus-zinn}
Marcus, M.B.; Zinn, J. 
The bounded law of the iterated logarithm for the weighted empirical 
distribution process in the non-i.i.d.
case. 
{\em Ann. Probab. {\bf 12} (1984), 335--360.}

\bibitem{montgomery-smith-semenov}
Montgomery-Smith, S.; Semenov, E. 
Random rearrangements and operators, 
{\em Voronezh Winter Mathematical Schools,
157--183, Amer. Math. Soc. Transl. Ser. 2, 184, 
Amer. Math. Soc., Providence, RI, 1998.}

\bibitem{rosenthal} Rosenthal, H.P. On the subspaces of $L_p$ 
($p > 2$) spanned by
sequences of independent symmetric random variables, {\em Israel
J. Math.} {\bf 8} (1970), 273--303.

\bibitem{van zuijlen 1}
van Zuijlen, M.C.A. 
Some properties of the empirical distribution function in the non-i.i.d.\
case.
{\em Ann. Statisitics {\bf 4}, (1976), 406--408.}

\bibitem{van zuijlen 2}
van Zuijlen, M.C.A. 
Properties of the empirical distribution function for independent 
nonidentically distributed random variables.
{\em Ann. Probability {\bf 6}, (1978), 250--266.}

\bibitem{van zuijlen 3}
van Zuijlen, M.C.A. 
Properties of the empirical distribution function for independent 
nonidentically distributed random vectors. 
{\em Ann. Probab. {\bf 10} (1982), 108--123.}

\end{thebibliography}

\end{document}
