\magnification=\magstep1
\baselineskip =5.5mm
\lineskiplimit =1.0mm
\lineskip =1.0mm

\def\sqr{\vcenter {\hrule height.3mm
\hbox {\vrule width.3mm height 2mm \kern2mm
\vrule width.3mm } \hrule height.3mm }}

\def\ts#1{{\textstyle{#1}}}
\def\ds#1{{\displaystyle{#1}}}
\def\moreproclaim{\par}
\def\heading #1:{\medskip\noindent{\bf #1.\ \ \ }}
\def\Proof:{\heading Proof:}
\def\Proofof #1:{\heading Proof of #1:}
\def\endproof{\hfill$\sqr$\bigskip}

\def\of{\bf}
\def\R{{\of R}}
\def\C{{\of C}}
\def\E{{\of E}}

\def\O{\Omega}
\def\w{\omega}
\def\widedot{\,\cdot\,}

\def\normo#1{\left\| #1 \right\|}

\def\modo#1{\left| #1 \right|}

\def\list#1#2{#1_1$, $#1_2,\ldots,$\ $#1_{#2}}
\def\ttss_#1{#1}
\def\set#1{\{\list\ttss#1\}}

\def\invp{{1\over p}}
\def\ominvp{{1-\invp}}
\def\half{{1\over2}}

\def\pipoT{\pi_{p,1}(T)}
\def\NCS{\ds{\left({N-1\atop S-1}\right)}}
\def\chiOn{{\chi_{\O_n}}}
\def\sumBnp{\sumN \modo{B_n}^p}

\def\sumN{\sum_{n=1}^N}
\def\sumS{\sum_{s=1}^S}
\def\summn{\sum_{m=1}^n}

\def\supS{\sup_{1\le s\le S}}


\centerline{\bf The $p^\invp$ in Pisier's Factorization Theorem}
\medskip
\centerline{\bf S.J.~Montgomery-Smith}

\bigskip

\beginsection Abstract

We show that the constants in Pisier's factorization theorem for
$(p,1)$-summing operators from $C(\O)$\ cannot be improved.

\bigskip
\centerline{* * * * * *}
\bigskip

A theorem of Pisier (see [3]) states the following.

\proclaim Theorem 1. Let $T:C(\O) \to X$\ be a bounded linear operator, where
$\O$\ is a compact Hausdorff topological space and $X$\ is a Banach space. Then
the following are equivalent.
\item{i)} $T$\ is $(p,1)$-summing.
\item{ii)} There is a constant $C<\infty$\ and a Radon probability measure
$\mu$\ on $\O$\ such that for all $f \in C(\O)$\ we have 
$$ \normo{Tf} \le C \, \normo f_{L_1(\mu)}^\invp
   \, \normo f_\infty^\ominvp . \eqno(1) $$
\item{iii)} There is a constant $C<\infty$\ and a Radon probability measure
$\mu$\ on $\O$\ such that for all $f \in C(\O)$\ we have 
$$ \normo{Tf} \le C \, \normo f_{L_{p,1}(\mu)} . \eqno(2) $$

If we examine the proof of this theorem carefully, we can deduce the following.
Let $\pipoT$\ denote the $(p,1)$-summing norm of $T$.

\proclaim Theorem 2. Let $T:C(\O) \to X$\ be a bounded linear operator as in
Theorem~1 that is $(p,1)$-summing. Then we can say the following.
\item{i)} The set of $C$\ and $\mu$\ that satisfy $(1)$\
coincide with the set of $C$\ and $\mu$\ that satisfy $(2)$.
\item{ii)} $\pipoT \le C$\ for all $C$\ satisfying $(1)$\ or $(2)$.
\item{iii)} We can choose $C = p^\invp \, \pipoT$\ in $(1)$\ and $(2)$.

%\item{i)} Let $C_1 = \pipoT$.
%\item{ii)} Let $C_2$ be the least $C$\ such that there is a Radon probability
%measure $\mu$\ on $\O$\ such that for all $f \in C(\O)$\ we have 
%$$ \normo{Tf} \le C \, \normo f_{L_1(\mu)}^\invp
%   \, \normo f_\infty^\ominvp . $$
%\item{iii)} Let $C_3$\ be the least $C$\ such that there is a Radon probability
%measure $\mu$\ on $\O$\ such that for all $f \in C(\O)$\ we have 
%$$ \normo{Tf} \le C \, \normo f_{L_{p,1}(\mu)} . $$
%\moreproclaim\noindent
%Then $C_2 = C_3$, and $C_1 \le C_2 \le p^\invp C_1$.

Part~(i) of this theorem is straight forward to demonstrate (we merely note
that one needs the inequality: $\normo f_{p,1} \le \normo f_1^\invp \,
\normo f_\infty^\ominvp $). Part~(ii) is also easy to verify. However,
to show part~(iii), the main part of Pisier's theorem, is much harder.
Furthermore, the methods do not indicate whether the $p^\invp$\ factor can be
made smaller, or even completely removed (that is, replaced by $1$).

The purpose of this paper is to show the rather surprising fact that the
$p^\invp$\ factor cannot be reduced at all.

\proclaim Theorem 3. Given $\epsilon>0$, there is an operator $T:C(\O) \to X$\
such that for any Radon probability measure $\mu$\ on $\O$, if $C$\ is the least
number satisfying $(1)$\ or $(2)$, then $C \ge p^\invp \, \pipoT \,
(1-\epsilon)$.

\heading Construction:
Let $1\le S \le N$\ be integers, and let $\O$\ be the collection of
$S$-subsets of $\set N$. For each $1\le n\le N$, let $\O_n = \{\, \w\in\O
: n\in\w \}$. We note the following facts for later on:
$$ \eqalignno{
   \modo{\O_n}             &= \NCS ,                 & (3) \cr
   {1\over N} \sumN \chiOn &= {S\over N}\,\chi_\O .  & (4) \cr }$$
We give $\O$\ the discrete topology, and define a norm
$\normo\widedot_*$\ on $C(\O)$\ by
$$ \normo f_* = \sup_{1\le n\le N} \sum_{\w\in\O_n} \modo{f(\w)} .$$
We let $T$\ be the canonical embedding
$$ T : \bigl( C(\O) , \normo\widedot_\infty \bigr)
   \to \bigl( C(\O) , \normo\widedot_* \bigr) .$$

\proclaim Lemma 4. For $1\le p<\infty$, the $(p,1)$-summing norm of
$T$\ may be estimated by
$$ \pipoT \le {N^{S-1+\invp}\over (S-1)!\,(Sp-p+1)^\invp} .$$

In order to show Lemma~4, we will need two more lemmas.

\proclaim Lemma 5. If $1\le p <\infty$, and $T:C(\O) \to X$\ is a bounded
linear operator, where $X$\ is a Banach space, then the $(p,1)$-summing
norm of $T$\ may be calculated by the formula
$$ \pipoT = \sup \left\{ \left( \sumS \normo{Tf_s}^p \right)^\invp
   \right\} ,$$
where the supremum is over all sequences $\list fS$\ of disjoint elements
of the unit ball of $C(\O)$.

\Proof: See [2], Lemma~6 or [1], Proposition~14.4.
\endproof

\proclaim Lemma 6. If $1\le n\le N$, then
$$ \modo{ \O_1 \cup \O_2 \cup \ldots \cup \O_n } \le
   {1\over (S-1)!} \summn (N-m)^{S-1} .$$

\Proof:
A simple counting argument shows that
$$ \modo{ \O_1 \cup \O_2 \cup \ldots \cup \O_m \setminus
   \O_1 \cup \O_2 \cup \ldots \cup \O_{m-1} } =
   \left({N-m \atop S-1}\right) ,$$
and this is bounded by $(N-m)^{S-1}/(S-1)!$.
\endproof

\Proofof Lemma 4:
By Lemma~5, it is easy to see that
$$ \pipoT
   = \sup \left\{ \left( \sumN \normo{\chi_{B_n}}_*^p \right)^\invp
   \right\}
   = \sup \left\{ \left( \sumN \modo{B_n}^p \right)^\invp
   \right\}, $$
where the supremum is over disjoint sets $\list BN \subseteq \O$\ such
that $B_n \subseteq \O_n$\ for each $1\le n\le N$. Since $\list \O N$\
interact with one another in a completely symmetric fashion, we may
assume, without loss of generality, that
$\modo{B_1} \ge \modo{B_2} \ge \ldots \ge \modo{B_N} $.

Now
$$ \sumBnp = \sumN \left( \summn \modo{B_m} \right)
                   \left( \modo{B_n}^{p-1} - \modo{B_{n+1}}^{p-1} \right)
.$$
(We take $B_{N+1}=\emptyset$.) Since $\modo{B_n}^{p-1} -
\modo{B_{n+1}}^{p-1} \ge 0$, and $B_1 \cup B_2 \cup \ldots \cup B_n
\subseteq \O_1 \cup \O_2 \cup \ldots \cup \O_n$, we have, by Lemma~6,
that
$$ \eqalignno{
   \sumBnp
   &\le {1\over(S-1)!} \sumN \left( \summn (N-m)^{S-1} \right)
              \left( \modo{B_n}^{p-1} - \modo{B_{n+1}}^{p-1} \right) \cr
   &= {1\over(S-1)!} \sumN (N-n)^{S-1} \modo{B_n}^{p-1} .\cr}$$
Now, applying H\"older's inequality and dividing, we deduce
$$ \left(\sumBnp\right)^\invp \le {1\over(S-1)!}
   \left( \sumN (N-n)^{Sp-p} \right)^\invp .$$
Finally, we estimate the last quantity by an integral, and derive
$$ \eqalignno{
   \left(\sumBnp\right)^\invp
   &\le {1\over(S-1)!} \left(\int_0^N x^{Sp-p} \,dx \right)^\invp \cr
   &\le {N^{S-1+\invp}\over (S-1)!\,(Sp-p+1)^\invp} ,\cr}$$
as desired.
\endproof

\Proofof Theorem 3:
By the hypothesis on $C$, there is a probability measure $\mu$\ on
$\O$\ such that
inequality $(1)$\ holds. In particular, if we substitute $f=\chiOn$, we
deduce that
$$ \modo{\O_n}^p = \normo{\chiOn}_*^p
   \le C^p \int \chiOn \,d\mu .$$
Hence
$$ {1\over N} \sumN \modo{\O_n}^p 
   \le C^p \int {1\over N} \sumN \chiOn
   \,d\mu ,$$
and so by equalities $(3)$\ and $(4)$\ we have
$$ C \ge {N^\invp \NCS \over S^\invp} .$$
Thus, by Lemma~4, we deduce
$$ C \ge {N^\invp \NCS (S-1)!\, (Sp-p+1)^\invp
             \over S^\invp N^{S-1+\invp}} \, \pipoT .$$
Choosing $N$\ much larger than $S$, we find that
$$ C \ge \left({Sp-p+1\over S}\right)^\invp \, \pipoT \, (1-\ts{\half}\epsilon)
   .$$ 
Finally, choosing $S$\ large, we have the desired result, that is,
$C \ge p^\invp \, \pipoT \, (1-\epsilon)$.
\endproof

\beginsection References

\halign{\rm#\hfil & \quad\vtop{\hsize=5.5 true
in\parindent=0pt\hangindent=0pt \strut\rm#\strut\smallskip}\cr
1. & Jameson G.J.O.: Summing and Nuclear Norms in Banach Space
Theory.
London Math.\ Soc., Student Texts 8, 1987.\cr
2. & Maurey B.: Type et cotype dans les espaces munis de structures
locales
inconditionelles, Expos\'es~24--25. In: Seminaire Maurey-Schwartz
1973--74 (Ecole Polytechnique).\cr
3. & Pisier G.: Factorization of operators through $L_{p\infty}$\
or $L_{p1}$\ and non-com\-mut\-at\-ive generalizations. Math.\
Ann.\
{\bf 276} 105--136 (1986).\cr
}

\bigskip

S.J.~Montgomery-Smith,

Department of Mathematics,

University of Missouri at Columbia,

Columbia, Missouri 65211,

U.S.A.

\bye
