%this is version of 30 june 1995
\documentstyle[12pt]{article}
\setlength\textwidth{6in}
\setlength\oddsidemargin{0.25in}
\topmargin -20mm
\footskip 12mm
%\textwidth 38pc
\textheight 56pc
\def\Bbb#1{{\mathchoice{\mbox{\bf #1}}{\mbox{\bf #1}}%
{\mbox{$\scriptstyle \bf #1$}}{\mbox{$\scriptscriptstyle \bf #1$}}}}
\def\N{\Bbb N}
\def\R{\Bbb R}
\def\C{\Bbb C}
\def\Z{\Bbb Z}
\def\T{\Bbb T}
\def\Q{\Bbb Q}
\def\E{\Bbb E}
\def\O{\Omega}
\def\sgn{{\rm sgn}}
\def\H{{\cal H}_P}
\def\e{\epsilon}
\def\A{{\cal A}}
\def\cT{{\cal T}}



\begin{document}
\title{Hahn's Embedding Theorem for orders
and harmonic analysis on groups with ordered duals}
\author{Nakhl\'e H. \ Asmar\\
 and 
 \\Stephen J.\ Montgomery--Smith\\
Department of Mathematics\\
University of
Missouri--Columbia\\
Columbia, Missouri 65211  U.\ S.\ A.
}
\date{}
\maketitle
                             %
                             %
                             %
                             %
                             %
                             %
                             %
                             %
\begin{abstract}
\baselineskip=10 pt
Let $G$ be a locally compact abelian group whose dual
group $\Gamma$ contains a Haar measurable order $P$.
Using the order $P$ we define the conjugate function
operator on $L^p(G)$, $1\leq p<\infty$,
 as was done by Helson \cite{hel1}.
We will show how to use Hahn's Embedding Theorem
for orders and the ergodic Hilbert transform to study the 
conjugate function.
Our approach enables us to 
define a filtration of the Borel $\sigma$-algebra
on $G$, which in turn will allow us to introduce 
tools from martingale theory into the analysis on 
groups with ordered duals.
We illustrate our methods by 
describing a concrete way to construct the conjugate function
in $L^p(G)$.  
This construction is
in terms of an unconditionally convergent 
conjugate series
whose individual terms are 
constructed from specific ergodic Hilbert transforms.
We also present a study of the square function associated with
the conjugate series.
\end{abstract}

\section{Introduction}
%
%
%
Throughout this paper, $G$ will denote a 
locally compact abelian group with dual group 
$\Gamma$.   
A Haar measure on $G$ will be symbolized by $\mu$.  
For $1\leq p<\infty$,
we denote by $L^p(G)$ the 
Banach space of Haar measurable functions
$f$ on $G$ such that $|f|^p$
is integrable.  The space of essentially
bounded Haar measurable functions
on $G$ will be denoted by $L^\infty(G)$.


An order on $\Gamma$ is a subset 
$P$ such that 
$P+P \subseteq P; P\cap (-P)=\{0\}; P\cup (-P)=\Gamma$.  
Given such a set $P$ we will write $\alpha >\beta$ 
to mean that $\alpha - \beta \in P\backslash \{0\}$.  
For $f\in L^2(G)$ we use the Fourier 
transform to define the conjugate function 
$\widetilde{f}$ of $f$ with respect to the order 
$P$ by the Fourier multiplier relation 
%%%%%%%%%%%%%%%%%%%%%%%%%%%%%%%%
%conjugatefunction%
%%%%%%%%%%%%%%%%%%%%%%%%%%%%%%%%%
\begin{equation}
\widehat{\widetilde{f}}(\chi)=- i \ {\rm sgn}_P
(\chi)\widehat{f}(\chi)
\label{conjugatefunction}
\end{equation}
%%%%%%%%%%%%%%%%%%%%%%%%%%%%%%%%
%conjugatefunction%
%%%%%%%%%%%%%%%%%%%%%%%%%%%%%%%%%
%
for almost all $\chi\in\Gamma$, where  
${\rm sgn}_P (\chi)= -1, 0 ,{\rm or}\ 1$, 
according as $\chi\in(-P)\backslash \{0\}$, 
$\chi=0, {\rm or}\ \chi\in P\backslash \{0\}$.
When $G$ is compact, these definitions are due to
Helson \cite{hel1} and \cite{hel2}.\\
%
%
%
In \cite{gar}, Garling observed that the conjugate 
function on $\T^N$,
 defined using a 
lexicographic order on $\Z^N$ is 
connected in a natural way to martingale theory.  
Using this connection, 
Garling gave simple proofs of basic properties of 
the conjugate function in this setting.
Our goal in this paper is to show how
certain notions related to 
Hahn's Embedding Theorem for orders (\cite[Chapter IV]{fu})
can be used to introduce similar tools from
probability theory in the study of conjugate functions on groups
with an arbitrary measurable order on the dual group. 
We present the material related to Hahn's Theorem 
in Section 2.  In particular,
we will state a structure theorem 
(Theorem (\ref{theorem3}) below),
which describes an order in terms of a 
unique chain of convex subgroups of $\Gamma$.
When $\Gamma$ is not necessarily discrete,
the chain of subgroups may contain elements that are not
Haar measurable.  
In Section 3, we will show how to obtain a structure
theorem, similar to the one in Section 2,
while avoiding nonmeasurabele subgroups of 
$\Gamma$.  This study is based on the work of Hewitt and Koshi 
\cite{hk} concerning measurable orders.
In Sections 4,
we use the structure theorems for orders 
to define our construction of the conjugate function
as a martingale difference series whose individual 
terms are constructed by using different ergodic Hilbert transforms.
In Section 5, we will show that the conjugate series of 
$f\in L^p(G)$ is unconditionally
convergent in $L^p(G)$ when $1<p<\infty$, 
and is unconditionally convergent in
$L^{1,\infty}(G)$ when $p=1$.
These results yield a concrete way for 
constructing the conjugate function
on abstract groups.
We end the paper with a study of the
square function associated with the conjugate series.


%%%%%%%%%%%%%%%%%%%%%%%%%%%%%%%%%%%%%%%%%%%%%%%%%%
\section{Orders on discrete groups}
\newtheorem{theorem1}{Theorem}[section]
\newtheorem{archimedean}[theorem1]{Theorem}
\newtheorem{theorem2}[theorem1]{Theorem}
\newtheorem{remarks}[theorem1]{Definitions and Remarks}
\newtheorem{lemma1}[theorem1]{Lemma}
\newtheorem{convex-implies-pure}[theorem1]{Lemma}
\newtheorem{lemma2}[theorem1]{Lemma}
\newtheorem{theorem3}[theorem1]{Theorem}
\newtheorem{theorem4}[theorem1]{Theorem}
%%%%%%%%%%%%%%%%%%%%%%%%%%%%%%%%%%%%%%%%%%%%%%%%%%%%%%%%%
%%%%%%%%%%%%%%%%%%%%%%%%%%%%%%%%%%%%%%%%%%%%%%%%%%%%%%%%%
%%%%%%%%%%%%%%%%%%%%%%%%%%%%%%%%%%%%%%%%%%%%%%%%%%%%%%%%%
%%%%%%%%%%%%%%%%%%%%%%%%%%%%%%%%%%%%%%%%%%%%%%%%%%%%%%%%%
%%%%%%%%%%%%%%%%%%%%%%%%%%%%%%%%%%%%%%%%%%%%%%%%%%%%%%%%%
%%%%%%%%%%%%%%%%%%%%%%%%%%%%%%%%%%%%%%%%%%%%%%%%%%%%%%%%%
We start by collecting facts leading to a 
 structure theorem for orders on discrete groups
 (Theorem \ref{theorem3} below).  
 This requisite material is taken
 from \cite[Chapter IV]{fu} where it is presented 
 as a background for the proof of Hahn's
 Embedding Theorem for orders \cite[Theorem 16, p. 59]{fu}.
 Our presentation is simplified by the fact 
 that the groups are all abelian.  

%%%%%%%%%%%%%%%%%%%%%%%%%%%%%%%%%%%%%%%%%%%%%%%%%%%%%%%%%
%%%%%%%%%%%%%%%%%%%%%%%%%%%%%%%%%%%%%%%%%%%%%%%%%%%%%%%%%
%%%%%%%%%%%%%%%%%%%%%%%%%%%%%%%%%%%%%%%%%%%%%%%%%%%%%%%%%
{\bf Definitions and basic Properties} 
Let $\Gamma$ denote an infinite torsion-free
abelian group.  The topology on $\Gamma$ will play no role
in this section.  An order on $\Gamma$ will be denoted by
$P(\Gamma)$ or simply $P$.  Because sometimes we
will be dealing with more than one order
on a given group, it will be convenient
to write $(\Gamma,P)$ or $(\Gamma,P(\Gamma))$ 
to denote the group and the given order on it.
A subset $J\subset \Gamma$ is called convex
if whenever $a , b  \in J, c \in \Gamma$
and $a\leq c \leq b$, then $c\in J$.
As we will see, this notion plays a prominent 
role in the theory
of orders. For the reader's convenience, we list
a few properties of convexity 
that will be used 
in the sequel. 
(See \cite[pp. 18-19, and Chapter IV]{fu}.)\\
%%%%%%%%%%%%%%%%%%%%%%%%%%%%%%%%%%%%%%%%%%%%%%%%%%%%%%%%%
%%%%%%%%%%%%%%%%%%%%%%%%%%%%%%%%%%%%%%%%%%%%%%%%%%%%%%%%%
(a)  A subgroup $C\subset \Gamma$ is convex
if and only if $P(\Gamma)\cap C$ is convex in $P(\Gamma)$.\\
(b)  If $B\subset C\subset \Gamma$, and if
$B$ is convex in $C$ and $C$ is convex in $\Gamma$,
 then $B$ is convex in $\Gamma$.
\\
(c)  The intersection of convex subgroups is again
a convex subgroup.  Thus, if $A$ is a
subset of  $\Gamma$, there is a smallest
convex subgroup containing $A$.  We will denote
this subgroup by 
$\{A\}_\Box$.
If $A$ is a subgroup, then 
$\{A\}_\Box = (A+P)\cap (A-P)$.\\
(d)  A subgroup $C\subset \Gamma$ is called
principal if 
$C=\{c\}_\Box$ for some $c\in \Gamma$.\\
(e) Let $\Gamma$ and  
$\Gamma^\prime$ 
be two ordered groups.  A homomorphism 
$\phi \ : (\Gamma, P(\Gamma) ) \longrightarrow
 		(\Gamma^\prime, P(\Gamma^\prime) )$,
is called an order homomorphism if 
$\phi (P(\Gamma) )\subset P(\Gamma^\prime)$.
It is clear that if $\phi $ is an order homomorphism,
then $\ker \phi$ is a convex subgroup of
$\Gamma$.  Conversely, if $H$ is a 
convex subgroup of $(\Gamma, P(\Gamma) )$,
then we can define an order on the quotient group 
$\Gamma/H$ by:  $a+H\in P(\Gamma/H) \Leftrightarrow
a\in P$.  To verify this claim, 
suppose that  $a + H= b+H$, and, say, 
$a\in P$ and $b\in -P$.  
Then $0\leq a\leq a-b \in H$.  Since $H$ 
is convex, it follows that $a \in H$, and so 
$a+H=b+H=0+H$, which shows that
$ P(\Gamma/H)$ is indeed an order on  $\Gamma/H$.
It is also clear that the 
natural homomorphism 
$\pi :\ (\Gamma, P(\Gamma) )\longrightarrow 
(\Gamma/H, P(\Gamma/H) )$
is an order homomorphism.  We have thus the following 
useful theorem.  (See \cite[Theorem 7, p.21]{fu}.)
 		





%%%%%%%%%%%%%%%%%%%%%%%%%%%%%%%%%%%%%%%%%%%%%%%%%%%%%%%%%
%%%%%%%%%%%%%%%%%%%%%%%%%%%%%%%%%%%%%%%%%%%%%%%%%%%%%%%%%
%%%%%%%%%%%%%%%%%%%%%%%%%%%%%%%%%%%%%%%%%%%%%%%%%%%%%%%%%
%%%%%%%%%%%%%%%%%%%%%%%%%%%%%%%%%%%%%%%%%%%%%%%%%%%%%%%%%
\begin{theorem1}
Suppose that $\Gamma$ is an ordered group and 
$H$ is a subgroup of $\Gamma$.  If $H$ is convex 
then the natural homomorphism $\pi : \ \ 
(\Gamma,P) \rightarrow (\Gamma/ H,\pi(P))$ is 
an order homomorphism.  Conversely, suppose that
$\phi : \ \ 
(\Gamma,P) \rightarrow (\Gamma^\prime,P^\prime)$ 
is an order homomorphism with $\ker \phi= H$.  Then $H$ is a
convex subgroup of $\Gamma$.  
\label{theorem1}
\end{theorem1}
%%%%%%%%%%%%%%%%%%%%%%%%%%%%%%%%%%%%%%%%%%%%%%%%%%%%%%%%%
%%%%%%%%%%%%%%%%%%%%%%%%%%%%%%%%%%%%%%%%%%%%%%%%%%%%%%%%%
%%%%%%%%%%%%%%%%%%%%%%%%%%%%%%%%%%%%%%%%%%%%%%%%%%%%%%%%%
%%%%%%%%%%%%%%%%%%%%%%%%%%%%%%%%%%%%%%%%%%%%%%%%%%%%%%%%%
{\bf Definition}  An order $P$ on $\Gamma$ is called 
Archimedean if
 given $a,b\in P\setminus \{0\}$, then there is 
 a positive 
 integer $n$ such that $na>b$.\\
%%%%%%%%%%%%%%%%%%%%%%%%%%%%%%%%%%%%%%%%%%%%%%%%%%%%%%%%%
%%%%%%%%%%%%%%%%%%%%%%%%%%%%%%%%%%%%%%%%%%%%%%%%%%%%%%%%%

 Archimedean orders have a simple characterization
 in terms of real-valued homomorphisms 
 due to O.\ H\"{o}lder
 (\cite[Theorem 1, p. 45]{fu}).
%%%%%%%%%%%%%%%%%%%%%%%%%%%%%%%%%%%%%%%%%%%%%%%%%%%%%%%%%
%%%%%%%%%%%%%%%%%%%%%%%%%%%%%%%%%%%%%%%%%%%%%%%%%%%%%%%%%
%%%%%%%%%%%%%%%%%%%%%%%%%%%%%%%%%%%%%%%%%%%%%%%%%%%%%%%%%
%%%%%%%%%%%%%%%%%%%%%%%%%%%%%%%%%%%%%%%%%%%%%%%%%%%%%%%%%

\begin{archimedean}
An order $P$ on $\Gamma$ is Archimedean 
if and only if $\Gamma$ is isomorphic to a subgroup of $\R$.
\label{archimedean}
\end{archimedean}
%%%%%%%%%%%%%%%%%%%%%%%%%%%%%%%%%%%%%%%%%%%%%%%%%%%%%%%%%
%%%%%%%%%%%%%%%%%%%%%%%%%%%%%%%%%%%%%%%%%%%%%%%%%%%%%%%%%
There is another useful characterization of 
Archimedean orders in terms of convex subgroups 
(\cite[Corollary 5, p.\ 47]{fu}).

%%%%%%%%%%%%%%%%%%%%%%%%%%%%%%%%%%%%%%%%%%%%%%%%%%%%%%%%%
%%%%%%%%%%%%%%%%%%%%%%%%%%%%%%%%%%%%%%%%%%%%%%%%%%%%%%%%%
%%%%%%%%%%%%%%%%%%%%%%%%%%%%%%%%%%%%%%%%%%%%%%%%%%%%%%%%%
%%%%%%%%%%%%%%%%%%%%%%%%%%%%%%%%%%%%%%%%%%%%%%%%%%%%%%%%%

\begin{theorem2}
Suppose that $\Gamma$ is an ordered group, 
then $\Gamma$ is Archimedean ordered 
if and only if the only convex subgroups 
of $\Gamma$ are $\{0\}$ and $\Gamma$.
\label{theorem2}
\end{theorem2}

%%%%%%%%%%%%%%%%%%%%%%%%%%%%%%%%%%%%%%%%%%%%%%%%%%%%%%%%%
%%%%%%%%%%%%%%%%%%%%%%%%%%%%%%%%%%%%%%%%%%%%%%%%%%%%%%%%%

Following \cite[Chapter IV, Section 3]{fu}, 
we let $\Sigma$ denote the system of all 
convex subgroups in $\Gamma$.  This system 
is in fact a chain containing $\{0\}$ 
and $\Gamma$.  Hence if $C$ and $D$ are in $\Sigma$, then
either $C\subset D$ or $D\subset C$.  Also,
 whenever 
$\{C_\lambda\}_{\lambda\in\Lambda}$ 
is a collection from $\Sigma$, then 
$\bigcap_{\lambda\in\Lambda}C_\lambda$ 
and 
$\bigcup_{\lambda\in\Lambda}C_\lambda$ 
are again in $\Sigma$.  By a jump in 
$\Sigma$ we mean a pair of subgroups 
$C$ and $D$ such that $D\subset C, D\neq C$, 
and $\Sigma$ contains no subgroups 
between $C$ and $D$.  A jump will be denoted by 
$D\prec C$. \\ 
It is a fact that a 
subgroup $C\in\Sigma$ is the greater 
member of a jump (i.\ e.\ $D\prec C$) if and only 
if $C$ is a principal convex subgroup.  That is,
$$C= \{a\}_\Box$$
for some $a\in \Gamma$ (see p. 54 of \cite{fu}).\\
Let 
$\Sigma_0$
denote the system of principal convex 
subgroups, and let $\Pi$ be an indexing set for 
$\Sigma_0$.  Order $\Pi$ as follows:  
for $\rho, \pi\in \Pi$, 
set 
$\pi\leq\rho$ if and only if 
$C_\rho\subset C_\pi$.  
With this order, $\Pi$ has a 
maximal element 
$\alpha_0$ 
corresponding to $\{0\}\in \Sigma_0$.
  Thus $C_{\alpha_0}=\{0\}$.
For notational convenience, let 
$D_{\alpha_0}=\emptyset$.  
Let 
$D_\pi\prec C_\pi$
denote a jump in $\Sigma$, with $\pi<\alpha_0$.  
The quotient group $C_\pi/ D_\pi$ has 
no nontrivial convex subgroups.  
By Theorem (\ref{theorem2}), $C_\pi/ D_\pi$ is 
Archimedean ordered.  Let
%
%%%%%%%%%%%%%%%%%%%%%%%%%%%%%%%%
%%%%%%%%%%%%%%%%%%%%%%%%%%%%%%%
\begin{equation}
\psi_\pi :\ \  C_\pi/ D_\pi \rightarrow \R
\label{psi-sub-pi}
\end{equation}
%%%%%%%%%%%%%%%%%%%%%%%%%%%%%%%%
%%%%%%%%%%%%%%%%%%%%%%%%%%%%%%%%
%
denote the order isomorphism mapping $C_\pi/D_\pi$ 
into a subgroup of $\R$, and let
%
%
%
\begin{equation}
L_\pi :\ \  C_\pi \rightarrow \R
\label{L-sub-pi}
\end{equation}
%
%
%
denote the composition of $\psi_\pi$ 
with the natural homomorphism of 
$C_\pi$ onto $C_\pi/ D_\pi$.  Then $L_\pi$ 
is an order homomorphism of $C_\pi$ into 
$\R$ with $\ker L_\pi=D_\pi$.  
Since $\R$ is a divisible group,
the homomorphism $L_\pi$
can be extended to a homomorphism of the 
entire group $\Gamma$ into $\R$.
(See \cite[Theorem A.7, p. 441]{hr1}.)
We keep the same notation for this extension.
The next 
theorem summarizes this discussion.  
It is a basic result of this section
and will be used in defining our construction
of the conjugate function.
%%%%%%%%%%%%%%%%%%%%%%%%%%%%%%%%%%%%%
%%%%%%%%%%%%%%%%%%%%%%%%%%%%%%%%%%%%%
%THEOREM3%%%%%%%%%%%%%%%%%%%%%%%%%%%%%%%%%%%%
%%%%%%%%%%%%%%%%%%%%%%%%%%%%%%%%%%%%
\begin{theorem3}
Let $\Gamma$ be an infinite discrete 
(torsion-free) ordered group with order 
$P$.  Let $\Sigma$ denote the chain of 
convex subgroups of $\Gamma$ and $\Sigma_0$ 
the subcollection of principal convex 
subgroups indexed by the ordered set $\Pi$.  
There is a collection of real-valued 
homomorphisms 
$\{L_\pi, \pi\in \Pi \}$
of $\Gamma$ into $\R$ 
such that, for every jump 
$D_\pi\prec C_\pi$, 
we have \\
(i)  \ \ $ L_\pi\left( D_\pi\right)=\{0\}$; and\\
(ii) $\sgn (L_\pi(\chi))=\sgn_P(\chi)$\\
for all $\chi\in C_\pi\setminus D_\pi$.
\label{theorem3}
\end{theorem3}
%%%%%%%%%%%%%%%%%%%%%%%%%%%%%%%%%%%%%
%%%%%%%%%%%%%%%%%%%%%%%%%%%%%%%%%%%%%
%%%%%%%%%%%%%%%%%%%%%%%%%%%%%%%%%%%%%
%%%%%%%%%%%%%%%%%%%%%%%%%%%%%%%%%%%%%
%
Observe that, in the notation of the previous theorem, we have 
%
%
%
\begin{equation}
\Gamma =\bigcup_{\alpha\in\Pi} C_\alpha\backslash D_\alpha.
\label{slices}
\end{equation}
%
%
%
In fact, given $x\in\Gamma$, we have 
$\{x\}_\Box=C_\alpha$ for some $\alpha\in \Pi$.  
Since $D_\alpha$ is strictly contained 
in $C_\alpha$, it follows that $x$ 
belongs to $C_\alpha\backslash 
D_\alpha$ which proves (\ref{slices}).\\
The following example will illustrate many of the results of this section.\\

{\bf  Example}  Suppose that $( T, <)$ is an
ordered set and that $\gamma$ is a limit 
ordinal.  Let $\Z^{(T,\gamma)}$ be the group
of sequences $(z_t)_{t\in T}$ such that the set 
$\{ t:\ \ z_t\neq 0\}$ is reverse well ordered (i.\ e.,
every subset has a largest element) with order type less than
$\gamma$.  Define an order as follows:  If 
$z=(z_t)_{t\in T}\in \Z^{(T,\gamma)}$ and 
$t_0=\max \{t:\ \ z_t\neq 0\}$, then $z\in P$
if and only if $z_{t_0} \ge 0$.\\
In this example we see that $\Sigma_0$
is order isomorphic $-T$, that
$C_\alpha=\{(z_t):\ z_t=0,\ {\rm for}\ t\geq\alpha\}$,
that 
$D_\alpha=\{(z_t):\ z_t=0,\ {\rm for}\ t> \alpha\}$,
and that $L_\alpha ((z_t))=z_\alpha$.\\
%
We can also construct $\R^{(T,\gamma)}$ similarly.\\


%%%%%%%%%%%%%%%%%%%%%%%%%%%%%%%%%%%%%%%%%%%%%%%%%%%%%%%%%
%%%%%%%%%%%%%%%%%%%%%%%%%%%%%%%%%%%%%%%%%%%%%%%%%%%%%%%%%


%%%%%%%%%%%%%%%%%%%%%%%%%%%%%%%%%%%%%%%%%%%%%%%%%%%%%%%%%%
%%%%%%%%%%%%%%%%%%%%%%%%%%%%%%%%%%%%%%%%%%%%%%%%%%%%%%%%%%
%%%%%%%%%%%%%%%%%%%%%%%%%%%%%%%%%%%%%%%%%%%%%%%%%%%%%%%%%%
%%%%%%%%%%%%%%%%%%%%%%%%%%%%%




\section{Orders on locally compact abelian groups}
%
\newtheorem{theorem3.1}{Theorem}[section]
\newtheorem{remark3.2}[theorem3.1]{Remark}
\newtheorem{addedlemma}[theorem3.1]{Lemma}
\newtheorem{theorem3.3}[theorem3.1]{Theorem}
\newtheorem{remarks3.4}[theorem3.1]{Remarks}
\newtheorem{proposition31}[theorem3.1]{Proposition}
\newtheorem{lemma31}[theorem3.1]{Lemma}
\newtheorem{theorem32}[theorem3.1]{Theorem}
\newtheorem{theorem33}[theorem3.1]{Theorem}
\newtheorem{theorem34}[theorem3.1]{Theorem}
\newtheorem{theorem35}[theorem3.1]{Theorem}
\newtheorem{theorem36}[theorem3.1]{Theorem}
%
%
In this section we prove a general version 
of Theorem (\ref{theorem3}) for measurable orders.  
%
The difficulty here is due to 
the fact that, in general, the chain
of convex subgroups in Theorem (\ref{theorem3})
may contain nonmeasurable 
subgroups, or jumps of the form $D_\alpha\prec C_\alpha$
with $C_\alpha\setminus D_\alpha$ having measure zero.
To overcome these measure theoretic problems, we will
find a smallest open principal convex subgroup
of $\Gamma$ which will determine when to stop the chain
while still being able to separate with
continuous real-valued homomorphisms as in Theorem
 (\ref{theorem3}).  
 This section is based on the
 study of orders of Hewitt and Koshi \cite{hk}.
 Indeed, Theorem \ref{theorem35} 
 below, is a combination of results from \cite{hk} and the
 material from the previous section.
%
% 
%
Throughout this section $\Gamma$ will 
denote an infinite locally compact torsion-free 
abelian group.  
The following
basic properties of measurable orders will be needed.

%%
%%
%%
\begin{theorem3.1}
(a)  If $P$ is a measurable order, then $P$ 
has a nonvoid interior (\cite[Theorem (3.1)]{hk}.
Consequently,  if 
$P$ is a measurable order, then $(-P)$ has nonvoid interior.\\
(b)  If $\Gamma$ is an infinite compact 
torsion-free group, then every 
order on $\Gamma$ is dense and has 
void interior (\cite[Theorem (3.2)]{hk}.  
Consequently, every order on a compact 
infinite group is nonmeasurable.\\
\label{theorem3.1}
\end{theorem3.1}
%
%
%
\begin{remark3.2}
{\rm Suppose that $\Gamma$ is a locally compact 
abelian group, and that $P$ is a 
measurable order on $\Gamma$.  
%
Use the structure theorem for locally 
compact abelian groups to write $\Gamma$ 
as $\Gamma=\R^a\times\O$ where $a$ is a nonnegative
integer and $\O$ 
contains a compact open subgroup $\O_0$
(\cite[Theorem (24.30)]{hr1}).  
The fact that $P$ is measurable automatically 
implies that either $\Gamma$ is 
discrete or $a>0$. 
%
In fact, if $a=0$, then $\Gamma=\O$ and so $\O_0$ 
is a compact open 
subgroup of $\Gamma$.  
%
The restriction of $P$ to $\O_0$ is 
a measurable order in $\O_0$.  
But since $\O_0$ is compact, it 
follows from Theorem (\ref{theorem3.1})(b) that $\O_0=\{0\}$, 
and so $\Gamma$ is discrete. }  
\label{remark3.2}
\end{remark3.2}
%
%
%
%
 Henceforth to avoid the cases 
 treated in the previous section, 
 we will assume that $\Gamma=\R^a\times\O$ where $a>0$.
 For use in the sequel, we need the following result due to 
 Hewitt and Koshi \cite[Theorem (3.12)]{hk}. %
 %
 %
  %
  %
\begin{theorem3.3}
Let $P$ be an order on $\R^a\times B$\ where 
$a>0$ and $B$ is an infinite torsion-free
locally compact abelian group that is 
the union of its compact open subgroups.  
Suppose that $P$ has nonempty interior.
Then there is a continuous real-valued 
homomorphism $L:\ \R^a\longrightarrow \R$\ such that
\begin{equation}
L^{-1}(] 0,\infty [ )\times B\subset {\rm int} (P) \subset P\subset
L^{-1}([ 0,\infty [ )\times B.
\label{orders-on-R-a}
\end{equation}
The order $P\cap (L^{-1} (\{0\})\times B)$
is arbitrary.  
\label{theorem3.3}
\end{theorem3.3}
%
%
%
%
%
%
%
%
\begin{remarks3.4}
{\rm (a)  It follows from Theorems (\ref{theorem3.1}) and
(\ref{theorem3.3}) that an order 
$P$ on $\R^a$ is measurable if and only if 
it is not dense in $\R^a$.\\
(b)  Let $P$ be a measurable order on $\R^a \times \O$, 
and let $B$ denotes the union of all the compact open
subgroups of $\O$. 
The restriction of $P$ to $\R^a \times B$
is a nondense order, and so by
Theorem (\ref{theorem3.3}) there is a
continuous real-valued homomorphism
$L:\ \R^a \longrightarrow \R$ such that
$$ L^{-1} (]0,\infty]) \times B \subset P .$$}
\label{remarks3.4}
\end{remarks3.4}
%
%
%
We can now describe our candidate 
for a smallest open convex 
principal  subgroup of $\Gamma$.
Let
%
%
%
\begin{equation}
H=\left\{
y\in\O:\ \ \R^a\times\{y\} \ 
{\rm \ has\ nonvoid\ intersections\ with}\  P\  
{\rm \ and\ } \ (-P)
\right\}.
\label{candidate}
\end{equation}
%
%
%
%
%
%
\begin{proposition31}
Let $P$ be a measurable order 
on $\R^a\times \O$, and let $H$ be as in (\ref{candidate}).  
%
Then $H$ is a subgroup of $\O$ that contains all the compact
open subgroups of $\O$;
and $R^a\times H$ is an open convex 
subgroup of $\Gamma$.
\label{proposition31}
\end{proposition31}
%
%
%
{\bf Proof.}  The fact that $H$ contains all the 
compact open subgroups of 
$\O$ follows from Theorem (\ref{theorem3.3}).
Also, the fact that $H$ is a subgroup is easily verified.
Since $\R^a\times H$ is a subgroup with nonvoid interior
it follows immediately that the subgroup is open.
%
%
%
To establish the convexity of $\R^a\times H$,
suppose that 
%
%
%
$$0< (t,y) < (t^\prime, y^\prime)$$
% 
%
%
with $y^\prime\in H,\ t,\ t^\prime \in \R^a$. 
To show that 
 $(t,y)\in \R^a\times H$, it is enough to 
 find $x\in \R^a$ with $(x,y)\in -P$.
 Since $y^\prime \in H$, we can find
 $x\in \R^a$ such that $(x,y^\prime)\in -P$.
 Hence 
 $(t-t^\prime ,y-y^\prime) +(x,y^\prime)\in (-P)$, 
 or,
 $(t-t^\prime +x , y)\in (-P)$. \\
%
%
%
\begin{lemma31}
Let $P$ be a measurable order in $\Gamma$ 
and let $H$ be as in (\ref{candidate}).  
%
For $y\in H$, let $A_y=(\R^a\times\{y\})\cap P$ and 
$B_y=(\R^a\times\{y\})\cap (-P)$.  
%
Then $A_y$ and $B_y$ are nondense in $\R^a\times\{y\}$. 
\label{lemma31}
\end{lemma31}
%
%
%
{\bf Proof.}  
It is enough to deal with the set $A_y$
with $y\neq 0$.  
%
Assume that 
$(\R^a\times\{y\})\cap P$ is dense 
in $\R^a\times\{y\}$.  
%
Let $(s,y)$ be any element of $\R^a\times\{y\}$, and let 
$(s_0,y)\in (\R^a\times\{y\}) \cap P$ 
be such that $L(s_0)> L(s)$.  
%
We have $L(s_0-s)>0$, and so from Remark (\ref{remarks3.4}) (b)
we see that $(s_0-s,0)\in P$.  
%
Hence $(s_0,y)+(s-s_0,0)=(s,y)\in P$ 
implying that $\R^a\times\{y\}\subset P$ 
which contradicts the fact that $y\in H$.  
%
Thus $P\cap(\R^a\times\{y\})$ is 
nondense in $\R^a\times\{y\}$.\\
%
%
%
%

We can now prove a special separation theorem for orders. 
(Compare with \cite[Theorem 3.7]{hk}.) 
%
%
%
\begin{theorem32}
Let $P$ be a measurable
order on $\Gamma$, let 
$H$ be as in (\ref{candidate}), 
and let $L$ be as in Remarks (\ref{remarks3.4}) (b).  
For every $y\in H$, there is a real number 
$\alpha ( y )$ such that \\
(i)  $L^{-1}\left(]-\infty,\alpha(y)
[\right) \times \{y\}\subset -P,$\\
%
%
(ii)  $L^{-1}\left(]\alpha(y),\infty[\right) \times \{y\}\subset P.$\\
Moreover, the mapping $y\mapsto \alpha(y)$ is a continuous real-valued homomorphism from $H$.
\label{theorem32}
\end{theorem32}
%
{\bf Proof.}  We will write the elements of 
$\Gamma=\R^a\times \O$ as $(x,y)$ where $x \in \R^a$, 
and $y\in \O$.  
%
Suppose that $(x_1,y) \in P$ and $L(x_2)>L(x_1)$.  
%
Then, $L(x_2-x_1)>0$, and so from Remarks (\ref{remarks3.4}) (b)
we have that $(x_2-x_1,0)\in P\backslash \{0\}$, 
and consequently, $(x_2,y)\in P\backslash \{0\}$.  
%
Similarly, if $(x_1,y)\in (-P)$, and $L(x_2)<L(x_1)$, 
then $(x_2,y) \in (-P)\backslash \{0\}$.  
%
>From these observations and the definition of 
$H$, we see that the following holds  for every $y\in H$:
%%
%%
%%
\begin{eqnarray}
-\infty		&<&	\sup\left\{ 
			L(x):\ \ x\in \R^a,\ (x,y)\in (-P)\right\}
			\nonumber\\
		&=&	\inf\left\{ 
			L(x):\ \ x\in \R^a,\ (x,y)\in P)\right\}<\infty.
\label{prfth32,1}
\end{eqnarray}
%%
%%
%%
%%
For $y\in H$, let $\alpha(y)$ be defined 
by either the inf or the sup in (\ref{prfth32,1}).  
That $\alpha$
 is a continuous homomorphism follows exactly as in the 
 proof of \cite[Theorem 3.7]{hk}.  We omit the details.
 %
%
%
%
%

We restate Theorem (\ref{theorem32}) 
using separating homomorphisms that 
reflect the 
order on $\R^a\times H$.
%
\begin{theorem33}
Let $P$ be a measurable order on $\Gamma$, 
and let $H$, $L$, and $\alpha$ be as 
Theorem (\ref{theorem32}).  
%
Define the homomorphism $\tau$ on $\R^a\times H$ by\\
$$\tau (x,y) = L(x)- \alpha (y).$$
Then\\
%
(i)  $\tau$ is continuous on $\R^a\times H$;\\
%
(ii)
$\tau^{-1}\left(]-\infty,0[\right) \subset (-P);$\\
%
(iii)
$\tau^{-1}\left(]0,\infty[\right) \subset P.$\\
%
(iv)  The kernel of $\tau$ is locally null (i.\ e.\ , if $K$ is any compact subset of $\tau ^{-1}\left(\{0\}\right)$, then $\mu_\Gamma (K)=0$).
\label{theorem33}
\end{theorem33}
%
%
%
{\bf Proof.}  Assertions (i)-(iii) follow from the 
definitions of the homomorphisms $\tau$ and $\alpha$.  
%
Now suppose that 
$K$ is a compact subset of 
$\tau ^{-1}\left(\{0\}\right)$ and $\mu_\Gamma (K)>0$,
where $\mu_\Gamma$ is a Haar measure on $\Gamma$..  
%
Then $K-K$ contains an open neighborhood 
of the identity in $\Gamma$, and hence 
$\tau ^{-1}\left(\{0\}\right)$ is 
an open subgroup of $\Gamma$ which 
implies that 
$\R^a\times\{0\}\subset \tau ^{-1}\left(\{0\}\right)$.  
%
This is plainly a contradiction, and so (iv) holds.\\

  

  




\begin{theorem34}
Let $H$ be as in (\ref{candidate}).  
Then $\R^a\times H$ is a principal convex
 open subgroup of $\Gamma$.
\label{theorem34}
\end{theorem34}
%
%
%
{\bf Proof.}  Because of Proposition (\ref{proposition31}), 
all we need to show is that $\R^a\times H$ 
is a principal subgroup.  
%
For this purpose, let $z$ be any element 
in $\R^a\times H$ such that $\tau (z) >0$.  
%
We claim that $\R^a\times H=\{z\}_\Box$.  
%
Since by definition $\{z\}_\Box$ is the smallest 
convex subgroup containing $z$, and since 
$\R^a\times H$ is convex, it is 
enough to show that 
$\{z\}_\Box\supset \R^a\times H$.  
%
Let $x$ be any element of $P\cap (\R^a\times H)$.  
%
>From Theorem (\ref{theorem33}), $\tau (x)\geq 0$.  
%
Choose a positive integer 
$n$ such that $\tau (n z ) > \tau (x) \geq 0$.  
%
Again, from Theorem (\ref{theorem33}) 
we see that $0\leq x\leq n z$, and since 
$\{z\}_\Box$ is convex, it follows 
that $x\in \{z\}_\Box$.  
%
Hence $\{z\}_\Box\supset P\cap (\R^a\times H)$, 
and consequently $\{z\}_\Box\supset \R^a\times H$.\\ 
%
%
%

We are now ready to state the main result of this section.  
%
We write the group $\Gamma$ as $\R^a\times \O$, 
where $a>0$.  Let $\Sigma(H)$ denote the 
chain of convex subgroups of 
$\Gamma$ containing $\R^a\times H$, and let 
$\Sigma_0 (H)$
denote the chain of principal 
convex subgroups containing $ \R^a\times H$. 
%
Let $\Pi(H)$ be an indexing set for 
$\Sigma_0(H)$.  Order $\Pi(H)$ as 
we did in the discrete case:  
%
for $\rho, \pi\in \Pi(H)$, 
set 
$\pi\leq\rho$ if and only 
if $C_\rho\subset C_\pi$.  
%
With this order, $\Pi(H)$ has a maximal 
element $\alpha_0$ corresponding to 
$\R^a\times H\in \Sigma_0(H)$.
%
Hence $C_{\alpha_0}=\R^a\times H$.  
%
As in the discrete case, we denote 
a jump in $\Sigma(H)$ by 
$D_\pi\prec C_\pi$, where $C_\pi$ 
is a principal convex subgroup of $\Gamma$
containing $\R^a\times H$.  
%
Note that for $C_{\alpha_0}=\R^a\times H$ 
the jump occurs with an element outside 
of $\Sigma_0(H)$.  
%
We set, by definition, 
$D_{\alpha_0}=\ker \tau$ where 
$\tau$ is the homomorphism of Theorem (\ref{theorem33}).  
Hence $D_{\alpha_0}$ is locally null by
Theorem (\ref{theorem33}).
%%
%
%
Note that 
%
%
\begin{equation}
\Gamma=D_{\alpha_0}\cup 
\bigcup_{\alpha\in\Pi(H)} 
C_\alpha\backslash D_\alpha.
\label{slices'}
\end{equation}
%
%
% 
With the exception of 
$D_{\alpha_0}$, each set on the right 
side of (\ref{slices'}) is open.\\


%%%%%%%%%%%%%%%%%%%%%%%%%%%%%%%%%%%%%
%%%%%%%%%%%%%%%%%%%%%%%%%%%%%%%%%%%%%
%THEOREM35%%%%%%%%%%%%%%%%%%%%%%%%%%%%%%%%%%%%
%%%%%%%%%%%%%%%%%%%%%%%%%%%%%%%%%%%%
\begin{theorem35}
With the above notation, for every $\alpha \in \Pi(H), 
\alpha\neq \alpha_0$, there is a continuous 
real-valued homomorphism $L_\alpha$ on $\Gamma$ such that \\
%
%
%
(i)  $ L_\alpha \left(D_\alpha\right)=\{0\}$;\\
%
%
%
(ii) $\sgn_P (\chi)=\sgn(L_\alpha(\chi))$\\
for all $\chi\in C_\alpha\setminus D_\alpha$.  
When $\alpha=\alpha_0$, there is a real-valued 
homomorphism $L_{\alpha_0}$ on $\Gamma$, such that\\
%
%
%
%
%
%
(iii)   $ L_{\alpha_0}\left( D_{\alpha_0}\right)=\{0\}$;\\
%
%
%
(iv) $\sgn_P (\chi)=\sgn(L_{\alpha_0} (\chi))$\\
for all $\chi\in C_{\alpha_0}\setminus D_{\alpha_0}$.  \\
(Since $D_{\alpha_0}$ is locally null, (iv) holds for
locally almost all $\chi\in C_{\alpha_0}$.)
\label{theorem35}
\end{theorem35}
%
%
%
{\bf Proof.}  We treat first the case $\alpha=\alpha_0$.  
%
Consider the homomorphism $\tau$ 
provided by Theorem (\ref{theorem33}).  
%
Since $\tau$ maps into $\R$ and $\R$ is a 
divisible group, then $\tau$ can 
be extended to a homomorphism on all of 
$\Gamma$ (\cite[Theorem (A.7)]{hr1}.  
We denote the extended homomorphism by 
$L_{\alpha_0}$.  
%
By the properties of $\tau$ from Theorem (\ref{theorem33}), 
it is clear that (iii) and (iv) hold.  
%
Now since $L_{\alpha_0}$ is continuous on an open 
subgroup of $\Gamma$, it follows 
by linearity that $L_{\alpha_0}$ 
is continuous on all $\Gamma$.  
%
This proves the theorem in this case.  
%
We now treat the remaining cases.  
%
Since $\R^a\times H$ is convex and open, 
the group $\Gamma/(\R^a\times H)$ 
is discrete and can be ordered as in 
Theorem (\ref{theorem1}).  Let $\Phi$ denote 
the natural homomorphism of $\Gamma$ onto 
$\Gamma/(\R^a\times H)$.  
%
It is easy to see that $C$ is 
a convex subgroup of $\Gamma$ 
if and only if $\Phi (C)$ is a convex 
subgroup of $\Gamma/(\R^a\times H)$.  
%
Consequently, if $D_\alpha\prec C_\alpha$ 
is a jump in $\Gamma$, with $\alpha\neq \alpha_0$, 
then 
$\Phi(D_\alpha)\prec \Phi(C_\alpha)$
is a jump in $\Gamma/(\R^a\times H)$.  
%
The theorem follows now by composing $\Phi$ with
the homomorphisms provided by Theorem (\ref{theorem3})
for the discrete ordered group $\Gamma/(\R^a\times H)$.\\
%
With Theorem (\ref{theorem35}) in hand we can 
give a simple proof of a separation 
theorem for measurable orders \cite[Theorem (5.14)]{ah}.  
The statement here slightly improves on \cite{ah}.  
%
%%%%%%%%%%%%%%%%%%%%%%%%%%%%%%%%%%%%%
%%%%%%%%%%%%%%%%%%%%%%%%%%%%%%%%%%%%%
%THEOREM36%%SEPARATION-THEOREM%%%%%%%%%%%
%%%%%%%%%%%%%%%%%%%%%%%%%%%%%%%%%%%%
\begin{theorem36}
Let $P$ be a measurable order on $\Gamma$ and let 
$K$ be an arbitrary compact subset of $\Gamma$.  
%
Let $N=\emptyset$ if $\Gamma$ is discrete and 
$N=D_{\alpha_0}$ if $\Gamma$ is not discrete, 
where $D_{\alpha_0}$ is as in Theorem (\ref{theorem35}). 
%
Then there is a continuous real-valued 
homomorphism $\psi$ of $\Gamma$ such that\\
%
%
%
$$\sgn_P(\chi)=\sgn(\psi(\chi))$$
%
for all $\chi\in K\setminus D_{\alpha_0}$.
\label{theorem36}
\end{theorem36}
%
{\bf Proof.}  We treat the discrete case first.  
%
Without 
loss of generality, we may assume that $K$
is a finite subset of $\Gamma$ not containing $0$.  
%
We appeal to Theorem (\ref{theorem3}) 
and use its notation.  
%
Let 
%
%
$$D_{\alpha_1}\subset 
C_{\alpha_1}\subset D_{\alpha_2}
\subset C_{\alpha_2}\subset\ldots 
D_{\alpha_n}\subset C_{\alpha_n}$$
%
%
be a finite collection in $\Sigma$ such 
that 
$K\cap (C_{\alpha_j}\backslash D_{\alpha_j})\neq \emptyset$ 
for all $j=1,2,\ldots , n$, 
and 
%
%
$$K\subset \bigcup_{j=1}^n\left(C_{\alpha_j}
\backslash D_{\alpha_j}\right),$$
%
%
and let $L_{\alpha_j}$ be the real-valued 
homomorphism of $\Gamma$ corresponding to 
$\alpha_j$.  
We have from Theorem (\ref{theorem3})
%
%
\begin{equation}
L_{\alpha_j}(D_{\alpha_j})=\{0\};
\label{*}
\end{equation} 
%
%
\begin{equation}
L_{\alpha_j}\left(
			K\cap P \cap (C_{\alpha_j}\backslash D_{\alpha_j})\right)\subset ]0,\infty [;
\label{**}
\end{equation}
and 
\begin{equation}
L_{\alpha_j}\left(
			K\cap (-P) \cap (C_{\alpha_j}\backslash D_{\alpha_j})\right)\subset ]- \infty , 0[.
\label{***}
\end{equation}
%
%
We construct the homomorphism $\psi$ as a linear combination 
%
%
$$\psi=\sum_{j=1}^n a_j L_{\alpha_j}$$
%
%
where the coefficients $a_j$ are 
defined inductively as follows.  
Set $a_1=1$.  If $a_j$ is defined for 
$j=1, \ldots , k-1$, 
let
%
%
$$A_k = \max_{x\in K}
\sum_{j=1}^{k-1} | a_j L_{\alpha_j}(x)|,$$	
%
%
$$B_k = \min_{x\in K \cap (C_{\alpha_k}\backslash D_{\alpha_k})} |L_{\alpha_k}(x)|.$$	
%
%
Note that $B_k$ is positive.  
%
Choose $a_k$ so that $a_k B_k >A_k$.  
%
Using (\ref{*})--(\ref{***}), 
it is straightforward to check that 
the homomorphism $\psi$ has the desired property.\\
To treat the general case, 
we appeal to Theorem (\ref{theorem35}), 
and borrow its notation.  Let $\Phi$ denote 
the quotient homomorphism of $\Gamma$ 
onto the discrete group $\Gamma/ (\R^a\times H)$
(recall that $\R^a\times H$ is open).  
%
Since $K$ is compact, $\Phi (K)$
is a finite subset of  $\Gamma/ (\R^a\times H)$.
%
Order $\Gamma/ (\R^a\times H)$ as 
in Theorem (\ref{theorem1}).  
%
By the case we just treated, 
we can find a homomorphism $\psi^*$ of 
$\Gamma/ (\R^a\times H)$ separating 
the set $\Phi (K)$.  
%
It is clear that the 
homomorphism $\psi^*\circ\Phi$ 
separates the set $K\backslash (\R^a\times H)$.  
%
If $K\cap ((\R^a\times H)\backslash 
D_{\alpha_0})=\emptyset$ then we are done.  
%
If not, consider the homomorphism 
%
$$\psi = L_{\alpha_0}  + \frac{a}{b} \psi^*\circ\Phi  $$
%
%
where $a=\max_{x\in K}|L_{\alpha_0} (x)|$ and 
$b=\min_{x\in K\backslash \R^a\times H}|\psi^*\circ\Phi (x) |.$
%
%
A simple argument that we omit shows that 
$\psi$ has the desired property.





%%%%%%%%%%%%%%%%%%%%%%%%%%%%%%%%%%%%%%%%%%%%%%%%%%%%%%%%%%%%%%%%%%%%%%%%%%%%%%%%%%%%%%%%%%%%%%%%%%%%%%%%%%%%%%%%%%%%%%%%%%%%%%%%%%%%%%%%%%%%%%%%%%%%%%%%%%%%%%%%%%%%%%%%%%%%%%%%%%%%%%%%%%%%%%%%%%%%%%%%%%%%%%%%%%%%%%%%%%%%%%%%%%%%%%%%%%%%%%%%%%%%%%%%%%%%%%%%%%%%%%%%%%%%%%%%%%%%%%%%%%%%%%%%%%%%%%%%%%%%%%%%%%%%%%%%%%%%%%%%%%%%%%%%%%%%%%%%%%%%%%%%%%%%%%%%%%%%%%%%%%%%%%%%%%%%%%%%%%%%%%%%%%%%%%%%%%%%%%%%%%%%%%%%%%%%%%%%%%%%%%%%%%%%%%%%%%%%%%%%%%%%%%%%%%%%%%%%%%%%%%
%%%%%%%%%%%%%%%%%%%%%%%%%%%%%%%%%%%%%%%%%%%%%%%%%%%%%%%%%%%%%%%%%%%%%%%%%%%%%

\section{The conjugate function and its basic properties}

\newtheorem{ergodic-hilbert-transform}{Theorem}[section]
\newtheorem{generalized-m-riesz}[ergodic-hilbert-transform]{Theorem}
\newtheorem{generalized-kolmogorov}[ergodic-hilbert-transform]{Theorem}
\newtheorem{conjugate-of-slice}[ergodic-hilbert-transform]{Theorem}
%%%%%%%%%%%%%%%%%%%%%%%%%%%%%%%%%%%%%%%%%%%%%%%%%%%%%%%%%%%%%%%%%%%%%%%%%%%%%%%%%%%%%%%%%%%%%%%%%%%%%%%%%%%%%%%%%%%%%%%%%%%%%%%%%%%%%%%%%%%%%%
%%%%%%%%%%%%%%%%%%%%%%%%%%%%%%%%%%%%%%%%%%%%%%%%%%%%%%%%%%%%%%%%%%%%%%%%%%%%%
In this section, we use the structure of orders
that we derived to define the 
conjugate series of a function in 
$L^p(G), \ 1\leq p <\infty$.
For use in the following section, we also recall some basic properties of the 
conjugate function operator, such as generalized versions
 of M. Riesz's and Kolmogorov's Theorems.

 Let $\Gamma$ denote a locally
compact abelian group containing a measurable 
order $P$, and let $G$ denote the 
dual group of $\Gamma$.  
Recall that the conjugate function operator
$f\mapsto \widetilde{f}$ is defined on 
$L^2(G)$ by the multiplier relation 
(\ref{conjugatefunction}).  It is convenient to 
write $\H (f)$ as  an alternative notation for the
conjugate function. We appeal to Theorems (\ref{theorem3}) 
and (\ref{theorem35}) and write the group $\Gamma$ as a 
disjoint union of open sets
%%%%%%%%%%%%%%%%%%%%%%%%%%%%%%%%%%%
%%%%%%%%%%%%%%%%%%%%%%%%%%%%%%%%%%%%%%%%%
\begin{equation}
\Gamma=C_{\alpha_0}\cup \bigcup_{\alpha\neq \alpha_0}
 C_\alpha\setminus D_\alpha.
\label{decomposition-of-gamma}
\end{equation}
%%%%%%%%%%%%%%%%%%%%%%%%%%%%%%%%%%%
%%%%%%%%%%%%%%%%%%%%%%%%%%%%%%%%%%%%%%%%%
For each $\alpha \neq \alpha_0$,
let $L_\alpha$ denote the continuous homomorphism
from $\Gamma$ into $\R$ such that 
%%%%%%%%%%%%%%%%%%%%%%%%%%%%%%%%%%%
%%%%%%%%%%%%%%%%%%%%%%%%%%%%%%%%%%%%%%%%%
\begin{equation}
\sgn (L_\alpha (\chi))=\sgn_P(\chi)
\label{separating-L}
\end{equation}
%%%%%%%%%%%%%%%%%%%%%%%%%%%%%%%%%%%
%%%%%%%%%%%%%%%%%%%%%%%%%%%%%%%%%%%%%%%%%
for all $\chi \in C_\alpha\setminus D_\alpha$, and let
$L_{\alpha_0}$ denote the continuous homomorphism
from $\Gamma$ into $\R$ such that 
%%%%%%%%%%%%%%%%%%%%%%%%%%%%%%%%%%%
%%%%%%%%%%%%%%%%%%%%%%%%%%%%%%%%%%%%%%%%%
\begin{equation}
\sgn (L_{\alpha_0} (\chi))=\sgn_P(\chi)
\label{separating-L0}
\end{equation}
%%%%%%%%%%%%%%%%%%%%%%%%%%%%%%%%%%%
%%%%%%%%%%%%%%%%%%%%%%%%%%%%%%%%%%%%%%%%%
for locally almost all $\chi \in C_{\alpha_0}$.
To simplify notation, let us write 
$\Pi$ for the indexing set in both
cases of Theorems (\ref{theorem3}) and (\ref{theorem35}).
%%%%%%%%%%%%%%%%%%%%%%%%%%%%%%%%%%%
%%%%%%%%%%%%%%%%%%%%%%%%%%%%%%%%%%%%%%%%%
For each $\alpha\in \Pi$, the subgroup $C_\alpha$
is open, and similarly, $D_\alpha$ is
 open for $\alpha\neq \alpha_0$.  Hence the annihilators
in $G$,
$A(G,C_\alpha)$ and $A(G,D_\alpha)$,
are compact.  Let $\mu_\alpha$, respectively,
$\nu_\alpha$, denote the normalized Haar measure on
$A(G,C_\alpha)$, respectively, $A(G,D_\alpha)$.
We have
%%%%%%%%%%%%%%%%%%%%%%%%%%%%%%%%%%%
%%%%%%%%%%%%%%%%%%%%%%%%%%%%%%%%%%%%%%%%%
\begin{equation}
\widehat{\mu_\alpha}=1_{C_\alpha}\ \ {\rm and}\ \ 
\widehat{\nu_\alpha}=1_{D_\alpha}
\label{fourier-transform-of-mu-and-nu}
\end{equation}
%%%%%%%%%%%%%%%%%%%%%%%%%%%%%%%%%%%
%%%%%%%%%%%%%%%%%%%%%%%%%%%%%%%%%%%%%%%%%
where, if $A$ is a set, $1_A$ is the indicator of $A$.
For $f\in L^p(G)$, $1\leq p<\infty$, we have
$$\|f*\mu_\alpha\|_p\leq \|f\|_p,\ \ {\rm and}\ \ 
\|f*\nu_\alpha\|_p\leq \|f\|_p,$$
(\cite[Theorem (20.12)]{hr1}).
For $\alpha\neq\alpha_0$, we let
%%%%%%%%%%%%%%%%%%%%%%%%%%%%%%%%%%%
%%%%%%%%%%%%%%%%%%%%%%%%%%%%%%%%%%%%%%%%%
$$d_\alpha f=f*\mu_\alpha- f*\nu_\alpha,$$
%%%%%%%%%%%%%%%%%%%%%%%%%%%%%%%%%%%
%%%%%%%%%%%%%%%%%%%%%%%%%%%%%%%%%%%%%%%%%
and
$$d_{\alpha_0} f =f*\mu_{\alpha_0}.$$
%%%%%%%%%%%%%%%%%%%%%%%%%%%%%%%%%%%
%%%%%%%%%%%%%%%%%%%%%%%%%%%%%%%%%%%%%%%%%
It is clear that if $f\in L^2(G)$, then 
the support of $\widehat{f}$ is $\sigma-$compact.
Hence it has a nonvoid intersection with only 
countably many of the sets appearing on the right 
side of 
(\ref{decomposition-of-gamma}).  It follows from 
(\ref{fourier-transform-of-mu-and-nu}) that,
except for countably many $\alpha$'s,
$d_\alpha f$ is zero almost everywhere.
By approximating with functions in $L^2(G)$,
we see that the same is true for any $f\in L^p(G)$,
$1\leq p<\infty$.
As a convention, when $d_\alpha f=0$ a.e., we 
take it to be identically 0.  With this convention,
the formal difference series 
%%%%%%%%%%%%%%%%%%%%%%%%%%%%%%%%%%
%difference-series%%%%%%%%%%%
%%%%%%%%%%%%%%%%%%%%%%%%%%%%%%%%%%
%%%%%%%%%%%%%%%%%%%%%%%%%%%%%%%%%%
\begin{equation}
\sum_{\alpha\in \Pi} d_\alpha f
\label{difference-series}
\end{equation} 
%%%%%%%%%%%%%%%%%%%%%%%%%%%%%%%%%%
%%%%%%%%%%%%%%%%%%%%%%%%%%%%%%%%%%
has only countably many 
nonzero terms.
The conjugate function will be defined by a series
conjugate to (\ref{difference-series}).  
Central to our construction is the ergodic
Hilbert transform which we introduce next.
This transform has been systematically studied
by Cotlar \cite{cot}, Calder\'on \cite{cal},
and Coifman and Weiss \cite{cw}. 

Let $L_\alpha$ be as in (\ref{separating-L0}) or 
(\ref{separating-L}), and let $\phi_\alpha$ denote its
adjoint homomorphism.  Thus $\phi_\alpha$ is a continuous
homomorphism mapping $\R$ into $G$ and satisfying
%%%%%%%%%%%%%%%%%%%%%%%%%%%%%%%%%%
%%%%%%%%%%%%%%%%%%%%%%%%%%%%%%%%%%
\begin{equation}
\chi \circ \phi_\alpha (r)=L_\alpha (\chi)(r)
\label{adjoint-map}
\end{equation}
%%%%%%%%%%%%%%%%%%%%%%%%%%%%%%%%%%
%%%%%%%%%%%%%%%%%%%%%%%%%%%%%%%%%%
for all $r\in \R$ and all $\chi \in \Gamma$ 
(\cite[Section 24]{hr1}).  The truncated Hilbert 
transform in the direction of $L_\alpha$ is the
operator defined on $L^p(G)$, $1\leq p<\infty$, by
%%%%%%%%%%%%%%%%%%%%%%%%%%%%%%%%%%
%%%%%%%%%%%%%%%%%%%%%%%%%%%%%%%%%%
\begin{equation}
H_{L_\alpha ,n}f (x)=\frac{1}{\pi} \int_{\frac{1}{n}\leq |t|\leq n}
			f(x-\phi_\alpha (t))\frac{1}{t}dt.
\label{truncated-hilbert-transform}
\end{equation}
%%%%%%%%%%%%%%%%%%%%%%%%%%%%%%%%%%
%%%%%%%%%%%%%%%%%%%%%%%%%%%%%%%%%%
%
The (ergodic) Hilbert transform in the direction of $L_\alpha$ 
is the operator defined on 
$L^p(G)$, $1\leq p<\infty$,  by
%%%%%%%%%%%%%%%%%%%%%%%%%%%%%%%%%%
%%%%%%%%%%%%%%%%%%%%%%%%%%%%%%%%%%
\begin{equation}
H_{L_\alpha}f(x) = \lim_{n\rightarrow\infty} H_{L_\alpha ,n}f (x).
\label{hilbert-transform}
\end{equation}
%%%%%%%%%%%%%%%%%%%%%%%%%%%%%%%%%%
%%%%%%%%%%%%%%%%%%%%%%%%%%%%%%%%%%
The fact that this limit exits $\mu-$a.e. on $G$ follows from 
\cite{cot} (see also \cite{cal} or
\cite{cw}).  In fact, several other
properties of this transform follow from those
of the Hilbert transform on $\R$ and the transference
methods of \cite{cal} and
\cite{cw}.  For ease of reference, we state some 
properties that are needed in the sequel.  
We let $L$ denote
an arbitrary continuous nonzero 
homomorphism from $\Gamma$ into $\R$, and
let $\phi$ denote its adjoint homomorphism.
The operator $H_{L,n}$ is defined as in 
(\ref{truncated-hilbert-transform}).
%%%%%%%%%%%%%%%%%%%%%%%%%%%%%%%%%%%%%%
%%%%%%%%%%%%%%%%%%%%%%%%%%%%%%%%%%%%%%%
%%%%%ERGODIC_HILBERT_TRANSFORM%%%%%%%%%%%
%%%%%%%%%%%%%%%%%%%%%%%%%%%%%%%%%%%%%%
%%%%%%%%%%%%%%%%%%%%%%%%%%%%%%%%%%%%%%%
\begin{ergodic-hilbert-transform}
Let $f\in L^p(G)$, where $1\leq p<\infty$. \\
(i)  The limit
$$H_L f(x) = \lim_{n\rightarrow\infty} H_{L,n}f (x)$$
exists $\mu-$a.e.\\
(ii)  If $1<p<\infty$, then the limit converges in 
$L^p(G)$, and 
$$\|H_L f\|_p\leq A_p \|f\|_p$$
where $A_p$ is the bound of the Hilbert transform
operator on $L^p(\R)$.\\
(iii)  For $f\in L^1(G)$, we have
$$\mu\left(\{x\in G :\ \ | H_L f(x)|>y\}\right)\leq
		\frac{A}{y} \|f\|_1$$
for all $y>0$, where $A$ is the weak type $(1,1)$
norm of the Hilbert transform on $L^1(\R)$.\\
(iv)  For $f\in L^2(G)$, we have
$$\widehat{H_L f}(\chi)=- i \sgn (L (\chi))\widehat{f}(\chi)$$
for almost all $\chi\in \Gamma$.   
\label{ergodic-hilbert-transform}
\end{ergodic-hilbert-transform}
%%%%%%%%%%%%%%%%%%%%%%%%%%%%%%%%%%%%%%
%%%%%ERGODIC_HILBERT_TRANSFORM%%%%%%%%%%%%%%%%%
%%%%%%%%%%%%%%%%%%%%%%%%%%%%%%%%%%%%%%
%%%%%%%%%%%%%%%%%%%%%%%%%%%%%%%%%%%%%%%
The usefulness of this theorem
is due in great part to the fact that all the
estimates are independent of $L$ or $G$.
Property (iv) justifies using the terminology
"the Hilbert transform in the direction of 
$L$" and shows a clear 
connection between the ergodic Hilbert transform 
and the conjugate function on groups.  
The proof of (iv) is straightforward, 
using (ii) and (\ref{adjoint-map}).  (See
\cite[Theorem (6.7)]{ah}.) For use in the sequel,
we recall the generalizations of M. Riesz's
Theorem and Kolmogorov's Theorem from \cite{ah}.
(These results are due to Helson \cite{hel1} and \cite{hel2},
when $G$ is compact.)  
Also, having all the necessary ingredients to prove
these results, we will sketch short proofs
to make the paper more self contained
and to illustrate the use of the separation theorems.
%%%%%%%%%%%%%%%%%%%%%%%%%%%%%%%%%%%%%%
%%%%%%%%%%%%%%%%%%%%%%%%%%%%%%%%%%%%%%%
%generalized-m-riesz%%%%%%%%%%%%%%%%%%%%
%%%%%%%%%%%%%%%%%%%%%%%%%%%%%%%%%%%%%%%
\begin{generalized-m-riesz}
%%%%%%%%%%%%%%%%%%%%%%%%%%%%%%%%%%%%%%
Let $G$ be a locally compact abelian group with 
dual group $\Gamma$, and let $P$ denote an arbitrary
measurable order on $\Gamma$. For all
$f\in L^p(G)$, $1<p<\infty$, we have
$$\|\H f\|_p\leq A_p \| f \|_p   $$
where $A_p$ is the norm of the Hilbert 
transform on $L^p(\R)$.
%%%%%%%%%%%%%%%%%%%%%%%%%%%%%%%%%%%%%%
\label{generalized-m-riesz}
\end{generalized-m-riesz} 
%%%%%%%%%%%%%%%%%%%%%%%%%%%%%%%%%%%%%%
%%%%%%%%%%%%%%%%%%%%%%%%%%%%%%%%%%%%%%%
%%%%%%%%%%%%%%%%%%%%%%%%%%%%%%%%%%%%%%
%%%%%%%%%%%%%%%%%%%%%%%%%%%%%%%%%%%%%%%




%%%%%%%%%%%%%%%%%%%%%%%%%%%%%%%%%%%%%%
%%%%%%%%%%%%%%%%%%%%%%%%%%%%%%%%%%%%%%%
%generalized-kolmogorov%%%%%%%%%%%%%%%%
%%%%%%%%%%%%%%%%%%%%%%%%%%%%%%%%%%%%%%%
\begin{generalized-kolmogorov}
%%%%%%%%%%%%%%%%%%%%%%%%%%%%%%%%%%%%%%%
Let $G$ be a locally compact abelian group with 
dual group $\Gamma$, and let $P$ denote an arbitrary
measurable order on $\Gamma$. For all
$f \in L^2 \cap L^1(G)$ 
and all $y>0$, we have
$$\mu\left(\{x\in G :\ \ | \H f(x)|>y\}\right)\leq
		\frac{A}{y} \|f\|_1$$
where $A$ is the weak type $(1,1)$ norm of the 
Hilbert transform on $L^1(\R)$.
%%%%%%%%%%%%%%%%%%%%%%%%%%%%%%%%%%%%%%%
\label{generalized-kolmogorov}
\end{generalized-kolmogorov}
%%%%%%%%%%%%%%%%%%%%%%%%%%%%%%%%%%%%%%
%%%%%%%%%%%%%%%%%%%%%%%%%%%%%%%%%%%%%%%
%%%%%%%%%%%%%%%%%%%%%%%%%%%%%%%%%%%%%%
%%%%%%%%%%%%%%%%%%%%%%%%%%%%%%%%%%%%%%%

Both theorems are proved in a similar way.
It is enough to consider $f\in L^2(G)$
with compactly supported Fourier transform.
Let $K\subset\Gamma$ denote the compact support of 
$\widehat{f}$.  Apply Theorem (\ref{theorem36})
and obtain a real-valued homomorphism $L$
of $\Gamma$ such that 
$$\sgn_P(\chi)=\sgn(L(\chi))$$
for almost all $\chi \in K$.  Thus, 
from Theorem (\ref{ergodic-hilbert-transform})(iv)
and the fact that $\widehat{f}$ is supported in
$K$, it follows from the uniqueness 
of the Fourier transform that
$$\H f=H_L f$$
a.e. on $G$.  The inequalities in
Theorems (\ref{generalized-m-riesz})
 and (\ref{generalized-kolmogorov}) 
follow now from the corresponding ones 
for $H_L$ in
Theorem (\ref{ergodic-hilbert-transform}).\\
%%%%%%%%%%%%%%%%%%%%%%%%%%%%%%%%%%%%%%%%%%%
%%%%%%%%%%%%%%%%%%%%%%%%%%%%%%%%%%%%%%%%%%%
Because of Theorem (\ref{generalized-kolmogorov}),
the operator $\H$ extends from $L^2 \cap L^1 (G)$
to an operator on $L^1(G)$ satisfying the 
same weak type
$(1,1)$ estimate.  We keep the same notation for the 
extended operator.

%%%%%%%%%%%%%%%%%%%%%%%%%%%%%%%%%%%%%%%%%%%
%%%%%%%%%%%%%%%%%%%%%%%%%%%%%%%%%%%%%%%%%%%
The next theorem is our first step toward
building the conjugate function.  We continue 
with the notation leading to (\ref{difference-series}).
%%%%%%%%%%%%%%%%%%%%%%%%%%%%%%%%%%%%%%
%%%%%%%%%%%%%%%%%%%%%%%%%%%%%%%%%%%%%%%
%conjugate-of-slice%%%%%%%%%%%%%%%%
%%%%%%%%%%%%%%%%%%%%%%%%%%%%%%%%%%%%%%%
\begin{conjugate-of-slice}
%%%%%%%%%%%%%%%%%%%%%%%%%%%%%%%%%%%%%%%
Let $f\in L^p(G)$ where $1\leq p<\infty$,
and let $\alpha\in \Pi$.  Then\\
(i)\ \ $\H (d_\alpha f)= H_{L_\alpha} (d_\alpha f)$
$\mu-$a.e.\\
If $f\in L^2\cap L^p(G)$, then we also have\\
(ii)\ \ $\H (d_\alpha f)= d_\alpha (\H f)$ and
$H_{L_\alpha} (d_\alpha f)= d_\alpha (H_{L_\alpha} f)$\ $\mu-$a.e.
%%%%%%%%%%%%%%%%%%%%%%%%%%%%%%%%%%%%%
%%%%%%%%%%%%%%%%%%%%%%%%%%%%%%%%%%%%%%%
%conjugate-of-slice%%%%%%%%%%%%%%%%
%%%%%%%%%%%%%%%%%%%%%%%%%%%%%%%%%%%%%%%
\label{conjugate-of-slice}
\end{conjugate-of-slice}
%%%%%%%%%%%%%%%%%%%%%%%%%%%%%%%%%%%%%%%
{\bf Proof.}  The equalities in  (ii) 
are clear since all operators in question are
multiplier operators and so they commute.  
To prove (i) we note that
since $d_\alpha$ is a bounded operator from
$L^1(G)$ into $L^1(G)$, and since 
$\H$ and $H_{L_\alpha}$ are bounded from 
$L^1(G)$ into $L^{1,\infty}(G)$, it is enough
to consider $f\in L^2(G)$.  Since $\sgn_P$ and
$\sgn (L_\alpha(\cdot))$ agree a.e. on 
$C_\alpha\setminus D_\alpha$, and since 
$d_\alpha$ projects the Fourier transform
on $C_\alpha\setminus D_\alpha$, it 
is easy to see that the Fourier 
transforms of $\H (d_\alpha f)$ and
$ H_{L_\alpha} (d_\alpha f)$ agree almost everywhere
on $\Gamma$,
and so (i) follows.\\
%%%%%%%%%%%%%%%%%%%%%%%%%%%%%%%%%%%%%
%%%%%%%%%%%%%%%%%%%%%%%%%%%%%%%%%%%%%%%
As we argued for (\ref{difference-series}),
we will agree that, for $f\in L^p(G)$
$(1\leq p<\infty)$, the formal
series
%%%%%%%%%%%%%%%%%%%%%%%%%%%%%%%%%%
%conjugate-series%%%%%%%%%%%
%%%%%%%%%%%%%%%%%%%%%%%%%%%%%%%%%%
%%%%%%%%%%%%%%%%%%%%%%%%%%%%%%%%%%
\begin{equation}
\sum_{\alpha\in \Pi} H_{L_\alpha}(d_\alpha f)
\label{conjugate-series}
\end{equation} 
%%%%%%%%%%%%%%%%%%%%%%%%%%%%%%%%%%
%%%%%%%%%%%%%%%%%%%%%%%%%%%%%%%%%%
has only countably many terms.
We will refer to (\ref{conjugate-series}) as 
the conjugate (difference) series of $f$.\\
%%%%%%%%%%%%%%%%%%%%%%%%%%%%%%%%%%
%%%%%%%%%%%%%%%%%%%%%%%%%%%%%%%%%%
%%%%%%%%%%%%%%%%%%%%%%%%%%%%%%%%%%
%%%%%%%%%%%%%%%%%%%%%%%%%%%%%%%%%%
%%%%%%%%%%%%%%%%%%%%%%%%%%%%%%%%%%
%%%%%%%%%%%%%%%%%%%%%%%%%%%%%%%%%%
%%%%%%%%%%%%%%%%%%%%%%%%%%%%%%%%%%
%%%%%%%%%%%%%%%%%%%%%%%%%%%%%%%%%%
%%%%%%%%%%%%%%%%%%%%%%%%%%%%%%%%%%
%%%%%%%%%%%%%%%%%%%%%%%%%%%%%%%%%%%%%%%%%%%%%%%%%%%%%%%%%%%%%%%%%%%%
%%%%%%%%%%%%%%%%%%%%%%%%%%%%%%%%%%
%%%%%%%%%%%%%%%%%%%%%%%%%%%%%%%%%%
%%%%%%%%%%%%%%%%%%%%%%%%%%%%%%%%%%
%%%%%%%%%%%%%%%%%%%%%%%%%%%%%%%%%%%%%%%%%%%%%%%%%%%%%%%%%%%%%%%%%%%%
%%%%%%%%%%%%%%%%%%%%%%%%%%%%%%%%%%
%%%%%%%%%%%%%%%%%%%%%%%%%%%%%%%%%%
%%%%%%%%%%%%%%%%%%%%%%%%%%%%%%%%%%
%%%%%%%%%%%%%%%%%%%%%%%%%%%%%%%%%%%%%%%%%%%%%%%%%%%%%%%%%%%%%%%%%%%%
%%%%%%%%%%%%%%%%%%%%%%%%%%%%%%%%%%
%%%%%%%%%%%%%%%%%%%%%%%%%%%%%%%%%%
%%%%%%%%%%%%%%%%%%%%%%%%%%%%%%%%%%
%%%%%%%%%%%%%%%%%%%%%%%%%%%%%%%%%%%%%%%%%%%%%%%%%%%%%%%%%%%%%%%%%%%%
%%%%%%%%%%%%%%%%%%%%%%%%%%%%%%%%%%
%%%%%%%%%%%%%%%%%%%%%%%%%%%%%%%%%%
%%%%%%%%%%%%%%%%%%%%%%%%%%%%%%%%%%
%%%%%%%%%%%%%%%%%%%%%%%%%%%%%%%%%%
\section{Unconditional convergence of conjugate difference series}
\newtheorem{conjugate-projection}{Theorem}[section]
\newtheorem{finite-conjugate-projection}[conjugate-projection]{Corollary}
\newtheorem{unconditional-convergence}[conjugate-projection]{Theorem}
\newtheorem{partial-sum}[conjugate-projection]{Corollary}
\newtheorem{theorem-square-function}[conjugate-projection]{Theorem}
%%%%%%%%%%%%%%%%%%%%%%%%%%%%%%%%%%
%%%%%%%%%%%%%%%%%%%%%%%%%%%%%%%%%%
%%%%%%%%%%%%%%%%%%%%%%%%%%%%%%%%%%
%%%%%%%%%%%%%%%%%%%%%%%%%%%%%%%%%%
%%%%%%%%%%%%%%%%%%%%%%%%%%%%%%%%%%%%%%%%%%%%%%%%%%%%%%%%%%%%%%%%%%%%
%%%%%%%%%%%%%%%%%%%%%%%%%%%%%%%%%%
%%%%%%%%%%%%%%%%%%%%%%%%%%%%%%%%%%
%%%%%%%%%%%%%%%%%%%%%%%%%%%%%%%%%%
%%%%%%%%%%%%%%%%%%%%%%%%%%%%%%%%%%%%%%%%%%%%%%%%%%%%%%%%%%%%%%%%%%%%
%%%%%%%%%%%%%%%%%%%%%%%%%%%%%%%%%%
%%%%%%%%%%%%%%%%%%%%%%%%%%%%%%%%%%
%%%%%%%%%%%%%%%%%%%%%%%%%%%%%%%%%%
%%%%%%%%%%%%%%%%%%%%%%%%%%%%%%%%%%
%%%%%%%%%%%%%%%%%%%%%%%%%%%%%%%%%%
%%%%%%%%%%%%%%%%%%%%%%%%%%%%%%%%%%
%%%%%%%%%%%%%%%%%%%%%%%%%%%%%%%%%%
We will show that the conjugate series 
(\ref{conjugate-series})
converges unconditionally in $L^p(G)$
when $1<p<\infty$ and unconditionally in $L^{1,\infty}(G)$
when $p=1$.  This will further justify our notation
in (\ref{conjugate-series})
since the order of summation will become irrelevant
in (\ref{conjugate-series}).\\
%%%%%%%%%%%%%%%%%%%%%%%%%%%%%%%%%%
%LORENTZ SPACES%%%%%%%%%%%%%%%%%
%%%%%%%%%%%%%%%%%%%%%%%%%%%%%%%%%%
For use with weak type estimates, we
recall a few facts about the Lorentz spaces
$L^{p,\infty}(G)$.  All details can be found
in \cite[Chapter V, Section 3]{sw}.  Although
the presentation in the cited reference is confined
to $\sigma-$finite measure spaces,
the results that we need on locally
compact abelian groups follow easily
by restricting a given function
to its $\sigma-$compact support.\\
Given a measurable function $f$ on $G$,
let $\lambda_f$ denote its
distribution function, and let $f^*$ denote the
decreasing rearrangement of $f$.  Define
%%%%%%%%%%%%%%%%%%%%%%%%%%%%%%%%%%
%%%%%%%%%%%%%%%%%%%%%%%%%%%%%%%%%%
\begin{equation}
\| f\|^*_{p,\infty}=\sup_{y>0}
y\left( \lambda_f (y)\right)^\frac{1}{p}=
\sup_{y>0}y^{1/p} f^*(y),
\label{lorentz-pseudo-norm}
\end{equation} 
%%%%%%%%%%%%%%%%%%%%%%%%%%%%%%%%%%
and
%%%%%%%%%%%%%%%%%%%%%%%%%%%%%%%%%%
\begin{equation}
\| f\|_{p,\infty}=\sup_{y>0}
y^\frac{1}{p}  m_f (y)
\label{lorentz-norm}
\end{equation} 
%%%%%%%%%%%%%%%%%%%%%%%%%%%%%%%%%%
%%%%%%%%%%%%%%%%%%%%%%%%%%%%%%%%%%
where 
%%%%%%%%%%%%%%%%%%%%%%%%%%%%%%%%%%
%%%%%%%%%%%%%%%%%%%%%%%%%%%%%%%%%%
$$ m_f(y)=\frac{1}{y} \int_0^y f^*(u)du.$$
%%%%%%%%%%%%%%%%%%%%%%%%%%%%%%%%%%
%%%%%%%%%%%%%%%%%%%%%%%%%%%%%%%%%%
Let $L^{p,\infty}(G)$ consist of 
all measurable functions on $G$ such that
$\| f\|^*_{p,\infty}<\infty$.
It is well-known that, when $1<p<\infty$,
(\ref{lorentz-pseudo-norm}) and  (\ref{lorentz-norm})
are equivalent and define a norm on
$L^{p,\infty}(G)$.  
In fact, $\| f\|_{p,\infty}$ is a norm
for all $1\leq p <\infty$, and when
$1<p<\infty$, we also have
%%%%%%%%%%%%%%%%%%%%%%%%%%%%%%%%%%
%%%%%%%%%%%%%%%%%%%%%%%%%%%%%%%%%%
\begin{equation}
\| f\|^*_{p,\infty} \leq  \| f\|_{p,\infty} 
\leq
\frac{p}{p-1} \| f\|^*_{p,\infty}.
\label{equivalent-lorentz}
\end{equation} 
%%%%%%%%%%%%%%%%%%%%%%%%%%%%%%%%%%
%%%%%%%%%%%%%%%%%%%%%%%%%%%%%%%%%%
(See \cite[Chap. V, Theorem 3.21]{sw}.)\\
%%%%%%%%%%%%%%%%%%%%%%%%%%%%%%%%%%
%%%%%%%%%%%%%%%%%%%%%%%%%%%%%%%%%%

%%%%%%%%%%%%%%%%%%%%%%%%%%%%%%%%%%
%%%%%%%%%%%%%%%%%%%%%%%%%%%%%%%%%%
%%%%%%%%%%%%%%%%%%%%%%%%%%%%%%%%%%
%%%%%%%%%%%%%%%%%%%%%%%%%%%%%%%%%%
Let 
$\e\in \{ - 1 , 1\}^\Pi$.
We will write $\e (P)$ for 
the subset of $\Gamma$ obtained from $P$
by changing the sign on $C_\alpha\setminus D_\alpha$
according to $\e(\alpha)$.  That is,
if $x\in C_\alpha\setminus D_\alpha$ and 
$\alpha\neq \alpha_0$, or,
if $x\in C_{\alpha_0}$, then
$x\in \e(P)$ if and only if $\e(\alpha) x\in P$.\\
It is easy to see that $\e(P)$ is an order 
on $\Gamma$.  \\
Suppose that $\eta\in \{ 0 , 1\}^\Pi$.
Define a projection operator 
${\cal P}_\eta$ 
on $L^2(G)$ by
$$\widehat{{\cal P}_\eta (f)}=
\widehat{f}1_{\bigcup_{\alpha\in \Pi, \ \eta(\alpha)=1}
C_\alpha\setminus D_\alpha}.$$
Define the conjugate projection operator
$\widetilde{{\cal P}_{\eta, P}}$
by 
%%%%%%%%%%%%%%%%%%%%%%%%%%%%%%%%%%
%%%%%%%%conjugate-projection%%%%%%%%%
%%%%%%%%%%%%%%%%%%%%%%%%%%%%%%%%%%
\begin{equation}
\widetilde{{\cal P}_{\eta, P}}f=
\H ({\cal P}_\eta f).
\label{conjugate-projection-operator}
\end{equation}
%%%%%%%%%%%%%%%%%%%%%%%%%%%%%%%%%%
%%%%%%%%%%%%%%%%%%%%%%%%%%%%%%%%%%
%%%%%%%%%%%%%%%%%%%%%%%%%%%%%%%%%%
%
Thus,
%%%%%%%%%%%%%%%%%%%%%%%%%%%%%%%%%%
%%%%%%%%%%%%%%%%%%%%%%%%%%%%%%%%%%
%%%%%%%%%%%%%%%%%%%%%%%%%%%%%%%%%%
\begin{equation}
\widehat{\widetilde{{\cal P}_{\eta, P}}f}(\chi) =
\left\{
\begin{array}{ll}
0 	& 	\mbox{if $\chi\not\in 
		\bigcup_{\alpha, \eta(\alpha) =1}
		C_\alpha\setminus D_\alpha$}\\
- i \sgn_P(\chi ) \widehat {f} (\chi) &	\mbox{otherwise}
\end{array}
\right.
\label{fourier-transform-of-conjugate-slice}
\end{equation}
%%%%%%%%%%%%%%%%%%%%%%%%%%%%%%%%%%
%%%%%%%%%%%%%%%%%%%%%%%%%%%%%%%%%%
%%%%%%%%%%%%%%%%%%%%%%%%%%%%%%%%%%
To establish the unconditional 
convergence of the conjugate series,
the following result is fundamental.  
It is a simple consequence of 
Theorems (\ref{generalized-m-riesz}) and
(\ref{generalized-kolmogorov}).
%%%%%%%%%%%%%%%%%%%%%%%%%%%%%%%%%%
%%%%%%%%%%%%%%%%%%%%%%%%%%%%%%%%%%
%%%%%%%%%%%%%%%%%%%%%%%%%%%%%%%%%%
\begin{conjugate-projection}
Let $\eta$ be any element of
$\{0,1\}^\Pi$, and let $P$ be an arbitrary order
on $\Gamma$.  
(i)  The operator 
$\widetilde{{\cal P}_{\eta, P}}$
is bounded on $L^p(G)$ for $1<p<\infty$ with
norm $\leq A_p$ where
$A_p$ is as in Theorem (\ref{generalized-m-riesz}).\\
(ii) The operator 
$\widetilde{{\cal P}_{\eta, P}}$
is of weak type $(1,1)$ on $L^2\cap L^1(G)$
with norm $\leq 2A$
where $A$ is the weak type norm in 
Theorem (\ref{generalized-kolmogorov}).
\label{conjugate-projection}
\end{conjugate-projection}  
%%%%%%%%%%%%%%%%%%%%%%%%%%%%%%%%%%
%%%%%%%%%%%%%%%%%%%%%%%%%%%%%%%%%%
%%%%%%%%%%%%%%%%%%%%%%%%%%%%%%%%%%
{\bf Proof.}  Define $\e \in \{-1,1\}^\Pi$ by 
$\e (\pi)=1 $ if $\eta (\pi) =1$ and 
$\e (\pi)=-1 $ if $\eta (\pi) =0$.  It is easy
to check using the Fourier transform that
for all $f\in L^2(G)$, we have
%%%%%%%%%%%%%%%%%%%%%%%%%%%%%%%%%%
%%%%%%%%%%%%%%%%%%%%%%%%%%%%%%%%%%
$$\widetilde{{\cal P}_{\eta, P}}f=
\frac{1}{2}({\cal H}_P f + {\cal H}_{\e(P)} f).$$
The theorem follows now from 
Theorems (\ref{generalized-m-riesz}) and 
(\ref{generalized-kolmogorov}) applied to the 
operators $\H$ and ${\cal H}_{\e(P)}$.\\

As a simple corollary, we have the following.
%%%%%%%%%%%%%%%%%%%%%%%%%%%%%%%%%%
%%%%%%%%%%%%%%%%%%%%%%%%%%%%%%%%%%
%%%%%%%%%%%%%%%%%%%%%%%%%%%%%%%%%%
%%%%%%%%%%%%%%%%%%%%%%%%%%%%%%%%%%
%%%%%%%%%%%%%%%%%%%%%%%%%%%%%%%%%%
%%%%%%%%%%%%%%%%%%%%%%%%%%%%%%%%%%
\begin{finite-conjugate-projection}
Let $\{\alpha_1,\alpha_2,\ldots,\alpha_n\}$
be a finite subset of $\Pi$.
Then, the operator
$$f\mapsto \sum_{j=1}^n H_{L_{\alpha_j}} (d_{\alpha_j} f)$$
is of weak type $(1,1)$ 
on $L^1(G)$ with norm $\leq 2 A$ and is bounded from
$L^p(G)$ into $L^p(G)$ with
norm $\leq A_p$, where $A$ and $A_p$
are as in Theorems (\ref{generalized-m-riesz}) and 
(\ref{generalized-kolmogorov}).
\label{finite-conjugate-projection}
\end{finite-conjugate-projection}
%%%%%%%%%%%%%%%%%%%%%%%%%%%%%%%%%%
%%%%%%%%%%%%%%%%%%%%%%%%%%%%%%%%%%
%%%%%%%%%%%%%%%%%%%%%%%%%%%%%%%%%%
{\bf Proof.}  Define $\eta\in \{0,1\}^\Pi$
by $\eta(\alpha_j)=1$ for $j=1,2,\ldots,n$ and 
and $\eta (\pi)=0$ otherwise.  Then
%%%%%%%%%%%%%%%%%%%%%%%%%%%%%%%%%%
\begin{equation}
\sum_{j=1}^n H_{L_{\alpha_j}} (d_{\alpha_j} f)=
\widetilde{{\cal P}_{\eta, P}}f.
\label{conjugate-eta-projection}
\end{equation}
%%%%%%%%%%%%%%%%%%%%%%%%%%%%%%%%%%
Now apply Theorem (\ref{conjugate-projection}).\\
%%%%%%%%%%%%%%%%%%%%%%%%%%%%%%%%%%
%%%%%%%%%%%%%%%%%%%%%%%%%%%%%%%%%%
%%%%%%%%%%%%%%%%%%%%%%%%%%%%%%%%%%
We are now ready to establish the unconditional
convergence of the conjugate series 
(\ref{conjugate-series}).
%%%%%%%%%%%%%%%%%%%%%%%%%%%%%%%%%%
%%%UNCONDITIONAL-CONVERGENCE%%%%%%%%%%%%%%
%%%%%%%%%%%%%%%%%%%%%%%%%%%%%%%%%%
\begin{unconditional-convergence}
Let $f\in L^p(G)$, $1\leq p<\infty$, and let
$\{\alpha_j\}\subset \Pi$ be an arbitrary enumeration
of the countable set of $\alpha\in \Pi$ such that
$d_\alpha f\not\equiv 0$.\\
(i)  If $p=1$, the series 
$\sum_j H_{L_{\alpha_j}}
d_{\alpha_j}f$ 
converges in $L^{1,\infty} (G)$ to $\H f$.\\
(ii) If $1<p<\infty$, the series 
$\sum_jH_{L_{\alpha_j}}
d_{\alpha_j}f$ 
converges in $L^p (G)$ to $\H f$.
\label{unconditional-convergence}
\end{unconditional-convergence}
%%%%%%%%%%%%%%%%%%%%%%%%%%%%%%%%%%
%%%%%%%%%%%%%%%%%%%%%%%%%%%%%%%%%%
%%%%%%%%%%%%%%%%%%%%%%%%%%%%%%%%%%
{\bf Proof.} 
We will
deal with the case $p=1$ only.
The other case is done similarly.
The assertions of the theorem
are clear if $\hat{f}$ is compactly supported,
since in this case only finitely many $d_\alpha f$
are nonzero.
Suppose that $f$ is an arbitrary function
in $L^1 (G)$, and   
 approximate $f$
in $L^1(G)$ by
functions with compactly supported Fourier transforms,
say $\{g_n\}$.  Then using 
Corollary (\ref{finite-conjugate-projection}) 
and Theorem (\ref{generalized-kolmogorov}),
we get
%%%%%%%%%%%%%%%%%%%%%%%%%%%%%%
%%%%%%%%%%%%%%%%%%%%%%%%%%%%%%%%%%%%%%
\begin{eqnarray*}
\| \sum_{j=1}^N H_{L_{\alpha_j}} d_{\alpha_j} f -\H f\|^*_{1,\infty}
& \leq &
2 \| \sum_{j=1}^N H_{L_{\alpha_j}} d_{\alpha_j} (f-g_n) -
\H (f-g_n)\|^*_{1,\infty} \\
& & + 
2 \| \sum_{j=1}^N H_{L_{\alpha_j}} d_{\alpha_j} g_n -\H g_n\|^*_{1,\infty}\\
& \leq&
12 A \| f-g_n\|^*_{1,\infty} +
2 \| \sum_{j=1}^N H_{L_{\alpha_j}} d_{\alpha_j} g_n -\H g_n\|^*_{1,\infty}.
\end{eqnarray*}
Given $\e >0$, we can make the left side $<\e$
by first choosing $n$ so that
$\| f-g_n\|^*_{1,\infty}<\frac{\e}{12 A}$ and
then choosing $N=N(n)$ so that 
$\| \sum_{j=1}^N H_{L_{\alpha_j}} d_{\alpha_j} g_n -
\H g_n\|^*_{1,\infty}=0$.  
This completes the proof.\\
%%%%%%%%%%%%%%%%%%%%%%%%%%%%%%

%%%%%%%%%%%%%%%%%%%%%%%%%%%%%%%%%%%%%%
{\bf The conjugate square function}  We end this section 
with a study of the square function
associated with the conjugate series (\ref{conjugate-series}).
We start with a definition.  For
$f\in L^p(G), \ 1\leq p<\infty$, let
%%%%%%%%%%%%%%%%%%%%%%%%%%%%%%
%%%%%%%%%%%%%%%%%%%%%%%%%%%%%%%%%%%%%%
\begin{equation}
\widetilde{S} f =\left( \sum_{\alpha\in \Pi} 
| H_{L_\alpha} ( d_\alpha f )|^2
\right)^\frac{1}{2},
\label{square-function}
\end{equation} 
%%%%%%%%%%%%%%%%%%%%%%%%%%%%%%
%%%%%%%%%%%%%%%%%%%%%%%%%%%%%%%%%%%%%%
 where the index of
 summation runs over those
 $\alpha$'s for which $d_\alpha f\not\equiv 0$.
 %%%%%%%%%%%%%%%%%%%%%%%%%%%%%%
%%%%%%%%%%%%%%%%%%%%%%%%%%%%%%%%%%%%%%
%%%%%%%%%%%%%%%%%%%%%%%%%%%%%%
%%%%%%%%%%%%%%%%%%%%%%%%%%%%%%%%%%%%%%
\begin{theorem-square-function}
(i)  Let $1<p<\infty$.  There is a constant 
$B_p$, depending only on $p$, such that
for all $f\in L^p(G)$, we have
$$\| \widetilde{S} f \|_p\leq B_p \| f \|_p.$$
%%%%%%%%%%%%%%%%%%%%%%%%%%%%%%%%%%%%%%
(ii)  There is an absolute constant $B$ such 
that, for all $f\in L^1(G)$, and all $y>0$,
we have
$$ \mu \left( \left\{ x\in G: \ \ | \widetilde{S} f (x) |
> y  \right\} \right) \leq \frac{B}{y} \| f \|_1 .
$$ 
\label{theorem-square-function}
\end{theorem-square-function}
%%%%%%%%%%%%%%%%%%%%%%%%%%%%%%
%%%%%%%%%%%%%%%%%%%%%%%%%%%%%%%%%%%%%%
%%%%%%%%%%%%%%%%%%%%%%%%%%%%%%
%%%%%%%%%%%%%%%%%%%%%%%%%%%%%%%%%%%%%%
{\bf Proof.}  Part (i) is a well-known consequence
of Theorem (\ref{unconditional-convergence}) (ii).  
We will omit the proof.  (In fact one can 
prove it by reproducing the argument that 
we present for part (ii)).  
%%%%%%%%%%%%%%%%%%%%%%%%%%%%%%%%%%%%%%
To prove (ii), let $p$ be an arbitrary but 
fixed number in $]0,1[$.  We will need Kintchine's 
Inequality (\cite[Theorem V. 8.4, p.213]{zyg},
which we will cite here in a notation
convenient for our proof.  Let 
$a_1 , a_2,\ldots , a_N$ be arbitrary 
complex numbers, and write $\E$ for the
expected value over the probability space
$\{-1,1\}^N$.  Then,
Kintchine's Inequality asserts that there are
constants $\alpha_p$ and $\beta_p$, 
depending only on $p$, such that
%%%%%%%%%%%%%%%%%%%%%%%%%%%%%%%%%%%%%%
%%%%%%%%%%%%%%%%%%%%%%%%%%%%%%%%%%%%%%
\begin{equation}
\alpha_p \left\{ \sum_{j=1}^N |a_j |^2
\right\}^\frac{1}{2}
\leq
\left\{  \E \left| \sum_{j=1}^N a_j \e_j \right|^p
\right\}^\frac{1}{p}
\leq
\beta_p \left\{ \sum_{j=1}^N |a_j |^2
\right\}^\frac{1}{2}.
\label{kintchine-inequality}
\end{equation}
%%%%%%%%%%%%%%%%%%%%%%%%%%%%%%%%%%%%%%
%%%%%%%%%%%%%%%%%%%%%%%%%%%%%%%%%%%%%%
Returning to the proof of (ii), we note 
by monotone convergence that it is enough
to consider a finite sum 
$$ \left( \sum_{j=1}^N 
| H_{L_{\alpha_j}} d_{\alpha_j} f|^2
\right)^\frac{1}{2}. $$
%%%%%%%%%%%%%%%%%%%%%%%%%%%%%%%%%%%%%%
Applying Kintchine's Inequality, we see that,
pointwise on $G$, we have
%%%%%%%%%%%%%%%%%%%%%%%%%%%%%%%%%%%%%%
$$
 \left( \sum_{j=1}^N 
| H_{L_{\alpha_j}} d_{\alpha_j} f|^2
\right)^\frac{1}{2}
\leq C_p 
 \left(  \E | \sum_{j=1}^N 
 \e_j H_{L_{\alpha_j}} d_{\alpha_j} f|^p
 \right)^\frac{1}{p}.  
 $$
%%%%%%%%%%%%%%%%%%%%%%%%%%%%%%%%%%%%%%
We think of each $\e\in \{-1,1\}^N$
as an element of $\{-1,1 \}^\Pi$ by setting
$\e(\alpha_j)=\e(j)$ for $j=1,2,\ldots,N$, and
$\e(\pi)=1$ for $\pi \not\in 
\{\alpha_1, \alpha_2,\ldots , \alpha_N\}$. 
Let $\eta=\eta(\e) $
be defined as in the proof of 
Corollary (\ref{finite-conjugate-projection}), 
(see (\ref{conjugate-eta-projection})) so that
$$\sum_{j=1}^N 
 \e_j H_{L_{\alpha_j}} d_{\alpha_j} f=
 \widetilde{{\cal P}}_{\eta (\epsilon), \e(P)}f.$$
%%%%%%%%%%%%%%%%%%%%%%%%%%%%%%%%%%%%%%
%%%%%%%%%%%%%%%%%%%%%%%%%%%%%%%%%%%%%%
%%%%%%%%%%%%%%%%%%%%%%%%%%%%%%%%%%%%%%
Then
%%%%%%%%%%%%%%%%%%%%%%%%%%%%%%%%%%%%%%
\begin{equation}
\|  \left( \sum_{j=1}^N 
| H_{L_{\alpha_j}} d_{\alpha_j} f|^2
\right)^\frac{1}{2} \|^*_{1,\infty}
\leq C_p 
 \| \left(  \E | \widetilde{{\cal P}}_{\eta(\e), \e(P)}f|^p
 \right)^\frac{1}{p} 
 \|^*_{1,\infty}.
\label{kintchine}
\end{equation}
%%%%%%%%%%%%%%%%%%%%%%%%%%%%%%%%%%%%%%
%%%%%%%%%%%%%%%%%%%%%%%%%%%%%%%%%%%%%%
It is easy to prove from definitions that,
for any $s>0$ and for any
measurable function $f$ on $G$,
$\| | f |^s \|^*_{p,\infty}= 
\| f  \|^{*\ s}_{s p,\infty}.$
%
%
%
The
fact that $\| \cdot\|^*_{\frac{1}{p},\infty}$
is equivalent to a norm (see (\ref{equivalent-lorentz})),
implies that 
$$\| \E f \|^*_{\frac{1}{p},\infty}\leq 
		\frac{1}{1-p}   \E\| f\|^*_{\frac{1}{p},\infty}.$$
%
We can now estimate the right side of (\ref{kintchine})
as follows
%%%%%%%%%%%%%%%%%%%%%%%%%%%%%%%%%%%%%%
\begin{eqnarray*}
 C_p 
 \| \left(  \E | \widetilde{{\cal P}}_{\eta(\e), \e(P)}f|^p
 \right)^\frac{1}{p} 
 \|^*_{1,\infty}
 		&=&
C_p \left( \|   \E | \widetilde{{\cal P}}_{\eta (\e), \e (P)}f|^p
  \|^*_{\frac{1}{p} ,\infty} \right)^\frac{1}{p}\\
 		&\leq&
 C_p \left(
 \frac{1}{1-p}
   \E \|   | \widetilde{{\cal P}}_{\eta(\e), \e(P)}f |^p
  \|^{*}_{\frac{1}{p} ,\infty} \right)^\frac{1}{p}\\
		&=&
C_p \left(
\frac{1}{1-p}
   \E \|    \widetilde{{\cal P}}_{\eta(\e), \e(P)}f 
  \|^{*\ p}_{1 ,\infty} \right)^\frac{1}{p}\\
		&\leq&
 C_p \left(
 \frac{2^p A^p}{1-p}
    \E \| f 
 \|^p_1 \right)^\frac{1}{p}\\
		&=&
  2 A C_p \left( \frac{1}{1-p}\right)^\frac{1}{p}
  \| f\|_1 .
\end{eqnarray*}
%
%
%
The penultimate inequality follows from 
Corollary (\ref{finite-conjugate-projection}).
This completes the proof of the theorem.
%%%%%%%%%%%%%%%%%%%%%%%%%%%%%%%%%%%%%%

%%%%%%%%%%%%%%%%%%%%%%%%%%%%%%%%%%%%%%%%%%%%%%%%%%%%%%%%%%%%%%%%%%%%%%%%%%%%%%%%%%%%%%%%%%%%%%%%%%%%%%%%%%%%%%%%%%%%%%%%%%%%%%%%%%%%%%%%%%%%%%%%%%%%%%%%%%%%%%%%%%%%%%%%%%%%%%%%%%%%%%%%%%%%%%%%%%%%%%%%%%%%%%%%%%%%%%%%%%%%%%%%%%%%%%%%%%%%%%%%%%%%%%%%%%%%%%%%%%%%%%%%%%%%%%%%%%%%%%%%%%%%%%%%%%%%%%%%%%%%%%%%%%%%%%%%%%%%%%%%%%%%%%%%%%%%%%%%%%%%%%%%%%%%%%%%%%%%%%%%%%%%%%%%%%%%%%%%%%%%%%%%%%%%%%%%%%%%%%%%%%%%%%%%%%%%%%%%%%%%%%%%%%%%%%%%%%%%%%%%%%%%%%%%%%%%%%%%%%%%%%
%%%%%%%%%%%%%%%%%%%%%%%%%%%%%%%%%%%%%%%%%%%%%%%%%%%%%%%%%%%%%%%%%%%%%%%%%%%%%






{\bf Acknowledgements}  The research of the authors was supported by grants from the National Science Foundation (U.\ S.\ A.) and the Research Board of the University of Missouri.  Both authors are grateful for conversations with Professors Nigel Kalton, and Saleem Watson.

\begin{thebibliography}{Dillo 83}

%\bibitem{asm1}  N.\ Asmar, 
%{\em The conjugate function on the 
%finite dimensional torus}, Can. Bull. 
%Math., {\bf 32}, No.2 (1989), 140--148.


%\bibitem{abg3}  N.\ Asmar, E.\ Berkson, and T.\ A.\ Gillespie,
%{\em Maximal estimates on groups, subgroups, 
%and the Bohr compactification}, 
%Journal of Functional Analysis, to appear.


\bibitem{ah}  N.\ Asmar, and E.\ Hewitt, 
{\em Marcel Riesz's Theorem on conjugate 
Fourier series and its descendants}, in ``Proceedings, 
Analysis Conference, Singapore, 1986'' 
(S.\ T.\ Choy {\it et al.}\ Eds.), 
Elsevier Science, New York, 1988, 1--56.

%\bibitem{bergil}  E.\ Berkson, and T.\ A.\ Gillespie,
%{\em The generalized M.\ Riesz Theorem and 
%transference}, Pac.\ J.\ Math., {\bf 120} (1985), 279--288.

%\bibitem{bochner} S.\ Bochner, 
%{\em Generalized conjugate analytic functions 
%without expansions}, Proc.\ Nat.\ Acad.\ Sci.\ (U.\ S.\ A.), 
%{\bf 45} (1959), 855--857.

%\bibitem{bdg}  D.\ L.\ Burkholder, 
%B.\ Davis, and R.\ F.\ Gundy, 
%{\em Integral inequalities for convex 
%functions of operators on martingales},
%in ``Proceedings of the Sixth Berkeley 
%Symposium on Mathematical Statistics 
%and Probability'', Vol II, University 
%of California Press, Berkeley and Los Angeles, 1970, 223--241.


%\bibitem{bg}  D.\ L.\ Burkholder, and R.\ F.\ Gundy,
%{\em Extrapolation and interpolation of quasi-linear operators on %martingales}, Acta Math.\ {\bf 124}\ (1970), 249--304.


%\bibitem{bgs}  D.\ L.\ Burkholder, R.\ F.\ Gundy, and M.\ L.\ Silverstein
%{\em A maximal characterization of the class $H^p$}, Trans.\ Amer.\ Math.\ %Soc., {\bf 157}\ (1971), 137--153.


\bibitem{cal}  A.\ Calder\'{o}n, 
{\em Ergodic theory and translation-invariant operators}, Proc.\ Nat.\ Acad.\ Sci., {\bf 157}\ (1971), 137--153.


\bibitem{cw}  R.\ R.\ Coifman, and G.\ Weiss,
``Transference methods in analysis'', Regional Conference 
Series in Math. {\bf 31}, Amer.\ Math.\ Soc., Providence, R.\ I.\, 1977.

\bibitem{cot}  M.\  Cotlar,  
{\em A unified theory of Hilbert transforms and ergodic theorems}, 
Rev. Mat. Cuyana, {\bf 1}\ (1955), 105--167.

%\bibitem{doob}  J.\ L.\ Doob,
%``Stochastic Processes'', Wiley Publications in Statistics, New York 1953. 

%\bibitem{doob2}    J.\ L.\ Doob,
%{\em Semimartingales and subharmonic functions}, Trans.\ Amer.\ Math.\ Soc., %{\bf 77}\ (1954), 86--121.

%\bibitem{feff}  R.\ Fefferman,  
%{\em Fourier analysis in several parameters}, 
%Revista Math. Iberoamericana, {\bf 2}\ (1980), 89--98.

\bibitem{fu}  L. Fuchs,
`` Partially ordered algebraic systems'', 
Pergamon Press, Oxford, New York, 1960.

\bibitem{gar} D.\ J.\ H.\ Garling,
{\em Hardy martingales and the 
unconditional convergence of martingales},
Bull. London Math. Soc. {\bf 23}\ (1991), 190--192.

\bibitem{hel1}  H.\ Helson,
{\em Conjugate series and a theorem of Paley}, 
Pac.\ J.\ Math., {\bf 8}\ (1958), 437--446.


\bibitem{hel2}  H.\ Helson,
{\em Conjugate series in several variables}, 
Pac.\ J.\ Math., {\bf 9}\ (1959), 513--523.

%\bibitem{hl}  H.\ Helson, and D.\ Lowdenslager,
%{\em Prediction theory and Fourier series in 
%several variables}, 
%Acta Math.\ {\bf 99}\ (1958), 165--202.

\bibitem{hk} E.\ Hewitt and S.\ Koshi, 
{\em Orderings in locally compact Abelian 
groups and the theorem of F.\ and M.\ Riesz}, 
Math.\ Proc.\ Cam.\ Phil.\ Soc.\ , {\bf 93}\ (1983), 441--457.

\bibitem{hr1} E.\ Hewitt and K.\ A.\ Ross,
``Abstract Harmonic Analysis I,''  
$2^{nd}$ Edition, Grundlehren der
Math.\ Wissenschaften, Band 115, Springer--Verlag, Berlin 1979.

%\bibitem{hr2} E.\ Hewitt and K.\ A.\ Ross,
%``Abstract Harmonic Analysis II,'' Grundlehren der
%Math.\ Wissenschaften in Einzeldarstellungen, Band
%152, Springer--Verlag, New York, 1970.

%\bibitem{hirroch}  I.\ I.\ Hirschmann,  and R.\ Rochberg, 
%{\em Conjugate function theory in weak * Dirichlet algebras}, 
%J.\ Functional Anal., {\bf 16}\ (1974), 359-371. 

%\bibitem{peter} K.\ E.\ Petersen, ``Brownian Motion,
%Hardy Spaces and Bounded Mean Oscillation,'' 
%London Math. Soc. Lecture Notes Series, No. 28, 
%Cambridge University Press, 1977.

%\bibitem{rud} W.\ Rudin, ``Fourier Analysis on Groups,''
%Interscience Tracts in Pure and Applied Mathematics, No. 12, 
%John Wiely, New York, 1962.

\bibitem{sw} E. \ Stein, and G.\ Weiss, 
``Introduction to Fourier Analysis 
on Euclidean Spaces,'' Princeton Math. Series, 
No. 32, Princeton Univ. Press, Princeton, N.\ J., 1971.

%\bibitem{yud}  V.\ A. \ Yudin, {\em On Fourier sums in}\ $L^p$,
%Proc. Steklov Inst. Math. {\bf 180} (1989) 279--280. 

\bibitem{zyg}  A.\ Zygmund,
`` Trigonometric series'', 2nd Edition, 2 vols.\ , 
Cambridge University Press, 1959.
\end{thebibliography}
\end{document}












